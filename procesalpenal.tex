Artículo 1°. Apruébase el adjunto proyecto de Código de Procedimiento Penal.

    Art. 2°. Dos ejemplares de una edicion esmerada i correcta que deberá hacerse inmediatamente, firmados por el Presidente de la República i signados con el sello del Ministerio de Justicia, se depositarán en la Secretaría de cada Cámara, dos en el archivo de dicho Ministerio, i otros dos en la Biblioteca Nacional.
    El testo de estos ejemplares se tendrá como el auténtico del Código de Procedimiento Penal, i a él deberán conformarse las ediciones que de éste se hicieren.

    Art. 3°. Concédese a don Luis Barriga la cantidad de cuatro mil pesos ($ 4,000) en remuneracion de los servicios que ha prestado como secretario de la Comision Mista encargada del estudio del proyecto de Código de Procedimiento Penal.

    I por cuanto, oido el Consejo de Estado, he tenido a bien aprobarlo i sancionarlo; por tanto, promúlguese i llévese a efecto como lei de la República.
    Santiago, trece de febrero de mil novecientos seis.- JERMAN RIESCO.- Guillermo Pinto Agüero.


  Libro Primero
  DISPOSICIONES GENERALES RELATIVAS AL JUICIO
CRIMINAL
  Título I
  DE LA JURISDICCION Y COMPETENCIA EN MATERIA PENAL

    Artículo 1°. Los tribunales de la República ejercen jurisdicción sobre los chilenos y sobre los extranjeros para el efecto de juzgar los delitos que se cometan en su territorio, salvo los casos exceptuados por leyes especiales, tratados o convenciones internacionales en que Chile es parte o por las reglas generalmente reconocidas del Derecho Internacional.

NOTA:  1
    Veánse el Código de Derecho Internacional Privado y las Convenciones de Viena, de 18 de abril de 1961, sobre Relaciones Diplomáticas, y de 24 de abril de 1963, sobre Relaciones Consulares.
NOTA:  1.1.-
    Las modificaciones introducidas al presente Código por la Ley N° 18.857, publicada en el Diario Oficial de 6 de Diciembre de 1989, rigen, según lo dispone su artículo vigésimo, noventa días después de su publicación en el Diario Oficial.
    Artículo 2°. No se aplicarán en el territorio nacional las leyes penales y de procedimiento de otros países, sin perjuicio de su consideración previa cuando sea necesaria para determinar la aplicación de las leyes patrias.

    Art. 3°.- La ejecución de las sentencias en materia criminal se efectuará en la forma que para cada caso esté indicada en el Código Penal, sin perjuicio de lo  establecido en el Libro IV de este Código.
    Las sentencias extranjeras no se ejecutarán en Chile, en cuanto impongan penas.
    Sin embargo, si la sentencia penal extranjera recae sobre crímenes o simples delitos perpetrados fuera del territorio de la República que queden sometidos a la jurisdicción chilena, la pena o parte de ella que el procesado hubiere cumplido en virtud de tal sentencia, se computará en la que se le impusiere de acuerdo con la ley nacional, si ambas son de similar naturaleza y, si no lo son, se atenuará prudencialmente la pena.
    Tendrá también valor en Chile el fallo condenatorio extranjero para determinar la calidad de reincidente o delincuente habitual del procesado.
    La sentencia absolutoria pronunciada en el extranjero tendrá valor en Chile para todos los efectos legales, a menos que recaiga sobre algún delito cometido en el territorio nacional o en los demás lugares sometidos a la jurisdicción chilena, o sobre alguno  cometido en el extranjero y que deba juzgarse en Chile.
    Lo dicho en este artículo es sin perjuicio de lo dispuesto en normas especiales.


    Art. 4° (23) Siempre que para el juzgamiento criminal se requiera la resolución previa de una cuestión civil de que deba conocer otro tribunal, el juicio criminal no se adelantará sino para practicar aquellas diligencias del sumario necesarias a la comprobación de los hechos; y se paralizará en seguida hasta que sea fallada la cuestión civil.
    En el juicio civil prejudicial intervendrá el Ministerio Público, cuando la causa criminal verse sobre delito que deba perseguirse de oficio, para hacer todas las gestiones conducentes a la iniciación o a la pronta terminación de dicho juicio.
    Podrá también hacerse parte principal cuando lo estime conveniente.

NOTA:  2
    Véase, el DFL N° 426, de 28 de febrero de 1927, publicado en el Diario Oficial de 3 de marzo del mismo año.
    Art. 5° (24) Pueden ejercitarse separadamente ante el tribunal civil correspondiente las acciones para perseguir las responsabilidades civiles provenientes del hecho punible, salvo la que tenga por objeto la mera restitución de un cosa, que deberá ser deducida, precisamente, ante el juez que conozca del respectivo proceso penal.
    Cuando la acción civil se ejercite separadamente de la penal, aquélla podrá quedar en suspenso desde que el procedimiento criminal pase al estado de plenario, y se observará lo dispuesto en el artículo 167 del Código de Procedimiento Civil.

    Artículo 6°.- Cualquiera que sea el tribunal llamado a conocer de un juicio criminal, los jueces letrados con competencia penal y los demás jueces que tengan esta competencia, aunque sólo sea respecto de delitos menores, faltas o contravenciones, están obligados a practicar las primeras diligencias de instrucción del sumario con respecto a los delitos cometidos en el territorio de su jurisdicción, sin perjuicio de dar inmediato aviso al tribunal a quien por ley corresponda el conocimiento de la causa.
    INCISO SEGUNDO.- DEROGADO.-

    INCISO TERCERO.- DEROGADO.-

    Art. 7° (27) Considéranse como primeras diligencias: dar protección a los perjudicados, consignar las pruebas del delito que puedan desaparecer, recoger y poner en custodia cuanto conduzca a su comprobación y a la identificación de los delincuentes, decretar el arraigo de los inculpados cuando proceda y detenerlos en su caso, procediendo a la detención con arreglo a lo dispuesto en los párrafos 2° y 5° del Título IV, Primera Parte del Libro Segundo y resolver sobre la libertad de los detenidos.
    Para estos efectos, el juez de prevención dispondrá la atención prioritaria del ofendido por los servicios públicos pertinentes, decretará su resguardo policial o el de los testigos, interrogará a estos últimos y a los inculpados, y practicará los careos y reconocimientos que fueren necesarios.

    Artículo 7º bis.- Sin perjuicio de lo dispuesto en los artículos 6º y 7º precedentes, la Corte de Apelaciones respectiva establecerá un sistema de jueces de turno para atender las primeras diligencias de la instrucción, durante los días y horas en que no funcionan los tribunales, respecto de delitos cuyo conocimiento no se encontrare radicado en el tribunal competente.
    En dichos turnos, se incorporará a los secretarios de los juzgados con competencia en materia penal, quienes se entenderán habilitados para desempeñar tales funciones por el solo ministerio de la ley.
    El sistema de turno será semanal, excepto en aquellas localidades donde sólo exista un juez con competencia en materia penal, caso en el cual podrá establecerse una modalidad diversa.
    Las actuaciones, providencias o comunicaciones del juez de turno serán válidas para todos los efectos legales, sin la intervención de ministro de fe.
    Cuando resultare necesaria la constitución del juez de turno en el sitio del suceso, en el recinto del tribunal o en un recinto policial, se encontrará habilitado para ausentarse al día siguiente hábil, en el despacho del tribunal, el número de horas que hubiere ocupado en dicho procedimiento.
    La Corporación Administrativa del Poder Judicial informará anualmente a las Cortes de Apelaciones y al Ministerio de Justicia respecto de la aplicación que hubiese tenido el sistema de turno y de las disponibilidades presupuestarias para el año siguiente.
    En el ejercicio de sus facultades, la Corte Suprema, mediante auto acordado, podrá dictar instrucciones generales para el buen funcionamiento del sistema a que se refiere este artículo.


    Artículo 8°.- Los jueces de letras deberán practicar todas las diligencias que les cometan otros tribunales para la investigación de los hechos en materias criminales, sin que sea menester que la orden emane del superior jerárquico respectivo.
    Los jueces del crimen que conozcan de uno de los delitos tipificados en los artículos 346 a 372 del Código Penal, en que sea víctima un menor, deberán poner el hecho en conocimiento del juez de menores competente, a fin de que pueda dictar, si procediere, alguna medida de protección en su favor.

    Art. 9° (29) La competencia criminal no puede, en caso alguno, ser prorrogada por la simple voluntad de las partes.


  Título II
  DE LA ACCION PENAL Y DE LA ACCION CIVIL EN EL
  PROCESO PENAL.


    Art. 10. (30) Se concede acción penal para impetrar la averiguación de todo hecho punible y sancionar, en su caso, el delito que resulte probado. En el proceso penal podrán deducirse también, con arreglo a las prescripciones de este Código, las acciones civiles que tengan por objeto reparar los efectos civiles del hecho punible, como son, entre otras, las que persigan la restitución de la cosa o su valor, o la indemnización de los perjucios causados.
    En consecuencia, podrán intentarse ante el juez que conozca del proceso penal las acciones civiles que persigan la reparación de los efectos patrimoniales que las conductas de los procesados por sí mismas hayan causado o que puedan atribuírseles como consecuencias próximas o directas, de modo que el fundamento de la respectiva acción civil obligue a juzgar las mismas conductas que constituyen el hecho punible objeto del proceso penal.

    Art. 11. (31) La acción penal es pública o privada. La primera se ejercita a nombre de la sociedad para obtener el castigo de todo delito que deba perseguirse de oficio; la segunda sólo puede ejercitarse por la parte agraviada.
    Se concede siempre acción penal pública para la persecución de los delitos previstos en los artículos 361 a 366 quinquies del Código Penal, cometidos contra menores de edad.
    Artículo 12.- Cuando se ejercite sólo la acción civil respecto de un hecho punible que no puede perseguirse de oficio, se considerará extinguida por ese hecho la acción penal.

    Art. 13. (33) Cuando el acusado hubiere sido condenado en el juicio criminal como responsable del delito, no podrá ponerse en duda, en el juicio civil, la existencia del hecho que constituya el delito, ni sostenerse la inculpabilidad del condenado.
    Art. 14. (34) Extinguida la acción civil no se entiende extinguida por el mismo hecho la acción penal para la persecución del hecho punible.
    La sentencia firme absolutoria dictada en el pleito promovido para el ejercicio de la acción civil, no será obstáculo para el ejercicio de la acción penal correspondiente cuando se trate de delitos que deban perseguirse de oficio.

    Art. 15. (35) La acción penal pública puede ser ejercida por toda persona capaz de parecer en juicio, siempre que no tenga especial prohibición de la ley y que se trate de delitos que deban ser perseguidos de oficio.
    Art. 16. (36) No puede ejercitar la acción pública penal:
    1° El que fuere criminal o civilmente responsable del delito materia del proceso;
    2° El procesado o condenado por delito de igual o mayor gravedad que aquel de que se trata; y
    3° El que ha perjurado o recibido paga por acusar, en el mismo juicio o en otro distinto.
    Pueden, sin embargo, las personas designadas en los números 2° y 3°, ejercitar la acción pública por delitos cometidos contra ellas o contra sus ascendientes, descendientes, o hermanos legítimos o ilegítimos.
    Art. 17. (37) Tampoco pueden ejercitar entre sí acción penal, sea pública o privada:
    1° Los cónyuges; a no ser que por delito que el uno hubiere cometido contra la persona del otro o contra la de sus hijos, o por el delito de bigamia.

    2° Los consanguíneos legítimos o naturales en toda la línea recta, los colaterales hasta el 4° grado ni los afines hasta el 2°; a no ser por delitos cometidos por los unos contra la persona de los otros, o la de su cónyuge o hijos.

NOTA:  1.2
    El Artículo 37 de la Ley N° 19.335, publicado en el "Diario Oficial" de 23 de Septiembre de 1994, dispuso que la modificación introducida al presente artículo, rige transcurridos tres meses de su publicación en el Diario Oficial.
    Art. 18. (38) No podrán ser ejercidas por el Ministerio Público ni por otra persona que no fuere la ofendida o su representante legal, las acciones que nacen de los delitos siguientes:
    1.- Derogado

    2.- La comunicación fraudulenta de secretos de la fábrica en que el culpable ha estado o está empleado;
    3.- Derogado

    4.- Derogado
    5.- Derogado

    6.- El matrimonio del menor llevado a efecto sin el consentimiento de las personas designadas por la ley y celebrado de acuerdo con el funcionario llamado a autorizarlo; acción que se entiende abandonada cuando la acusación no se entablare en el término de dos meses después de tenerse noticia de la celebración del matrimonio;
    7.- La provocación a duelo y el denuesto o descrédito público inferido a otro por no haberlo aceptado;
    8.- La calumnia y la injuria contra personas privadas, delitos que pueden, además, ser perseguidos por el cónyuge, los hijos, nietos, padres, abuelos y hermanos legítimos y por los hijos y padres naturales del ofendido, que se encuentre moral o físicamente imposibilitado.  Si ha muerto el ofendido, las mismas personas, y además de sus herederos, pueden deducir las acciones correspondientes, y
    9.- La falta descrita en el número 11 del artículo 496 del Código Penal.

    Artículo 19.- DEROGADO

    Artículo 20.- Los empleados públicos tienen derecho a exigir que se entable acción para que se persiga la responsabilidad por las injurias y calumnias de que se les hiciere objeto con motivo del desempeño de sus funciones, en la forma prevista en el Estatuto Administrativo.
    Si no les fuera aplicable ese Estatuto, deberá deducirse la acción por el Ministerio Público, a requerimiento de la persona ofendida.
    Los agentes diplomáticos extranjeros acreditados ante el Gobierno de la Nación tienen el derecho indicado en este artículo, aun respecto de las calumnias o injurias que les fueren inferidas en su carácter privado. El requerimiento al Ministerio Público deberá hacerlo el propio afectado.
    Deducida la denuncia o querella, el procedimiento se seguirá de acuerdo con las reglas del juicio ordinario de acción pública.
    Lo dicho en los incisos anteriores no obsta lo dispuesto en leyes especiales.
    Para los efectos indicados en este artículo, actuará, aun en la primera instancia, el fiscal de la Corte de Apelaciones correspondiente.

    Art. 21. (41) Si varias personas no exceptuadas pretendieren ejercer la acción pública con respecto a un mismo delito, podrán hacerlo procediendo conjuntamente por medio de un mandatario común.
    Pero serán preferidas las personalmente ofendidas por el delito, si procedieren también conjuntamente. Si estas personas fallecieren o desistieren de la prosecución del juicio, revivirá el derecho de aquéllas, quienes podrán intervenir en el juicio tomándolo en el estado en que lo encontraren.
    Art. 22. (42) El que ejercita la acción pública está obligado a afianzar las resultas del juicio.
    Art. 23. (43) Los oficiales del Ministerio Público tienen obligación de ejercer la acción pública con respecto a todo delito que deba perseguirse de oficio. Si el delito es de aquellos que, para ser perseguido, necesita denuncia o requisición previa de la persona ofendida, la acción pública debe ponerse en ejercicio tan pronto como se presente la denuncia o requisición.

    Art. 24. (44) Siempre que se trate de delitos que deban perseguirse de oficio, los tribunales competentes estarán obligados a proceder, aun cuando el Ministerio Público no crea procedente la acción.
    En general, tienen los tribunales perfecta libertad para aceptar o rechazar las peticiones del Ministerio Público.
    Artículo 25.- La intervención del querellante que ha ejercitado la acción pública no obsta a la del Ministerio Público, ni la de éste a la de aquél. Sin embargo, el querellante o las otras partes del juicio no podrán oponerse a las diligencias probatorias que solicite el Ministerio Público.

    Art. 26. (46) El oficial del Ministerio Público de un tribunal superior encargado de rever el fallo del tribunal inferior, puede continuar el ejercicio de la acción pública ante el tribunal cerca del cual funciona, no obstante que el oficial del tribunal inferior haya aceptado expresa o tácitamente aquel fallo.
    Artículo 26 bis.- Los fiscales de las Cortes de Apelaciones podrán intervenir en la primera instancia en todos los juicios criminales de acción pública, cuando juzguen conveniente su actuación.
    El Fiscal de la Corte Suprema podrá ordenar a los Fiscales de las Cortes de Apelaciones que actúen en la primera instancia para efectos determinados o durante toda la tramitación de uno o más juicios, y en este último caso tendrán las mismas facultades y obligaciones que este Código señala al Ministerio Público en dicha instancia.
    Siempre que un Ministro, en carácter de juez de primera instancia, deba conocer de uno o más delitos, se entenderá designado para actuar en todas las instancias del proceso el Fiscal de la Corte correspondiente.

    Art. 27. (47) Es prohibido a los oficiales del Ministerio Público renunciar de antemano, expresa o tácitamente, al ejercicio de la acción pública, en los casos en que ella es procedente.

    Art. 28. (48) La acción penal pública no se extingue por la renuncia de la persona ofendida.
    Pero se extinguen por esa renuncia la acción penal privada y la civil derivada de cualquiera clase de delitos.
    Si el delito no puede ser perseguido sin previa denuncia o requisición, cualquiera que no sea el Ministerio Público puede renunciar al derecho de hacer la denuncia o la requisición; y en tal caso queda también extinguida la acción pública.
    Art. 29. (49) La renuncia de una acción civil o penal renunciable sólo afectará al renunciante y a sus sucesores, y no a otras personas a quienes también correspondiere una u otra acción.
    Art. 30. (50) El querellante podrá desistirse de la acción penal, sea ésta pública o privada.
    Si la acción fuere pública, el juicio seguirá adelante, constituyéndose el Ministerio Público en parte principal, a falta de otro acusador particular.
    Si la acción fuere privada, podrá, además, ponerse término al juicio mediante una transacción. Pero el desistimiento o la transacción no producirá en ningún caso el efecto de que se devuelva la multa que hubiere sido satisfecha por vía de pena.
    Art. 31. (51) El querellante que se desistiere del ejercicio de la acción pública no quedará por eso exento de la obligación de comparecer al tribunal cuando el juez lo creyere necesario para la instrucción del proceso.
    Art. 32. (52) El desistimiento de la acción privada producirá el sobreseimiento definitivo de la causa, cualquiera que sea el estado en que se encontrare; y el tribunal condenará al querellante al pago de las costas.


    Art. 33. (53) No se dará lugar al desistimiento de la acción privada si el querellado se opusiere a él.
    Art. 34. (54) El desistimiento de la acción pública o privada deja a salvo el derecho del querellado para ejercitar, a su vez, contra el querellante la acción penal o civil a que dieren lugar la querella o acusación calumniosa, y los perjuicios que les hubiere causado en su persona o bienes.
    Se exceptúa el caso de que el querellado haya aceptado expresa o tácitamente el desistimiento del querellante.
    Art. 35. (55) Es aplicable al desistimiento de una querella o acusación deducida, lo dispuesto en el artículo 29 con respecto a la renuncia de la acción civil o penal que aún no se ha hecho valer en juicio.
    Art. 36. (56) El Ministerio Público no podrá desistirse de la querella o acusación intentada; pero podrá pedir, a su tiempo, el sobreseimiento o la absolución del procesado cuando así lo estimare de derecho.




    Artículo 37.- La acción penal pública se suspende, con arreglo al Derecho Internacional:
    1.- Cuando el inculpado es entregado a los tribunales de la República por la vía de la extradición y la convención diplomática ha limitado los efectos de la persecución;
    2.- Cuando, entregado el procesado por un delito, se trata de procesarlo además por otro delito diferente del que ha motivado la extradición; y
    3.- Cuando el inculpado es arrestado a bordo de un buque que ha hecho arribada forzosa bajo bandera amiga o neutral.
    En este último caso no se suspende el procedimiento iniciado contra individuos que, cubiertos con aquella bandera, se encuentren en hostilidad contra el Gobierno de la República, o que hayan sido inculpados de crímenes o simples delitos contra la seguridad exterior o interior del Estado.

    Art. 38. (58) Muerto el querellante, sus herederos no están obligados a continuar el juicio; pero no quedan exentos de la responsabilidad civil que haya podido contraer el difunto respecto del querellado.
    Art. 39 (59) La acción penal, sea pública o privada, no puede dirigirse sino contra los personalmente responsables del delito o cuasidelito.
    La responsabilidad penal sólo puede hacerse efectiva a las personas naturales. Por las personas jurídicas responden los que hayan intervenido en el acto punible, sin perjuicio de la responsabilidad civil que afecte a la corporación en cuyo nombre hubieren obrado.
    Artículo 40.- La acción civil puede entablarse contra los responsables del hecho punible, contra los terceros civilmente responsables y contra los herederos de unos y otros.

    Art. 41. (62) Sin perjuicio de lo dispuesto en el presente título, la extinción de la responsabilidad penal, la prescripción de la acción civil y de la penal, y la prescripción de la pena, se regirán respectivamente por las reglas establecidas en el artículo 2332 del Código Civil, y en el Título V del Libro I del Código Penal.
    En cuanto a la prescripción de la acción civil, se estará además a lo dispuesto en los artículos 103 bis y 450 bis.


  Título III
  REGLAS APLICABLES A TODO JUICIO CRIMINAL
  1. Aplicación de la ley.


    Artículo 42.- A nadie se considerará culpable de delito ni se le aplicará pena alguna sino en virtud de sentencia dictada por el tribunal establecido por la ley, fundada en un proceso previo legalmente tramitado; pero el imputado deberá someterse a las restricciones que con arreglo a la ley se impongan a su libertad o a sus bienes durante el proceso.
    El procesado condenado, absuelto o sobreseído definitivamente por sentencia ejecutoriada, no podrá ser sometido a un nuevo proceso por el mismo hecho, sin perjuicio de lo dispuesto en el artículo 3°, inciso tercero, y en los Título III y VII del Libro III.


    Artículo 42 bis.- No se podrá citar, arrestar, detener, someter a prisión preventiva, separar de su domicilio o arraigar a ningún habitante de la República, sino en los casos y en la forma señalados por la Constitución y las leyes y sólo en estas mismas condiciones se podrá allanar edificios o lugares cerrados, interceptar, abrir o registrar comunicaciones y documentos privados.

    Artículo 43.- Son aplicables al procedimiento penal, en cuanto no se opongan a lo establecido en el presente Código o en leyes especiales, las disposiciones comunes a todo procedimiento, contenidas en el LIbro I del Código de Procedimiento Civil.

  2. Reglas generales sobre el proceso.


    Artículo 44.- No hay días ni horas inhábiles para las actuaciones del proceso, ni se suspenden los términos por la interposición de días feriados. No obstante, cuando un plazo de días concedido a las partes para recurrir o hacer uso de cualquier derecho, aunque sea término fatal, venza en día feriado, se considerará ampliado el término hasta las doce de la noche del día siguiente hábil.
    El feriado judicial establecido en el Código Orgánico de Tribunales no es aplicable al procedimiento criminal.




NOTA
      La Ley 20774, publicada el 04.09.2014, modificó diversas disposiciones legales con el fin de suprimir el feriado judicial y, en particular, el Nº 3 de su Art. 1º lo eliminó del Art. 313 del Código Orgánico de Tribunales, a que se refiere el presente artículo. Conforme a su Art. 7 las referencias al feriado judicial de febrero consignadas en cualquier cuerpo legal que no se encuentren previstas expresamente en ella, se entenderán derogadas para todos los efectos legales.
    Art. 45. (66) Son improrrogables los términos en los juicios criminales, cuando la ley no disponga expresamente lo contrario.
    Pero podrán suspenderse o abrirse de nuevo, cuando, sin retroceder el juicio del estado en que se halle, se pruebe la existencia de una causa que haya hecho imposible dictar la resolución o practicar la diligencia judicial, independientemente de la voluntad de quienes hubieren debido hacerlo.
    Art. 46. (67) Es obligación de los respectivos ministros de fe practicar las notificaciones y demás diligencias que les fueren encomendadas para dentro del recinto de la ciudad en que tiene su asiento el tribunal, a más tardar el día siguiente a aquel en que hubieren recibido el encargo.
    Las diligencias que hubieren de practicarse fuera de las ciudades deberán ser despachadas a más tardar dentro del tercero día.
    Art. 47. (68) Si se suscitare cuestión de competencia entre varios jueces para conocer o no conocer en una misma causa criminal, mientras no sea dirimida dicha competencia, todos ellos están obligados a practicar dentro del territorio de su respectiva jurisdicción, las primeras diligencias que se expresan en el artículo 7°.
    De los jueces en cuyo territorio jurisdiccional estuvieren detenidos los reos, resolverá acerca de la libertad provisional de éstos.



NOTA:
      El artículo 9° de la Ley 19047, publicada el 14.02.1991, modificado por la Ley 19158, otorga facultad para mantener la palabra reo por estar empleada en sentido genérico.
    Art. 48. (69) Dirimida la competencia, serán puestos inmediatamente a disposición del juez competente los reos y los antecedentes que obraren en poder de los demás jueces entre quienes se suscitó la contienda.
    Todas las actuaciones practicadas ante los jueces que resultaren incompetentes, serán válidas sin necesidad de que se ratifiquen ante el juez que fue declarado competente.



NOTA:
      El artículo 9° de la Ley 19047, publicada el 14.02.1991, modificado por la Ley 19158, otorga facultad para mantener la palabra reo por estar empleada en sentido genérico.
    Art. 49. (70) Recusado un juez o reclamada su implicancia pasará el conocimiento del negocio al llamado por la ley a subrogarlo, mientras se tramita y resuelve el incidente de implicancia o recusación. Pero el subrogante se limitará a practicar las primeras diligencias a que se refiere el artículo 7° y a dictar las providencias urgentes mientras penda el incidente.
    Recusado uno o más miembros de un tribunal de alzada, o reclamada su implicancia, los demás miembros continuarán en el conocimiento del negocio hasta que se resuelva el artículo o hasta que la causa se ponga en estado de sentencia definitiva.
    Art. 50. (71) En los procesos criminales las providencias se expedirán el mismo días en que presente la solicitud en que recaen; y los autos, a más tardar, el día siguiente.
    Las sentencias definitivas se pronunciarán dentro de los cinco días siguientes a aquel en que la causa quede en estado de fallo. Pero si el expediente constare de más de cien fojas, el plazo para fallar se extenderá a un día más por cada veinticinco fojas, sin que en ningún caso el plazo total pueda exceder de quince días.
    Artículo 51.- Los secretarios de los juzgados del crimen proveerán por sí solos las solicitudes de mera tramitación que no requieran conocimiento de los antecedentes para ser proveídas.
    Las rebeldías de trámites deberán ser declaradas por el secretario del juzgado de oficio o a petición de parte, según proceda.

    Art. 52. Las órdenes de citación a testigos o a inculpados, las que se den a la Prefectura respectiva o a Carabineros para que procedan a practicar investigaciones; los oficios que se envíen para pedir datos o antecedentes; el cúmplase de los exhortos de otros tribunales; el acuse de recibo de estos mismo exhortos y las órdenes necesarias para cumplirlos cuando no se encargue una detención o prisión, serán firmados únicamente por el secretario del juzgado, siempre que todas estas actuaciones emanen de resoluciones previas dictadas por el tribunal y estampadas en el expediente.
    En los casos de este artículo y en los indicados en el anterior la firma del secretario no necesita ser autorizada por ningún funcionario y deberá anteponérsele las palabras "por el Juez".
    Si se discuten las órdenes firmadas por el secretario de conformidad con las facultades que precedentemente se le otorgan, resolverá el juez sin ulterior recurso.
    Art. 53. Los jueces del crimen podrán subscribir con su media firma las actuaciones en que intervengan o las resoluciones que expidan, siempre que no se trate de decretos de detención, autos de procesamiento, autos acusatorios, autos de sobreseimiento definitivo o temporal y sentencias, los que deberán llevar la firma entera del magistrado que los dicte.
    Dentro de los quince días siguientes a la fecha en que se hagan cargo de sus puestos, los jueces letrados oficiarán a la Corte de Apelaciones respectiva, dándole cuenta de la media firma que usarán en el desempeño de sus funciones.

    Art. 53 bis.- Cuando el Juez de la causa estime necesario agregar al proceso documentos que tengan el carácter de secretos de acuerdo a las disposiciones del Código de Justicia Militar, procederá en conformidad a lo preceptuado en los artículos 144 y 144 bis de dicho Código.

    Artículo 53 bis A.- En todo proceso penal en que se exija juramento a los testigos, peritos u otras personas, se permitirá que formulen una promesa con las mismas solemnidades exigidas a aquél. La violación de esta promesa producirá los efectos que las leyes señalan a la violación del juramento.

    Artículo 53 bis B.- No se requerirá la intervención de los representantes legales para que los incapaces presten declaración como inculpados o testigos, ni para que sean procesados o participen en los demás actos del procedimiento criminal, sin perjuicio de las normas relativas a su responsabilidad civil.

    Artículo 54.- En general, el derecho a recurrir en contra de una resolución judicial corresponde al agraviado por ella.
    El Ministerio Público puede también recurrir en favor del inculpado o procesado. Puede además intervenir en cualquier estado de todo recurso deducido por las otras partes del juicio, a fin de impetrar las soluciones que estime conforme con la ley y las finalidades del proceso penal.



    Artículo 54 bis.- Son apelables las sentencias definitivas de primera instancia en causa criminal y las interlocutorias del mismo grado que pongan término al juicio o hagan imposible su continuación.
    Lo son también las demás resoluciones respecto de las cuales la ley concede el recurso y, en general, las que causen gravamen irreparable.
    La adhesión a la apelación sólo será admisible en los casos contemplados en el inciso primero y dentro del plazo a que se refiere el artículo 513.

    Art. 55. (77) Todo recurso contra una resolución judicial debe interponerse dentro de cinco días, si la ley no fijare un término especial para deducirlo.
    No obstante, el tribunal, de oficio o a petición de parte, podrá, en cualquier tiempo, rectificar las sentencias en los casos previstos en el artículo 182 del Código de Procedimiento Civil, especialmente si se han cometido errores en la determinación del tiempo que el procesado ha permanecido detenido o en prisión preventiva.


    Artículo 56.- De las sentencias interlocutorias, de los autos y de los decretos puede pedirse reposición al juez que los pronunció.
      La reposición sólo puede solicitarse dentro de tercero día y para ser admitida  deberá estar siempre fundada.
    El tribunal se pronunciará de plano, pero podrá conferir traslado si se ha deducido contra una sentencia interlocutoria o en un asunto cuya complejidad aconseje oír a la otra parte.
    Cuando la reposición se interponga respecto de una resolución que también es suceptible de apelación y no se deduzca a la vez este recurso para el caso de que la reposición sea denegada, se entenderá que la parte renuncia a la apelación.
    La reposición no tiene efecto suspensivo, salvo cuando contra la misma resolución proceda también la apelación en este efecto.

    Art. 57. (79) Son inapelables las sentencias pronunciadas en segunda instancia, a menos que tengan por objeto resolver acerca de la competencia del mismo tribunal.
    Art. 58. (80) Contra las resoluciones dictadas por la Corte Suprema no se da otro recurso que el de revisión, en su caso.
    Art. 59. (81) El recurso deberá entablarse ante el mismo tribunal que hubiere pronunciado la resolución, y éste lo concederá o lo negará según lo estimare procedente.
    Art. 60. (82) Por regla general, la apelación se concederá en ambos efectos, salvo que la ley disponga expresamente lo contrario para casos determinados, o que por hallarse el juicio en estado de sumario, pudiere entorpecerse la investigación a causa del recurso. En tales casos, la apelación será otorgada en el solo efecto devolutivo.
    Artículo 61.- Cuando se otorgue el recurso en ambos efectos, se remitirán los autos originales al tribunal de alzada, dentro del día siguiente al de la última notificación.
    Si el recurso fuere otorgado en el solo efecto devolutivo, el juez ordenará, según convenga a la rapidez y eficacia del proceso, su elevación en original, dejando las copias indispensables para continuar la tramitación, o bien la remisión de las compulsas necesarias para el conocimiento del recurso.
    Las compulsas serán hechas por la Secretaría. Para confeccionarlas, podrán adicionarse copias mecanografiadas, fotocopiadas o reproducidas de otra manera semejante, que estén en poder del tribunal o proporcionen las partes, de escritos, de documentos o de otras piezas del proceso, siempre que dichas copias se encuentren debidamente autentificadas por el Secretario. El juez señalará un plazo para hacer las compulsas, el que no podrá exceder de cinco días.
    En los casos a que se refiere este artículo no se aplicará lo dispuesto en el artículo 197 del Código de Procedimiento Civil.
    En uno y otro caso se adoptarán las precauciones necesarias para que se mantengan en secreto los antecedentes reservados.

    Artículo 62.- Denegado el recurso o concedido siendo improcedente u otorgado en el solo efecto devolutivo o en los efectos devolutivo y suspensivo, pueden las partes ocurrir de hecho ante el tribunal que debe conocer de la apelación, con el fin de que resuelva si ha lugar o no el recurso deducido o si debe ser otorgado en ambos efectos o en uno solo.
    El recurso de hecho se fallará en cuenta con los autos originales, si están en la Secretaría del tribunal o se pidieren para decidirlo, o con el informe del juez.


    Artículo 62 bis.- El querellante y las partes civiles no podrán suspender el conocimiento de las apelaciones o las consultas relativas a la libertad provisional de los inculpados o procesados, y sólo por razones que calificará el tribunal podrán suspender la vista en asuntos incidentales cuando hay algún detenido o procesado preso en la causa.
    Si en juicio criminal se recusa a un abogado integrante, el Presidente de la Corte deberá proveer a su inmediato reemplazo, para la misma audiencia, por un ministro u otro integrante.





NOTA 1.1
    Las modificaciones introducidas al presente Código por la Ley N° 18.857, publicada en el Diario Oficial de 6 de Diciembre de 1989, rigen, según lo dispone su artículo vigésimo, noventa días después de su publicación en el Diario Oficial.
    Art. 63. Las apelaciones y los recursos de casación se verán ante los tribunales que deben conocer de ellos sin esperar la comparecencia de las partes. En consecuencia no tendrán aplicación en los recursos de apelación y casación en materia penal lo dispuesto en el artículo 200 del Código de Procedimiento Civil.
    No se notificará a las partes que no hayan comparecido a la instancia las resoluciones que se dicten, las cuales producirán sus efectos respecto de ellas desde que se pronuncien.
    Cuando en un mismo expediente principal o de compulsas hubiere varias apelaciones en estado de ser vistas, las partes se considerarán emplazadas respecto de todas, las que serán conocidas conjuntamente.  No regirá esta regla en la vista de asuntos agregados extraordinariamente a la tabla, respecto de otros que deban figurar en ella en la forma común.
    En caso que proceda la vista de la causa y siempre que por ministerio de la ley o resolución judicial uno o más abogados tengan conocimiento del proceso o del cuaderno de éste en que incida el recurso, se estará a lo que establece el artículo 223 del Código de Procedimiento Civil, solamente respecto del o de los abogados que tengan dicho conocimiento.

    Artículo 63 bis.- En las apelaciones incidentales, sólo se admitirá nueva prueba documental, siempre que sea agregada antes de la vista de la causa.
    Los autos originales sólo se pedirán para resolver el recurso y, en tal caso, no se retendrán por más de dos días hábiles y uno más por cada cien fojas.
    Para decidir las apelaciones en estos asuntos, la Corte o Sala podrá solicitar de otros tribunales, aun telefónicamente, por sí o por medio del secretario o del relator, el envío de expedientes o documentos, o pedir informes escritos o verbales a los jueces o a los funcionarios auxiliares de la jurisdicción sobre datos de interés para la decisión.
    Podrá también llamar al procesado para interrogarlo, a cualquier empleado judicial que sirva dentro del territorio jurisdiccional para que dé las explicaciones o informaciones que se le soliciten, y a los policías y peritos que hayan actuado, con el mismo objeto.
    Estará asimismo facultada para trasladarse a cualquier tribunal u oficina del orden judicial o a establecimientos carcelarios o policiales con el objeto de hacer indagaciones, revisar libros, documentos, especies o locales, cuando ello fuere necesario o útil para la decisión del asunto o para establecer la corrección o incorrección de los procedimientos.
    Se podrá comisionar a uno de los miembros de la Corte o Sala para los efectos señalados en los dos incisos precedentes.
  Si se descubriere alguna infracción a la ley penal o falta a la disciplina, se dará cuenta inmediata al Presidente de la Corte.



    Artículo 63 bis A.- La duración de los alegatos de los abogados, por cada parte, se limitará a una hora en las apelaciones y consultas de la sentencia definitiva y a media hora en los asuntos incidentales. En el caso de la vista de la causa, en apelación o consulta, de resoluciones que recaigan sobre la libertad provisional, los alegatos se extenderán por un término de hasta quince minutos. El tribunal podrá, sin embargo, autorizar una prórroga hasta por el doble de la duración de los alegatos.
    El tribunal resolverá las apelaciones y consultas relativas a la libertad provisional sin oír el alegato del abogado del inculpado o procesado si después de escuchada la relación no lo estima necesario para concederla. No tendrá efecto esta regla cuando se anuncie el representante del Ministerio Público o el abogado del querellante para alegar.



    Art. 64. (85) Todo inculpado o procesado que se encuentre privado de libertad se presume pobre para todos los efectos legales.
      Puede, sin embargo, encomendar, a su costa, su defensa y representación a otro abogado o procurador designados por él.



    Art. 65. (86) Si la parte civil o el querellante que hubieren entablado una acción pública, no evacuaren un trámite, que les corresponda en el plazo respectivo, no se suspenderá la subtanciación del proceso, sin perjuicio de que puedan intervenir en los trámites posteriores.
    Art. 66. (87) Las notificaciones que hayan de hacerse a los representantes del Ministerio Público, se les harán personalmente en todo caso.
    Las notificaciones al privado de libertad que no tuviere defensor o mandatario constituido en el respectivo proceso, deberán hacérsele personalmente en el recinto donde se encontrare recluido. El secretario del tribunal comunicará al encargado de este recinto, de inmediato y por el medio más rápido posible, el nombre del detenido o preso, el número del proceso, la fecha y la resolución dictada. Este funcionario deberá comunicar dicha resolución al recluido sin dilación alguna, gestión de la cual dará cuenta al secretario del tribunal respectivo. El secretario dejará testimonio en el proceso de las actuaciones practicadas conforme a este inciso, con mención de la fecha en que se efectuaron, la individualización del encargado del recinto que recibió la comunicación y el hecho de que éste hubiere practicado la notificación.
    El privado de libertad que no tuviere defensor o mandatario judicial constituido en el proceso, podrá deducir verbalmente el recurso de apelación que procediera en el acto mismo de la notificación. El encargado del recinto deberá informar de este hecho al secretario del tribunal, de inmediato y por el medio más rápido posible. Este dejará testimonio de ello en el expediente. Concedida la apelación, se elevarán los autos a la respectiva Corte de Apelaciones.
    Tratándose de personas privadas de libertad que tuvieren defensor o mandatario constituido en el proceso, las resoluciones deberán notificarse solamente a dichos representantes. Las notificaciones se efectuarán por el estado diario, salvo que se trataré del auto de procesamiento, del auto acusatorio o de la sentencia definitiva de primera instancia, todas las cuales se notificarán por cédula. Sin perjuicio de lo anterior, la resolución que deniegue la libertad, la que someta a proceso al imputado, el auto acusatorio, la sentencia definitiva de primera instancia y el cúmplase de la sentencia de segunda instancia deberán, además, ser notificadas personalmente al detenido o preso en la forma establecida en los incisos precedentes. Los recursos que procedieren deberán ser interpuestos por el defensor o mandatario, contabilizándose los plazos para su interposición a partir de la fecha de la notificación a éstos. En todo caso, la apelación de la resolución que deniegue la libertad y de la sentencia definitiva de primera instancia, podrá ser deducida por el procesado en el acto mismo de la notificación personal recién aludida.
    Lo dispuesto en este artículo se aplicará aun cuando el lugar de reclusión no se encontraré dentro del territorio jurisdiccional del tribunal que hubiere dictado la resolución que deba notificarse.
    El reglamento establecerá la forma en que el encargado del recinto o establecimiento penitenciario dará cumplimiento a las obligaciones que se le imponen en este artículo.

    Artículo 66 bis.- No obstante lo establecido en el artículo precedente, el juez podrá disponer, por resolución fundada y de manera excepcional, que la notificación de aquellas resoluciones que deban comunicarse personalmente al privado de libertad sea practicada por el secretario en el recinto del tribunal.
    En todo caso, si el detenido o preso se encontraré en el recinto del tribunal al momento de dictarse la resolución, ésta deberá serle notificada de inmediato por el secretario, aplicándose en lo demás lo dispuesto en el artículo anterior.

      3. Del funcionamiento extraordinario de los
tribunales que ejercen competencia en materia penal

    Artículo 66 ter.- Sin perjuicio de lo previsto en los artículos 559 y 560 del Código Orgánico de Tribunales, las Cortes de Apelaciones podrán ordenar que los jueces que ejercen jurisdicción en materia penal en su territorio jurisdiccional se aboquen exclusiva y extraordinariamente a la tramitación de las causas, de competencia de su tribunal, relativas a la investigación y juzgamiento de uno o más delitos en los que se encontrare comprometido un interés social relevante o que produzcan alarma pública.
    En todo caso, el funcionamiento extraordinario podrá adoptarse respecto de ciertas causas o grupo de causas, cuando hubiere retardo en el despacho de los asuntos sometidos al conocimiento del tribunal y, en general, siempre que el mejor servicio judicial así lo exigiere.
    Asimismo, en uso de esta facultad, las Cortes de Apelaciones podrán ordenar que el juez titular de un juzgado de letras de competencia común se aboque exclusivamente al conocimiento de todos los asuntos de naturaleza criminal que se ventilen en dicho tribunal.
    La resolución que decrete el funcionamiento extraordinario señalará la periodicidad con que el juez deberá informar de los avances obtenidos en el curso de los procesos de que se trate.
    La Corporación Administrativa del Poder Judicial informará anualmente a las Cortes de Apelaciones y al Ministerio de Justicia respecto de la aplicación que hubiese tenido el sistema de funcionamiento extraordinario y de las disponibilidades presupuestarias para el año siguiente.

    Artículo 66 ter A.- Cuando se iniciare el funcionamiento extraordinario, se entenderá, para todos los efectos legales, que el juez falta en su despacho. En esa oportunidad, el secretario del mismo tribunal asumirá las demás funciones que le corresponden al juez titular, en carácter de suplente, y por el solo ministerio de la ley.
    Quien debiere cumplir las funciones del secretario del tribunal, de acuerdo a las reglas generales, las llevará a efecto respecto del juez titular y de quien lo supliere o reemplazare.

    Artículo 66 ter B.- Los tribunales que ejercen competencia en materia penal deberán, a lo menos en el mes de noviembre de cada año, remitir un informe a la Corte de Apelaciones respectiva, dando cuenta del estado de las causas pendientes en el tribunal que pudieren encontrarse en alguno de los casos previstos en el artículo 66 bis.
    Podrán, asimismo, cuando las condiciones hubieren variado, remitir nuevos informes para que se considere la adopción de las medidas que corresponda.

    Artículo 66 ter C.- Las atribuciones de las Cortes de Apelaciones previstas en este párrafo serán ejercidas por una sala integrada solamente por Ministros titulares.

    4. Derechos del inculpado.

    Artículo 67.- Todo inculpado, sea o no querellado, y aún antes de ser procesado en la causa, podrá hacer valer, hasta la terminación del proceso, los derechos que le acuerden las leyes y los que el tribunal estime necesarios para su defensa.
    En especial, podrá:
    1.- Designar abogado patrocinante y procurador;
    2.- Presentar pruebas destinadas a desvirtuar los cargos que se le imputen;
    3.- Rendir información sumaria de testigos para acreditar su conducta anterior, sin necesidad de ofrecerla o anunciarla por escrito previamente;
    4.- Pedir que se active la investigación;
    5.- Solicitar conocimiento del sumario, en conformidad a las reglas generales;
    6.- Solicitar reposición de la orden de detención librada en su contra;
    7.- Apelar de la resolución que niegue lugar al sobreseimiento o sobresea sólo temporalmente, y
    8.- Intervenir ante los tribunales superiores en los recursos contra la resolución que niega lugar a someterlo a proceso y en los recursos y consultas relativas al sobreseimiento.
    Los derechos en el proceso penal del simple inculpado menor de dieciocho años pueden ser ejercidos por sus padres o guardadores y los del demente por su curador.  Si no existieren tales representantes o estuvieren, en concepto del juez, inhabilitados, y no se hubieren designado abogado y procurador, el juez, una vez prestada la indagatoria, podrá designarles a los que corresponda de acuerdo con las reglas previstas en el Título XVII del Código Orgánico de Tribunales, aunque el inculpado se encuentre en libertad.





NOTA 1.1
    Las modificaciones introducidas al presente Código por la Ley N° 18.857, publicada en el Diario Oficial de 6 de Diciembre de 1989, rigen, según lo dispone su artículo vigésimo, noventa días después de su publicación en el Diario Oficial.
      4. Nulidades Procesales

    Artículo 68.- Regirán las disposiciones relativas a nulidades procesales contenidas en el Código de Procedimiento Civil, en cuanto puedan aplicarse al juicio penal y no fueren contrarias a las que se prescriben en este párrafo.

    Artículo 69.- Sólo pueden anularse los actos procesales cuando la violación de las normas que los establecen esté sancionada con la nulidad o se refiere a un acto o trámite declarado esencial por la ley.
    Se entiende siempre establecido bajo sanción de nulidad el cumplimiento de las disposiciones concernientes a la intervención del Ministerio Público en los actos en que ella es obligatoria; y a la intervención, patrocinio y representación del procesado, en los casos y formas establecidos por la ley.





NOTA 1.1
    Las modificaciones introducidas al presente Código por la Ley N° 18.857, publicada en el Diario Oficial de 6 de Diciembre de 1989, rigen, según lo dispone su artículo vigésimo, noventa días después de su publicación en el Diario Oficial.
    Artítulo 70.- No puede pedir la nulidad procesal la parte que sea causante del vicio ni aquella a quien no le afecta.
      El Ministerio Público podrá solicitarla dentro de su competencia.

    Artículo 71.- Las partes sólo podrán pedir incidentalmente la nulidad de los trámites y actos procesales en las siguientes oportunidades:
      1.- La de aquellos realizados en el sumario, durante él, o en el plazo señalado en el artículo 401 o en los escritos fundamentales del plenario, y
    2.- La de trámites y actos realizados en el plenario, dentro de los cinco días siguientes a aquel en que se tuvo conocimiento del vicio.

    Artículo 71 bis.- Las nulidades quedan subsanadas si las partes no las oponen en las oportunidades establecidas en el artículo anterior; cuando las partes que tengan derecho a oponerlas hayan aceptado expresa o tácitamente los efectos del acto; y cuando, no obstante el vicio de que adolezca el acto, éste haya conseguido su fin respecto de todos los interesados.

    Artículo 72.- La declaración de nulidad de un acto lleva consigo la de los actos consecutivos que de él emanan o dependen.
      El tribunal, al declarar la nulidad, determinará concretamente cuales son los actos a los que se extiende y, siéndole posible, ordenará que se renueven, rectifiquen o ratifiquen.
    El tribunal corregirá de oficio los errores que observe en la tramitación del proceso.  Podrá, asimismo, tomar las medidas que tiendan a evitar la nulidad de los actos de procedimiento.  No podrá, sin embargo, subsanar las actuaciones viciadas en razón de haberse realizado fuera del plazo fatal indicado por la ley.
    La rectificación, ratificación o corrección de una diligencia practicada en el sumario, decretada con posterioridad a su término, se cumplirá, cuando sea posible, durante el plenario o como medida para mejor acierto del fallo.
    Las resoluciones que resuelven sobre la nulidad son apelables, pero el recurso se concederá sólo en el efecto devolutivo.

    Artículo 73.- Declarada la nulidad de una notificación judicial, las partes se entenderán notificadas de la resolución a que aquella actuación se refiere por el solo ministerio de la ley, transcurridos tres días desde que se notifique por el estado diario la resolución que acoge la nulidad, o desde que se notifique el cúmplase de ella, si ha sido dictado por un tribunal superior.
    No obstante, el procesado que estuviere sometido a prisión preventiva y el Ministerio Público serán notificados nuevamente en la forma prevista en el artículo 66.

    Título IV
  DE LA POLICIA

    Artículo 74.- La Policía de Investigaciones de Chile deberá cumplir en sus respectivos territorios jurisdiccionales las órdenes y resoluciones emanadas de los Tribunales de Justicia y también fuera de ellos, cuando éstos así lo dispongan.
  Carabineros de Chile, deberá desempeñar las funciones indicadas precedentemente en aquellos lugares en que no exista Policía de Investigaciones y, aun existiendo, cuando el tribunal así lo resuelva.
  Gendarmería de Chile, deberá cumplir las órdenes y resoluciones de los Tribunales de Justicia respecto de los delitos cometidos en el interior de los establecimientos penales, sin perjuicio que el Tribunal pueda encomendar su cumplimiento a Carabineros de Chile o a la Policía de Investigaciones.
  El personal de las instituciones indicadas, cuando cumpla órdenes o resoluciones judiciales actuará de conformidad a lo dispuesto en este Código.



NOTA:  4
    Véase el Decreto-Ley N° 2.460, de 9 de enero de 1979, publicado en el Diario Oficial de 24 del mismo mes, Ley Orgánica de Policía de Investigaciones de Chile.
    Artículo 74 bis.- No obstante las atribuciones que correspondan al Presidente de la República, el personal de la Policía de Investigaciones de Chile, en cuanto auxiliar de la administración de justicia, quedará sujeto a la jurisdicción correccional y económica de los tribunales de justicia respectivos, sin perjuicio de la intervención de los tribunales superiores que corresponda.

    Artículo 74 bis A.- Los funcionarios de las instituciones indicadas en el artículo 74, sólo podrán cumplir, en lo que se refiere a la investigación de los delitos, las órdenes emanadas de autoridad competente.
    Estas órdenes deberán constar siempre por escrito y serán exhibidas a la persona a quien afecten al efectuarse su cumplimiento, cualquiera que sea la autoridad de que provengan o la persona contra quien se dicten, sin perjuicio de lo dispuesto en los artículos 260, 261, 262 y 282 y en el siguiente inciso.
    El juez que conoce de un proceso podrá designar funcionarios determinados de las instituciones señaladas en el artículo 74, para que se hagan cargo preferentemente de la investigación de delitos de especial gravedad o complejidad, o para el cumplimiento de órdenes judiciales.

    Artículo 74 bis B.- Se prohibe a todo funcionario de las instituciones indicadas en el artículo 74 dar informaciones sobre los resultados de las pesquisas que practiquen y de las órdenes que deban cumplir.
    El juez podrá dar conocimiento a los funcionarios investigadores de los datos del proceso que estime conducentes al éxito de las indagaciones que se les encarguen.  Asimismo, podrá proporcionarles copia de los informes, autopsias y demás pericias, cuando sean solicitadas por los jefes de las unidades que tengan a su cargo la investigación del caso.
    Los funcionarios que hayan tomado conocimiento de datos del proceso o recibido copias de los informes indicados en el inciso precedente, quedan obligados a no revelarlos.
    La infracción de las disposiciones de los incisos primero y tercero de este artículo será sancionada con reclusión o presidio menor en su grado mínimo a medio, a menos que los hechos constituyan otro delito sancionado con igual o mayor pena.

    Artículo 75.- El Fiscal de la Corte Suprema tendrá la supervigilancia del cumplimiento de las órdenes judiciales y podrá, en tal carácter, por sí o por medio de los oficiales del Ministerio Público, recabar informes, hacer inspecciones, prescribir órdenes para que los decretos judiciales sean legal y oportunamente acatados, practicar indagaciones y recibir declaraciones sin juramento, con el objeto de hacer efectiva la responsabilidad funcionaria o penal de los infractores.

    Artículo 75 bis.- Ejecutoriada una sentencia que imponga la pena de sujeción a la vigilancia de la autoridad, el juez determinará los lugares prohibidos al al penado, mientras quede sujeto a esa vigilancia, y también las obligaciones que le imponga conforme al artículo 45 del Código Penal.

    Libro Segundo
    DEL JUICIO ORDINARIO SOBRE CRIMEN O SIMPLE DELITO
    Primera Parte
    DEL SUMARIO
    Título I
    DEL SUMARIO EN GENERAL
    Art. 76. (97) Todo juicio criminal a que dé origen la perpetración de un crimen o simple delito comenzará por la investigación de los hehos que constituyan la infracción y determinen la persona o personas responsables de ella, y las circunstancias que puedan influir en su calificación y penalidad; sin perjuicio de las reglas especiales establecidas en el Libro III.
    Las diligencias dirigidas a preparar el juicio por medio de tales esclarecimientos y asegurar la persona de los presuntos culpables y su responsabilidad pecuniaria, constituyen el sumario.
    Art. 77. (98) Cada crimen o simple delito de que conozca un tribunal será materia de un sumario.
    Sin embargo, se comprenderá en un solo sumario:
    1° Los delitos conexos; y
    2° Los diversos crímenes, simples delitos y faltas que se imputaren a un solo procesado, ya sea al iniciarse la causa o durante el progreso de ésta.
    Art. 78. (99) Las actuaciones del sumario son secretas, salvo las excepciones establecidas por la ley. En las causas relativas a los delitos previstos en los artículos 361 a 363 y 366 a 367 bis y, en lo que fuere aplicable, también en los delitos previstos en los artículos 365 y 375 del Código Penal, la identidad de la víctima se mantendrá en estricta reserva respecto de terceros ajenos al proceso, a menos que ella consienta expresamente en su divulgación. El juez deberá decretarlo así, y la reserva subsistirá incluso una vez que se encuentre afinada la causa. La infracción a lo anterior será sancionada conforme a lo dispuesto en el inciso cuarto del artículo 189. El tribunal deberá adoptar las demás medidas que sean necesarias para garantizar la reserva y asegurar que todas las actuaciones del proceso a que deba comparecer la víctima se lleven a cabo privadamente.
    Art. 79. (100) El juez puede autorizar al procesado para que tome conocimiento de aquellas diligencias que se relacionen con cualquier derecho que trate de ejercitar, siempre que haciéndolo no se entorpezca la investigación.
    Todo aquel a quien se notifica una resolución tiene derecho a sacar copia de ella.


    Art. 80. (101) Si el sumario se prolongare por más de cuarenta días desde aquel en que el inculpado hubiere sido procesado, éste tendrá derecho para que se ponga en su conocimiento todo lo obrado, a fin de instar por la terminación. Esta solicitud no puede ser denegada sino en cuanto sea peligroso para el éxito de la investigación; y la apelación que en tal caso se entablare será otorgada en el solo efecto devolutivo cuando hubiere pendientes ante el tribunal diligencias de importancia que no deban retardarse.
    En los procesos por delitos de robos con violencia o intimidación en las personas, el sumario deberá cerrarse dentro de 40 días contados desde aquel en que el inculpado haya sido procesado. Este plazo será prorrogable por una sola vez y por igual tiempo, mediante resolución fundada. Sin perjuicio de lo anterior, el procesado tendrá siempre derecho al conocimiento del sumario transcurridos 120 días desde la fecha de la resolución que lo sometió a proceso.



  Título II
  DE LAS DIVERSAS MANERAS DE INICIAR EL PROCESO POR
CRIMINES O SIMPLES DELITOS PESQUISABLES DE OFICIO

    Art. 81. (102) Los juicios a que se refiere este Título pueden comenzar:
    1° Por denuncia;
    2° Por querella;
    3° Por requisición del Ministerio Público; y
    4° Por pesquisa judicial.

    Art. 82. (103) Denuncia un delito la persona que pone en conocimiento de la justicia o de sus agentes, el hecho que lo constituye, y, por lo regular, el nombre del delincuente o los datos que lo identifiquen, no con el objeto de figurar como parte en el juicio, sino con el de informar al tribunal a fin de que proceda a la instrucción del respectivo proceso.
    Art. 83. (104) Todo el que tenga conocimiento de un hecho punible puede denunciarlo.
    Son obligados a recibir la denuncia no solamente el tribunal a quien corresponda el conocimiento de la causa, sino también cualquier tribunal que ejerza jurisdicción en materia criminal y los funcionarios de Carabineros de Chile y de la Policía de Investigaciones. Todos ellos deben transmitir inmediatamente la denuncia al tribunal que juzguen competente.
    No será necesario citar a declarar a dichos funcionarios policiales acerca del hecho de haber recibido la denuncia y del contenido expresado en ella por el denunciante.
    El funcionario de Carabineros de Chile, de la Policía de Investigaciones o el tribunal que reciba una denuncia por hurto o robo deberá, en el acto de hacerlo, requerir del denunciante una declaración jurada, ante él sobre la preexistencia de las cosas sustraídas y una apreciación de su valor.
  Tratándose de delitos contra las personas, aborto, robo, hurto y de los contemplados en la ley N° 19.366, que sanciona el tráfico ilícito de estupefacientes y sustancias sicotrópicas, Carabineros de Chile o la Policía de Investigaciones, en su caso, deberán practicar de inmediato y sin previa orden judicial, las diligencias que se establecen en el artículo 120 bis, sin perjuicio de lo dispuesto en el artículo 260. Las diligencias que debieren practicarse en recinto cerrado, sólo se podrán realizar con autorización previa y expresa del propietario, arrendatario o persona a cuyo cargo esté el local en que deban efectuarse. El parte al tribunal en que se consigne la denuncia, deberá detallar las diligencias efectuadas y, en caso contrario, las razones por las cuales no se hicieron.

    Art. 84. (105) Están obligados a denunciar:
    1° El Ministerio Público, los hechos criminales que se pongan en su conocimiento;
    2° Los miembros de Carabineros de Chile, de la Policía de Investigaciones de Chile y de Gendarmería, todos los delitos que presencien o lleguen a su noticia.
Las Fuerzas Armadas están también obligadas a formular denuncia respecto de todos los delitos de que tomen conocimiento en el ejercicio de sus funciones, ante los tribunales de justicia;
    3° Los empleados públicos, los crímenes o simples delitos de que tomen conocimiento en el ejercicio de sus funciones, y especialmente los que noten en la conducta ministerial de sus subalternos;
    4° Los jefes de puertos, aeropuertos, estaciones de trenes o buses o de otros medios de locomoción o de carga, los capitanes de naves mercantes o aeronaves comerciales que naveguen en el mar o en el espacio territorial, y los conductores de los trenes, buses u otros medios de transporte o carga, los delitos que se cometan durante el viaje, en el recinto de una estación, puerto o aeropuerto o a bordo de un buque o aeronave, y
    5° Los jefes de establecimientos hospitalarios o de clínicas particulares y, en general, los profesionales en medicina, odontología, química, farmacia y otras ramas relacionadas con la conservación o restablecimiento de la salud, y los que ejerzan profesiones auxiliares de ellas, que noten en una persona o en un cadáver señales de un crimen o simple delito.

    La denuncia hecha por uno de los obligados en este número exime al resto.

    Artículo 85.- Las personas indicadas en el artículo anterior deberán hacer la denuncia dentro de las veinticuatro horas siguientes al momento en que tengan conocimiento del hecho criminal. Respecto de los capitanes de naves o aeronaves, se contará este plazo desde que arriben a cualquier puerto o aeropuerto de la República.

    Art. 86. (107) Las personas indicadas en el artículo 84 que omitan hacer la denuncia que en él se prescribe, incurrirán en la pena señalada en el artículo 494 del Código Penal, que impondrá el juez que deba conocer de la causa principal, observando las formalidades prescritas en el Título I del Libro III de este Código.
    Si hubiere mérito para estimar como encubridor al funcionario que ha omitido la denuncia, el juez procederá contra él con arreglo a la ley.
    Si el que ha omitido la denuncia es un miembro de las Fuerzas Armadas, Carabineros, Investigaciones o Gendarmería que ha debido obrar de acuerdo con lo establecido en el N° 2 del artículo 84, se comunicará la infracción al juzgado correspondiente.

    Art. 87. (108) El denunciante no contrae otra responsabilidad que la correspondiente a los delitos que hubiere cometido por medio de la denuncia o con ocasión de ella.
    Art. 88. (109) No pueden ser denunciantes las personas a quienes está prohibido el ejercicio de la acción penal por los artículos 16 y 17, en los mismos casos en que dichos artículos determinan.
    No obstante, no será nulo el procedimiento iniciado por delito de acción pública como consecuencia de la denuncia de alguna de las personas a que se refiere el inciso anterior.

    Art. 89. (110) La denuncia, puede ser hecha de palabra o por escrito, debe contener la narración circunstanciada de delito, la designación de los que lo hayan cometido y de las personas que hayan presenciado su perpetración o que tuvieren noticia de él, todo en cuanto le conste al denunciante.
    Art. 90. (111) La denuncia verbal se extenderá en un acta en presencia del denunciante; quien la firmará, si puede, junto con el funcionario que la reciba. Si el denunciante no pudiere o no supiere firmar, lo hará otra persona a su ruego.
    La denuncia escrita será firmada por el denunciante o por un apoderado especial, o por un tercero a ruego del denunciante que no pudiere o no supiere firmar.
    El acta de la denuncia describirá detalladamente el hecho punible y el lugar en que se cometió; individualizará de la forma más completa a la persona o cosa que ha sido objeto del delito, los presuntos culpables y los testigos, y, en general, contendrá los mayores datos que puedan servir para determinar el hecho punible, la persona del o de los responsables y las circunstancias que puedan influir en su calificación y penalidad. Dejará constancia, asimismo, de la información proporcionada a los testigos sobre el derecho a requerir reserva de su identidad y de aquellos que lo hayan ejercido, de conformidad a los incisos segundo y tercero del artículo 189.


    Art. 91. (112) Recibida la denuncia y sin más trámite, el juez procederá inmediatamente a la comprobación del hecho denunciado, salvo que éste no revista el carácter de delito o que la denuncia sea manifiestamente falsa. En estos dos casos se abstendrá el juez de todo procedimiento, pero incurrirá en responsabilidad si la desestima indebidamente.
    La comprobación inmediata del hecho denunciado a que se refiere el inciso anterior se llevará a cabo aunque la denuncia hubiere sido formulada ante la policía u otro tribunal. El denunciante no deberá concurrir a ratificar su denuncia, y sólo podrá ser citado a declarar cuando el juez por resolución fundada lo determine.

    Artículo 92.- Los tribunales no darán curso a denuncias hechas por personas desconocidas ni a delaciones, a no ser que contengan datos precisos que hagan verosímil que se ha cometido el hecho denunciado o delatado. En tal caso procederá el juez, previamente, a verificar los datos con el mayor secreto, procurando no comprometer la reputación de la persona inculpada.

    Art. 93. (114) Toda persona capaz de parecer en juicio por sí misma, puede querellarse ejercitando la acción pública de que se trata en los artículos 10 y 11 de este Código, si no le está expresamente prohibido por la ley.
    De los delitos enumerados en el artículo 18 no pueden querellarse sino las personas que en dicho artículo se indican.
    El querellante puede intervenir durante el sumario presentando todas las pruebas que obren en su poder y solicitando que se practiquen todas aquellas diligencias que creyere necesarias para el esclarecimiento de los hechos, y el juez ordenará que se lleven a efecto las que estime conducentes.

    Art. 94. (115) Toda querella criminal debe presentarse por escrito y contener:
    1° La designación del tribunal ante quien se entable;
    2° El nombre, apellido, profesión u oficio y residencia del querellante;
    3° El nombre, apellido, profesión u oficio y residencia del querellado, o una designación clara de su persona, si el querellante ignorare aquellas circunstancias. Si se ignoraren dichas determinaciones, siempre se podrá deducir querella para que se proceda a la investigación del delito y al castigo del o los culpables;
    4° La relación circunstanciada del hecho, con expresión del lugar, año, mes, día y hora en que se ejecutó, si se supieren;
    5° La expresión de las diligencias que se deberán practicar para la comprobación del hecho;
    6° El ofrecimiento de la fianza de calumnia, si el querellante no estuviere exento de ella;
    7° La petición de que se admita la querella, se practiquen las diligencias indicadas, se proceda a la citación o detención del presunto culpable, o a exigirle la fianza de libertad provisional, y de que se decrete el embargo de sus bienes en la cantidad necesaria, el embargo o medidas precautorias para asegurar la responsabilidad pecuniaria del querellado; todo esto según proceda de derecho; y
    8° La firma del querellante o de la otra persona a su ruego, si no pudiere firmar.

    Artículo 95.- La querella sólo podrá deducirse hasta el momento que quede ejecutoriada la resolución que declara cerrado el sumario.


    Art. 96. (117) Tratándose de los delitos de injuria o calumnia causadas en juicio que puedan ser perseguidos por el Ministerio Público, el querellante acompañará testimonio de estar terminado el litigio en que se causó la calumnia o la injuria y de la resolución en que el tribunal que conoció de él hubiere declarado que había mérito para proceder criminalmente.
    Art. 97. (118) Cuando la calumnia o injuria hecha en juicio no dé mérito para proceder criminalmente en concepto del tribunal que conoce de la causa en que se vertiere, éste, de oficio o a petición de parte, procediendo de plano y sin formalidad especial, corregirá la falta aplicando al que la hubiere cometido alguna de las personas disciplinarias que tuviere facultad de imponer con arreglo a lo dispuesto en el Código Orgánico de Tribunales.
    Art. 98. (119) El juez a quien corresponda conocer de la querella, calificará la fianza ofrecida por el querellante y fijará su cuantía, tomando en cuenta la gravedad del delito y las ciscunstancias que lo hagan verosímil.
    Extendida la fianza en un acta subscrita ante el secretario y presentada al juez, se dará curso a la querella y se practicarán las diligencias en ella indicadas o las que el juez estime conducentes.
    Art. 99. (120) El fiador de calumnia se obliga a responder por las penas pecuniarias a que pueda ser condenado el querellante y por el pago de costas e indemnizaciones de perjuicios irrogados al querellado, en el caso de que la querella resultare calumniosa.
    Artículo 100.- No están obligados a rendir fianza de calumnia:
    1° El ofendido ni sus herederos o representantes legales;
    2° En los delitos de homicidio o lesiones graves, el cónyuge del ofendido, sus ascendientes o descendientes legítimos o naturales; ni sus parientes colaterales legítimos hasta el segundo grado de consanguinidad o de afinidad; ni su adoptante ni su adoptado.
    3° El que se querella por el delito de falsificación de moneda que tenga curso legal, o de falsificación de documentos de crédito emitidos por organismos o empresas del Estado, sociedades anónimas, bancos comerciales o instituciones financieras, y
    4° Los oficiales del Ministerio Público y los representantes del Consejo de Defensa del Estado, de las Municipalidades, de la Contraloría General de la República y de los servicios fiscales, semifiscales y de administración autónoma, en las querellas que interpusieren en carácter de tales.

    Artículo 101.- Las personas designadas en los tres primeros números del artículo precedente serán responsables del delito de calumnia en los mismos casos en que lo sería cualquier otro querellante.

    Art. 102. (123) Si no constituyeren un delito los hechos expuestos en la querella, el juez no le dará curso y dictará al efecto un auto motivado.
    Si el juez se cree incompetente, lo declarará así; y el querellante podrá ocurrir ante el tribunal a quien corresponda el conocimiento del negocio, sin perjuicio de lo dispuesto en el artículo 47.
    Artículo 102 bis.- Cuando no se diere curso a una querella en que se persiga un delito de acción pública por defectos en la forma de interponerla, el juez la considerará una denuncia para los efectos de decidir la iniciación del sumario.

    Art. 103. (124) La previa requisición exigida por la ley en ciertos casos para que el Ministerio Público ponga en ejercicio la acción pública, debe contener las mismas indicaciones que para la denuncia requiere el artículo 89. La requisición será dirigida por el ofendido al correspondiente oficial del Ministerio Público.
    Artículo 103 bis.- El ejercicio de la acción civil durante el sumario, debidamente cursada, interrumpe la prescripción.
    No obstante, si dicha acción no se formalizare en conformidad a lo prescrito en el artículo 428, continuará la prescripción como si no se hubiere interrumpido.

    Art. 104. (125) El Ministerio Público, el querellante y el actor civil podrán pedir, durante el sumario, que se practiquen todas aquellas diligencias que creyeren necesarias para el esclarecimiento de los hechos; y el juez ordenará que se lleven a efecto los que estimare conducentes.
    El juez podrá permitir que el Ministerio Público o el querellante se impongan de lo obrado en el sumario, a menos que, para el mejor éxito de la investigación, conceptúe conveniente mantener secretas las diligencias.
    Los incidentes que promuevan durante el sumario las partes civiles se tramitarán en ramo separado y no retardarán la marcha del proceso penal. Las apelaciones que las partes civiles interpongan se concederán, cuando procedan, siempre en lo devolutivo.

    Artículo 105.- Sin esperar denuncia, ni querella alguna, deberá el juez competente instruir sumario de oficio, siempre que, por conocimiento personal, por avisos confidenciales, por notoriedad o por cualquier otro medio, llegare a su noticia la perpetración de un crimen o simple delito de acción pública.

    Art. 106. (127) En el caso a que se refiere el artículo precedente, el juez expedirá un auto cabeza de proceso en que, después de enunciar el conducto por donde ha llegado a su noticia el hecho punible, con todas las circunstancias que puedan influir en su calificación o suministrar datos para descubrir a los delincuentes, mandará practicar las primeras diligencias para la comprobación del delito.
    Sin embargo, no enunciará en ese auto los hechos o circunstancias cuya divulgación pueda perjudicar el éxito de la investigación, ni el nombre del denunciante, si éste exigiere su reserva.
    Art. 107. (128) Antes de proseguir la acción penal, cualquiera que sea la forma en que se hubiere iniciado el juicio, el juez examinará si los antecedentes o datos suministrados permiten establecer que se encuentra extinguida la responsabilidad penal del inculpado. En este caso pronunciará previamente sobre este punto un auto motivado, para negarse a dar curso al juicio.
  Título III
  DE LA COMPROBACION DEL HECHO PUNIBLE Y AVERIGUACION
DEL DELINCUENTE

    1. Disposiciones generales

    Artículo 108.- La existencia del hecho punible es el fundamento de todo juicio criminal, y su comprobación por los medios que admite la ley es el primer objeto a que deben tender las investigaciones del sumario

    Art. 109. (130) El juez debe investigar, con igual celo, no sólo los hechos y circunstancias que establecen y agravan la responsabilidad de los inculpados, sino también los que les eximan de ella o la extingan o atenúen.
    Art. 110. (131) El delito se comprueba con el examen practicado por el juez, auxiliado por peritos, en caso necesario, de la persona o cosa que ha sido objeto del delito, de los instrumentos que sirvieron para su perpetración y de las huellas, rastros y señales que haya dejado el hecho; con las disposiciones de los testigos que hayan visto o sepan de otro modo la manera como se ejecutó; con documentos de carácter público o privado; o con presunciones o indicios necesarios o vehementes que produzcan el pleno convencimiento de su existencia.
    Las informaciones que la policía proporcione sobre hechos en que haya intervenido, que se relaten en las comunicaciones o partes que se envíen a los tribunales, tendrán el mérito de un antecedente que el juez apreciará conforme a las reglas generales, sin perjuicio de que pueda citar a los funcionarios respectivos para interrogarlos sobre esos hechos, o para otras diligencias del proceso; y sin perjuicio también del derecho de los inculpados para solicitar que se les interrogue al respecto, se les caree o contrainterrogue.

    Artículo 111.- El delincuente puede ser determinado por uno o más de los medios expresados en el artículo que precede, y además por la confesión de él mismo, acorde con los datos que comprueben el hecho punible.

    Art. 112. (133) Cuando el delito que se persigue haya dejado rastros o señales, el juez procederá personalmente a tomar nota de ellos, y describirá detalladamente en el proceso los que puedan servir para determinar el hecho punible o la persona del delincuente.
    Con este fin, consignará la descripción del lugar en que se cometió el delito, del sitio y estado de los objetos que en él se encuentren, de los accidentes del terreno, de la situación de las habitaciones, y de todos los demás datos que pueden utilizarse en favor o en contra de los presuntos culpables.
    Del mismo modo, si fuere habida la persona o cosa objeto del delito, el juez describirá su estado, con aquellos datos especiales que tengan relación con el hecho punible.

    Artículo 113.- Siempre que fuere necesario para el esclarecimiento de los hechos, el juez hará levantar el plano del lugar, retratar a las personas que hayan sido objeto del delito, o poner en autos el diseño de los efectos o instrumentos del mismo, que fueren encontrados.
    Para los mismos fines, podrá también disponer la fotografía, filmación o grabación y, en general, la reproducción de imágenes, voces o sonidos por los medios técnicos que estime convenientes.  Asimismo, podrá valerse de resultados obtenidos por la utilización de aparatos destinados a desarrollar exámenes o demostraciones científicas o por medio de la computación.  Regirá, para estos efectos, lo que disponen los dos últimos incisos del artículo siguiente.

    Artículo 113 bis.- Podrán admitirse como pruebas películas cinematográficas, fotografías, fonografías, y otros sistemas de reproducción de la imagen y del sonido, versiones taquigráficas y, en general, cualquier medio apto para producir fe. Estos medios podrán servir de base a presunciones o indicios.
    La admisión como pruebas de los elementos de juicio a que se refiere este artículo se decretará con citación, cuando fuere ofrecida por una de las partes; pero en el sumario podrán tenerse en consideración aunque esté pendiente el plazo de citación o cualquiera objeción sobre ellas.
    El juez determinará la forma como ha de dejarse constancia en el proceso de estas pruebas, cuando hicieren necesarias operaciones técnicas especiales para ello o para su realización.  Para tal efecto, podrá designar un asesor técnico que desarrolle y explique la prueba, de entre los que ejercieren los oficios especializados.  Si la prueba fuere ofrecida por una de las partes y el juez lo estimare conveniente, ésta suministrará el personal e instrumentos necesarios para llevar a cabo la demostración, sin perjuicio de lo que se resuelva sobre costas.  En todo caso, si el tribunal cuenta con los instrumentos requeridos y no es necesaria la cooperación de un técnico, procederá a realizar la prueba por sí mismo.
    Se certificará, después de verificada la operación, el día y la hora en que se verificó, el nombre y dirección de los que intervinieron en ella, y el lugar, la persona, cosa, suceso o fenómeno que se reproduce o explica, y el juez deberá tomar las medidas necesarias para evitar que puedan ser alterados los originales de estas pruebas.

    Artículo 113 ter.- Cuando existieren sospechas fundadas de que una persona o una organización delictiva hubiere cometido o preparado la comisión de alguno de los delitos previstos en los artículos 366 quinquies, 367, 367 bis, 367 ter, 374 bis, inciso primero, y 374 ter del Código Penal, y la investigación lo hiciere imprescindible, el juez podrá ordenar la interceptación o grabación de las telecomunicaciones de esa persona o de quienes integraren dicha organización y la grabación de comunicaciones.
    La orden que dispusiere la interceptación o grabación deberá indicar el nombre o los datos que permitan la adecuada identificación del afectado por la medida y señalar la forma en que se aplicará y su duración, la que no podrá exceder de sesenta días. El juez podrá prorrogar este plazo por períodos de hasta igual duración, para lo cual deberá examinar cada vez la concurrencia de los requisitos previstos en los incisos precedentes. En todo caso, la orden judicial no podrá extenderse más allá de un año desde que se decretó.
    Las empresas o establecimientos que presten los servicios de comunicación a que se refiere el inciso primero deberán, en el menor plazo posible, poner a disposición de los funcionarios encargados de la diligencia todos los recursos necesarios para llevarla a cabo. Con este objetivo los proveedores de tales servicios deberán mantener, en carácter reservado, a disposición del tribunal, un listado actualizado de sus rangos autorizados de direcciones IP y un registro, no inferior a seis meses, de los números IP de las conexiones que realicen sus abonados. La negativa o el entorpecimiento en la práctica de la medida decretada será constitutiva del delito de desacato, conforme al artículo 240 del Código de Procedimiento Civil. Asimismo, los encargados de realizar la diligencia y los empleados de las empresas o de los establecimientos deberán guardar secreto acerca de la misma, salvo que se les citare como testigos al procedimiento.
    Si las sospechas tenidas en consideración para ordenar las medidas se disiparen o hubiere transcurrido el plazo de duración fijado para las mismas, ellas deberán ser interrumpidas inmediatamente.
    Bajo los mismos supuestos previstos en el inciso primero, el juez podrá autorizar la intervención de agentes encubiertos, o la realización de entregas vigiladas de material, de acuerdo a lo dispuesto en el artículo 369 ter del Código Penal.

    Art. 114. (135) Los instrumentos, armas u objetos de cualquiera clase que parezca haber servido o haber estado destinado para cometer el delito, y los efectos que de él provengan, ya estén en poder del presunto culpable o de otra persona, serán recogidos por el juez, quien mandará conservarlos bajo sello y levantar acta de la diligencia, que firmará, si pudiere, la persona en cuyo poder aquéllos han sido encontrados.
    El juez adoptará las medidas conducentes para que las especies recogidas se mantengan en el mejor estado posible.
    Si entre dichos objetos se encuentran vasos u otras cosas sagradas, el juez ordenará que sean separados de los demás y guardados con especial cuidado.
    Art. 115. (136) Las reclamaciones o tercerías que las partes o terceros entablen durante el juicio con el fin de obtener la restitución de los objetos de que se trata en el artículo precedente, se tramitarán por separado en la forma de un incidente, y la sentencia se limitará a declarar el derecho de los reclamantes sobre dichos objetos; pero no se efectuará la devolución de éstos sino después de terminado el juicio criminal o antes, si en concepto del juez no fuere necesario conservarlos.
    Lo dispuesto en el inciso precedente no se extiende a las cosas hurtadas, robadas o estafadas, las cuales se entregarán al dueño en cualquier estado del juicio, una vez que resulte comprobado su dominio y sean valoradas en conformidad a la ley.
    Artículo 116.- Si no hubieren quedado huellas de la perpetración del delito, el juez hará constar por cualquier medio de prueba el hecho de haber sido cometido, con las circunstancias que sirvan para graduar la pena, y acreditará, del mismo modo, la preexistencia de la cosa cuya subtracción fuere materia del sumario.

    Art. 117. (138) Toda diligencia practicada por el juez durante el sumario se extenderá por escrito en el acto mismo de llevarse a cabo, y será firmado por el juez, las personas que han intervenido en ella y el secretario.
    En la diligencia se mencionarán el lugar, día y hora en que se verificó el acto, el nombre de las personas que hubieren asistido y las indicaciones que permitan comprobar que se han cumplido las formas esenciales del procedimiento.
    Lo establecido en los dos incisos precedentes no se opone al uso de instrumentos de reproducción o captación de sonidos o de imágenes, como medios auxiliares para levantar el acta.
    Si el juez necesita dejar testimonio, como complemento de una diligencia, de la existencia o contenido de documentos públicos, oficiales, protocolizados o incorporados a registros públicos que se encuentren en otras oficinas, podrá cometer al secretario la inspección de ellos y el levantamiento del acta correspondiente, la cual tendrá el mismo mérito, que si hubiere sido hecha por el tribunal.
    Lo prescrito en el inciso anterior no obsta a que el juez recabe directamente, del funcionario correspondiente, las copias que estime necesarias para agregarlas al proceso.

    Art. 118. (139) Toda diligencia será leída a las personas que deban subscribirla y si alguna observa que la exposición contiene cualquiera inexactitud, se tomará nota de su observación, y si se negare a firmar, se expresará la razón que alegue para no hacerlo.
    Artículo 119.- Las diligencias deben extenderse sin abreviaturas, sin dejar blancos y sin raspar el papel para hacer enmiendas. Pero si fuere necesario enmendar o interlinear una o más palabras, el secretario rubricará al margen enfrente de las enmendaduras o interlineaciones, y las salvará al fin de la diligencia, antes de las firmas.

    Art. 120. (141) El querellante, el Ministerio Público, cuando fuere parte principal, y el que estuviere detenido o hubiere sido declarado procesado, deberán ser citados para cualquiera inspección personal que practique el juez para la averiguación de los hechos.
    Las personas citadas podrán concurrir a la diligencia, debiendo hacerlo personalmente el respectivo oficial del Ministerio Público y en la misma forma o por medio de su procurador o abogado las demás.
    El juez podrá prescindir de la citación y comparecencia antedicha si así conviniere al éxito de la investigación.



    Artículo 120 bis.- Las órdenes de investigar que el juez curse a la Policía de Investigaciones, Carabineros de Chile o Gendarmería, en su caso, facultan a estos organismos para practicar las diligencias que el juez determine y las siguientes, salvo expresa exclusión o limitación:
    1° Conservar las huellas del delito y hacerlas constar;
  2° Recoger los instrumentos usados para llevar a cabo el hecho delictuoso, salvo en cuanto sea necesario mantenerlos en el lugar en que fueron encontrados para su examen personal por el juez;
    3° Hacer constar el estado de las personas, cosas o lugares mediante inspecciones o con los medios a que se refiere el artículo 113 u otras operaciones aceptadas por la policía científica, y requerir la intervención de organismos especializados en la investigación, según la naturaleza del delito;
    4° Citar a los testigos presenciales del hecho delictuoso investigado para que comparezcan al tribunal a primera audiencia, entregándoles una boleta o comprobante de la citación.  Si el testigo no compareciere, el juez podrá ordenar su arresto para obtener la comparecencia.
    Tratándose de los delitos de hurto o robo, requerir del denunciante una declaración jurada sobre la preexistencia de las cosas sustraídas y una apreciación de su valor.
    5° Consignar sumariamente las declaraciones que se allanaren a prestar el inculpado o los testigos, y
    6° Proceder a la citación del inculpado.

  2. De la comprobación del delito en casos
especiales
    I.- Homicidio, aborto y suicidio

    Art. 121. (142) Cuando se sospeche que la muerte de una persona es el resultado de un delito, se procederá, antes de la inhumación del cadáver o inmediatamente después de exhumado, a efectuar la descripción ordenada por el artículo 112, a practicar el reconocimiento y autopsia del cadáver y a identificar la persona del difunto.
    La descripción expresará circunstanciadamente el lugar y postura en que fue hallado el cadáver, el número de heridas o señales exteriores de violencia y partes del cuerpo en que las tenía, el vestido y efectos que le hallaren, los instrumentos o armas encontrados y de que se haya podido hacer uso, y la conformidad de su forma y dimensiones con las heridas y señales de violencia.
    En los casos de muerte causada por vehículos en la vía pública, y sin perjuicio de las facultades que corresponden al juez competente, efectuará la descripción a que se refiere el inciso anterior y ordenará el levantamiento del cadáver un oficial de Carabineros, asistido por un funcionario del mismo servicio, quien actuará como testigo. Se levantará un acta de lo obrado, que firmarán ambos funcionarios, la que se agregará al proceso.

NOTA:
      El considerando 4 de la RES 1345 EXENTA, Servicio Médico Legal, publicada el 13.12.2000, señala que el Código de Procedimiento Penal, establecido por LEY 19696, en su Art. 201 deroga tácitamente el presente artículo, otorgando un nuevo concepto de muerte médico legal, lo que se aplicará gradualmente en las regiones del país de acuerdo a los artículos 483 y siguientes de la LEY 19696.-
    Artículo 122.- La identificación del occiso se hará mediante informes papilares, dactiloscópicos o de otro tipo, o por testigos que, a la vista de él, den razón
satisfactoria de su conocimiento.  Si existe alguna persona a quien se impute el delito, debe ser confrontada con el cadáver para que lo reconozca, siempre que sea posible esta diligencia.

    Artículo 123.- Si no se pudiere identificar en la forma indicada y el estado del cadáver lo permitiere, será expuesto, por lo menos durante veinticuatro horas, en un lugar que tenga acceso al público, y en un cartel fijado allí mismo, se expresarán el sitio, hora y día en que fue encontrado y el nombre del juez que instruye el sumario, a fin de que los que tuvieren algún dato que pueda contribuir al reconocimiento del difunto y a la averiguación del delito y sus circunstancia, lo comuniquen al juez de la causa.

    Art. 124. (145) Si, a pesar de las precauciones de que trata el artículo precedente, el cadáver no fuere reconocido, se hará de él una descripción que contenga sus señales y se conservarán las prendas del traje y los objetos que se le hubieren encontrado, a fin de que puedan servir oportunamente para hacer la identificación. Con el mismo objeto y cuando el caso lo requiera y las circunstancias lo permitan, el juez hará sacar la fotografía del cadáver, de la cual se agregará a los autos un ejemplar y otro conservará el secretario dentro de un sobre lacrado y sellado.
    Art. 125. (146) Aun cuando por la inspección externa del cadáver pueda colegirse cuál haya sido la causa de la muerte, el juez mandará que se proceda por facultativos a la autopsia judicial.
    Esta autopsia consiste en la apertura del cadáver en las regiones en que sea necesario para el efecto de descubrir la verdadera causa de la muerte.
    Puede ser llamado para presenciar la operación el médico que asistió al difunto en su última enfermedad, cuando convenga obtener de él datos sobre el curso de dicha enfermedad.
    Art. 126. (147) Los médicos deben expresar en su informe las causas inmediatas que hubieren producido la muerte, las que hubieren dado origen y, con la mayor aproximación posible, la data de ella.
    Si existen lesiones, deben manifestar su número, longitud y profundidad, la región en que se encuentran, los órganos ofendidos y el instrumento con que han sido hechas, especificando:
    1° Si son resultado de algún acto de tercero;
    2° Si, en tal caso, la muerte ha sido la consecuencia necesaria de tal acto, o si han contribuido a ella alguna particularidad inherente a la persona, o un estado especial de la misma, o circunstancias accidentales, o en general cualquiera otra causa ayudada eficazmente por el acto del tercero; y
    3° Si habría podido impedirse la muerte con socorros oportunos y eficaces.
    Los informes deben redactarse, en cuanto sea posible, en lenguaje vulgar, y responder a las cuestiones precedentes y a las que el juez propusiere sobre todas las circunstancias que interesen para formar juicio cabal de los hechos.

    Artículo 127.- Las autopsias se harán en un local dependiente del Servicio Médico Legal del Estado; donde no lo hubiere, se practicarán en las dependencias que para este fin existan en los Hospitales o Cementerios respectivos y, a falta de los anteriores, en el lugar donde lo ordene el juez.
    Cuando haya necesidad de practicar un reconocimiento a cadáveres en estado de descomposición, éste se hará en los cementerios de las correspondientes ciudades, y si no existe un depósito apropiado para tal intervención médico-legal, en el sitio que determine el juez.

NOTA:  5
    El Art. 10 del Decreto con Fuerza de Ley N° 196, de 25 de marzo de 1960, publicado en el Diario Oficial de 4 de abril del mismo año, que fijó el texto de la Ley Orgánica del Servicio Médico Legal, dispone que a la Sección Tanatología de los Servicios Médico-Legales de cabecera de provincias, departamentos y comunas corresponderán las pericias en cadáveres o restos humanos y que de dicha Sección dependerán las salas de autopsias y sus anexos.
    A su vez, el Art. 9° del Decreto con Fuerza de Ley N° 91 del Ministerio de Hacienda, de 23 de febrero de 1960, publicado en el Diario Oficial de 29 del mismo mes y año, dispone que el Jefe de la Sección Tanatología tiene jurisdicción nacional.
NOTA:  6
    El Art. 2° de la Ley N° 14.657, de 6 de octubre de 1961, dispone que "en las ciudades en que el Servicio Médico Legal carezca de locales especiales para practicar autopsias, los cadáveres serán conducidos para dicho objeto a las dependencias que para este fin existan en los Hospitales respectivos, en donde serán puestos a disposición del Legista de aquel Servicio o a falta de éste, del que designe el tribunal conforme a lo dispuesto por el artículo 224 del Código de Procedimiento Penal.
    En los casos de cadáveres en estado de descomposición, con respecto a los cuales haya necesidad de practicarles un reconocimiento, éste se hará en los cementerios de las correspondientes ciudades, en los cuales será obligatorio que exista un depósito apropiado para la intervención médico-legal que deba realizarse".
NOTA:  7
    Ver D.S. 427, de Justicia, de 28 de enero de 1943, sobre Reglamento Orgánico del Instituto Médico Legal "Dr. Carlos Ybar" y de los Sevicios Médicos Legales del país.
    Artículo 128.-  Corresponderá practicar las autopsias a los médicos que se indican en artículo 221 bis.
      En los lugares en que no haya facultativos que practiquen la autopsia judicial reconocerán el cadáver el juez y dos trestigos, y éstos extenderán sus certificados con los pormenores indicados en el artículo 126, en cuanto les sea posible.  El juez preferirá como testigos a dentistas, veterinarios, profesores, matronas, enfermeras u otras personas que tengan alguna idoneidad para el caso.

    Art. 129. (150) En el caso de muerte por envenenamiento y en todos aquellos en que se sospeche muerte violenta y no aparezcan lesiones exteriores que puedan haberlas causado, el juez hará reconocer los sitios en que estuvo el difunto inmediatamente antes de su muerte y con especialidad su casa, para ver si en aquéllos o en ésta se encuentran venenos o rastros de cualquiera especie que acredite haberse hecho uso de ellos, recogerá los que hallare, y pondrá testimonio en los autos de todas aquellas señales que contribuyan a impedir que se puedan confundir con otras, tales como la cantidad, color, peso y otras cualidades específicas.
    Estos objetos quedarán depositados en poder del secretario, quien los guardará en caja o lugar cerrado y sellado y no los sacará durante el proceso sino cuando sea necesario practicar su examen y en la cantidad que baste para ese fin.

    Art. 130. (151) En caso de presunto envenenamiento, las substancias sospechosas encontradas en el cadáver o en otra parte, serán analizadas por el funcionario especialmente encargado de ese género de operaciones y en su defecto, por químico o farmacéutico designado por el juez.
    El juez podrá ordenar que se haga el análisis con el concurso o bajo la dirección de un médico.
    Art. 131. (152) Cuando se extraiga del agua un cadáver, se averiguará principalmente:
    1° Si la muerte ha sido resultado de la asfixia producida por el agua;
    2° Si ha sido causada por alguna enfermedad de que padeciera el difunto; y
    3° Si la sumersión fue posterior a la muerte.
    Artículo 132.- En el caso de presunción de muerte causada por atropellamiento de algún vehículo, el facultativo tendrá especial cuidado de investigar si existen en el cadáver señales que manifiesten que la muerte se había producido con anterioridad o si fue resultado de las lesiones ocasionadas por el atropello.

    Artículo 133.- Si se pesquisa el delito de infanticidio, el juez tratará de acreditar, por los medios legales y especialmente por informe de facultativos, si la presunta madre estuvo embarazada, cuál fue la época probable del parto, si la criatura nació viva y en estado de poder vivir fuera del seno materno, las causas que probablemente han producido la muerte, las horas que permaneció viva, y si en el cadáver se notan lesiones.

    Art. 134. (155) En el caso de aborto, se hará constar la existencia de la preñez, la época del embarazo, los signos demostrativos de la expulsión del feto, las causas que la hubieren determinado y la circunstancia de haber sido provocado por la madre, o por un extraño que hubiere procedido, ya sea con su consentimiento, ya sea ejecutando en ella actos de violencia, ya, por fin, abusando de su oficio de facultativo.
    Cuando el feto muerto en el vientre materno no hubiere sido expulsado, se averiguará también si por acción provocada se puso fin al desarrollo intrauterino.

    Art. 135. (156) Si se encontrare ahorcado a un individuo la investigación se dirigirá principalmente a establecer:
    1° Si el sujeto fue ahorcado vivo o suspendido después de muerto; y
    2° Si se ahorcó a sí mismo o fue ahorcado por otro.
    Art. 136. (157) Si se presumiere que ha habido suicidio, debe procederse a averiguar si alguien prestó ayuda a la víctima y en qué consistió la cooperación.
    Art. 137. (158) Si el cadáver ha sido sepultado antes del examen pericial, y las circunstancias permitieren creer que la autopsia puede practicarse útilmente y sin peligro para la salud de los que deben ejecutarla, el juez dará aviso al administrador del cementerio de que va a proceder a la exhumación, indicándole el día y hora en que se va a practicar.
    Trasladándose en seguida a ese establecimiento, acompañado de uno o más facultativos, averiguará el sitio donde fue sepultado el cadáver, lo hará exhumar y lo identificará en la forma dispuesta por el artículo 122, siendo ello posible, o con el testimonio de las personas que lo inhumaron o de otras que puedan reconocer al difunto.
    Practicadas estas diligencias de que se pondrá testimonio en autos, y ejecutada la operación pericial, el cadáver será nuevamente sepultado.

    II.- Lesiones corporales

    Artículo 138.- Toda persona a cuyo cargo inmediato se encuentre un hospital u otro establecimiento de salud semejante, sea público o privado, dará en el acto cuenta al juzgado del crimen de la entrada de cualquier individuo que tenga lesiones corporales, indicando brevemente el estado del paciente y la exposición que hagan el o las personas que le hubieren conducido acerca del origen de dichas lesiones y del lugar y estado en que se le hubiere encontrado. La denuncia deberá consignar el estado del paciente, describir los signos externos de las lesiones e incluir la exposición que hagan el afectado, o las personas que lo hubieren conducido, acerca del origen de dichas lesiones y del lugar y estado en que se le hubiere encontrado.
    En ausencia del jefe del establecimiento, dará cuenta el que le subrogue en el momento de entrada del enfermo.
    El incumplimiento de la obligación prevista en este artículo se castigará con la pena que señala el artículo 494 del Código Penal.

    Art. 139. (160)  Siempre que llegue al conocimiento del juez, sea por el medio indicado en el artículo anterior o por cualquiera otro, que una persona ha sido herida, se trasladará al lugar en que se encuentre el herido con el fin de tomarle declaración, y dispondrá que uno o más facultativos procedan al examen de las lesiones.
    Si en el lugar no hubiere médico, el reconocimiento será hecho por el juez, asociado de dos testigos y se pondrá en autos testimonio de lo que observaren. Se considerará especialmente a los testigos que tengan alguna de las calidades mencionadas en el artículo 128.
    La descripción de las lesiones contenida en la denuncia a que se refiere el artículo 138, servirá de antecedente suficiente para acreditar la existencia de lesiones leves o menos graves, cuando entre la fecha en que éstas se ocasionaron y aquella en que se practique el examen médico pericial que decrete el tribunal, haya transcurrido un número tal de días que haya hecho desaparecer los signos y efectos de las lesiones.

    Art. 140. (161) El herido prestará su declaración bajo juramento y si, por razón de su estado, no pudiere referir todos los hechos cuyo conocimiento sea indispensable para la instrucción del sumario, debe tratarse de que exprese, a lo menos, quién le infirió las lesiones, para proceder a la citación o captura del inculpado en conformidad a la ley.
    La declaración completa será tomada en forma, tan pronto como el herido pueda prestarla.
    Art. 141. (162) Los facultativos describirán las lesiones, indicando el instrumento con que han sido causadas, su gravedad, los órganos afectados o mutilados, las consecuencias que ordinariamente tienen heridas de esta naturaleza y las que hayan acarreado en el caso actual, y expresarán además el tiempo que, según su parecer, el ofendido permanecerá enfermo o incapacitado para el trabajo a consecuencia de las lesiones.
    Artículo 142.- Mientras lo haga necesario la gravedad de sus lesiones, el herido deberá ser atendido y permanecer en los servicios hospitalarios del Estado, a menos que tuviere derecho o medios para ser atendido por su cuenta en un establecimiento particular.

    Artículo 143.- No siendo necesaria la hospitalización del herido, si careciere de medios para proveer por su cuenta a su curación será atendido en alguno de los establecimientos indicados en el artículo precedente.

    Art. 144. (165) El procesado tendrá también derecho de designar otro médico que, a su costa, intervenga en la asistencia del herido.
    Art. 145. (166) Los médicos que asistan al herido en un hospital u otro establecimiento público o privado semejante o fuera de él, darán parte de su estado en los períodos que el juez señale, y si el herido falleciere o sanare, comunicarán inmediatamente el hecho al mismo juez.
    En caso de muerte, se procederá con arreglo a lo dispuesto en los artículos 125 y 126.
    Si el herido sanare de las lesiones, los médicos, al dar cuenta del hecho, pondrán en conocimiento del juez el tiempo que ha durado la curación, o la circunstancia de haber quedado el ofendido temporal o absolutamente inútil para el trabajo, demente, impotente, impedido de algún miembro importante o notablemente deforme.
    Los datos expresados en los dos incisos anteriores deben constar en autos antes del pronunciamiento de la sentencia definitiva; salvo que el estado del herido permita presumir que no ocurrirá ninguna complicación posterior que influya en la agravación o disminución de la pena.

    III. Delitos sexuales
    Art. 145 bis. Tratándose de los delitos previstos en los artículos 361 a 367 bis y en el artículo 375 del Código Penal, los hospitales, clínicas y establecimientos de salud semejantes, sean públicos o privados, deberán practicar los reconocimientos, exámenes médicos y pruebas biológicas conducentes a acreditar el cuerpo del delito y a identificar a los partícipes en su comisión, debiendo conservar las pruebas y muestras correspondientes.
    Se levantará acta, en duplicado, del reconocimiento y de los exámenes realizados, la que será suscrita por el jefe del establecimiento o de la respectiva sección y por los profesionales que los hubieren practicado. Una copia se entregará a la víctima o a quien la tuviere bajo su cuidado y la otra, así como las muestras obtenidas y los resultados de los análisis y exámenes practicados, se mantendrán en custodia y bajo estricta reserva en la dirección del hospital, clínica o establecimiento de salud, por un período no inferior a un año, para ser remitidos al tribunal correspondiente.
    Las copias del acta a que se refiere el inciso precedente tendrán el mérito probatorio señalado en los artículos 472 y 473, según corresponda.

    IV.- Delitos contra la propiedad

    Art. 146. (167) En los sumarios que se instruyen sobre delitos de hurto, robo, estafa y otros engaños, se acreditará la preexistencia de los objetos substraídos; se comprobará, en cuanto fuere posible, la identidad de los que se encontraren en poder del procesado o de una tercera persona; se reconocerá la fractura de puertas, armarios, arcas u otros objetos cerrados o sellados, y se pondrá testimonio de los rastros o vestigios que hubiere dejado el delito.
    En los delitos de hurto o robo será antecedente suficiente para acreditar la preexistencia de los objetos sustraídos, para todos los efectos procesales, la declaración jurada a que se refiere el inciso cuarto del artículo 83 y el párrafo segundo del número 4° del artículo 120 bis.Las especies recuperadas se entregarán al dueño o legítimo tenedor en cualquier estado del procedimiento una vez comprobado su dominio o tenencia y establecido su valor. En todo caso, se dejará constancia en el expediente, mediante fotografías u otros medios que resultaren convenientes, de las especies restituidas o devueltas por orden del tribunal.".
    En los casos contemplados en el artículo 454 del Código Penal, no se requerirá acreditar la preexistencia de las cosas encontradas en poder del inculpado, ni el dominio ajeno. Ambas circunstancias se presumirán por el solo hecho de que el inculpado no pueda acreditar su legítima tenencia.
    Si del robo con violencia en las personas resultaren homicidios o lesiones, se procederá, además, en la forma que se indica en los artículos precedentes.




    Art. 147. (168) Siempre que sea necesario fijar el valor de la cosa objeto del delito, el juez la hará tasar por peritos. Al efecto, de estar la cosa en poder del tribunal, la entregará a éstos o les permitirá su inspección proporcionándoles los elementos directos de apreciación sobre los que deberá recaer el informe. De no estar la cosa en poder del tribunal, les proporcionará los antecedentes que obren en el proceso, en base a los cuales los peritos deberá emitir su informe.
    Si las cosas han sido hurtadas o robadas en lugar destinado al ejercicio de un culto permitido, se tasarán separadamente los objetos destinados a dicho culto de los que no lo son.
    Si la falta contemplada en el artículo 494 bis del Código Penal se cometiere en un establecimiento de comercio, el juez considerará que su valor corresponde al precio de venta, el cual se informará en el acta a que se refiere el número 4º del artículo 120 bis, salvo que la prueba que se reúna en autos le permita formarse una convicción diferente..
    Se tasarán por separado los animales hurtados o robados y las monturas u otros objetos que con ellos hayan sido substraídos.
    Todo lo anterior es sin perjuicio de lo dispuesto en el artículo 455 del Código Penal.


    Art. 148. (169) Si no se presentare dueño a reclamar las especies al parecer perdidas, que el presunto procesado hubiere encontrado y no entregado al dueño o a la autoridad, apropiándoselas indebidamente, se procederá a rematar dichas especies con las formalidades y para el objeto determinados en los artículos 629 y 634 del Código Civil.


    IV.- Falsedad

    Art. 149. (170) Si se tratare de la falsedad de un instrumento público o privado, en el acto de presentarse será firmado en todas las páginas por el juez y por la persona que lo presente.
    Antes de agregarlo al proceso o de ordenar su custodia, se levantará un acta en la que se expresará el estado material del instrumento y se enunciarán todas las circunstancias que puedan indicar la falsedad o alteración, pudiendo agregarse su fotocopia autorizada.

    Art. 150. (171) El instrumento denunciado como falso será cotejado por peritos con el verdadero.
    Art. 151. (172) En los casos de falsificación de monedas o de documentos de créditos del Estado, de las municipalidades, de los establecimientos públicos, sociedades anónimas o bancos de emisión, u otros que sean registrados en la Casa de Moneda, las monedas o documentos falsificados serán examinados por el Jefe de aquella Casa, a fin de que informe sobre la existencia de la falsificación y sobre la manera como probablemente hubiere sido verificada.
    Art. 152. (173) Todo depositario público o privado de documentos impugnados de falsos está obligado a entregarlos al juez; pero dejará copia autorizada de ellos, cuando deban conservarse en una oficina pública.
    Previa la diligencia indicada en el artículo 149, el documento se agregará al proceso.
    Si éste formare parte de un registro del cual no pudiere ser desglosado, podrá disponer el tribunal que el registro en que figura la pieza se deposite en la secretaría.
    Terminado el juicio se devolverán los documentos y el registro o protocolo a la persona u oficina que los entregó.
    Si se hubiere declarado falso en todo o en parte un instrumento público, el tribunal ordenará al mismo tiempo que su devolución, que se le reconstituya, cancele o modifique, de acuerdo con la sentencia que hubiere expedido.
    Art. 153. (174) Cuando la falsedad consista en haberse contrahecho o fingido letra, firma o rúbrica, el juez hará cotejar por peritos la letra, firma o rúbrica que se dice falsificada con otras cuya autenticidad no ofrezca dudas.
    Puede además el juez ordenar a quien se supone autor del delito que escriba en su presencia algunas palabras o frases, cuando crea que esta diligencia pueda arrojar luz para la averiguación del delito o del delincuente.
    Art. 154. (175) Si, para la existencia del delito, se requiere que haya perjuicio de tercero, el juez investigará en qué consiste este perjuicio.
    VI.- Incendio

    Artículo 155.- En los casos de incendio, deberá el juez inquirir si el fuego ha tenido origen en la casa o establecimiento de algún comerciante.
      Si así fuere y no se descubriere al autor desde el primer momento, hará tomar los libros y papeles del comerciante, averiguará si el establecimiento incendiado estaba o no asegurado, el monto del seguro, y el valor del edificio, mercaderías o muebles objeto del seguro, existentes en la casa o establecimiento en el momento del incendio.
    Lo dicho es sin perjuicio de las normas especiales establecidas en el decreto con fuerza de ley N° 251, de 1931.

NOTA: 8
    Véanse los artículos 30, 31, 32, 33, 34 y 36 del DFL 251, de 20 de mayo de 1931, sobre Compañías de Seguros, Sociedades Anónimas y Bolsas de Comercio.
  3. De la entrada y registro en lugar cerrado, del registro de libros, papeles y vestidos y de la detención y apertura de la correspondencia epistolar y telegráfica

    Artículo 156.- Los tribunales pueden decretar la entrada y registro en cualquier edificio o lugar cerrado, sea público o particular, cuando hubiere indicio de encontrarse allí el inculpado o procesado, o efectos o instrumentos del delito, o libros, papeles o cualesquiera otros objetos que puedan servir para descubrir un delito o comprobarlo.
    El registro deberá hacerse en el tiempo que media entre las siete y las veintiuna horas; pero podrá verificarse fuera de estas horas en casas de juego o de prostitución, o habitada por personas que se hallen sujetas a la vigilancia de la autoridad, o en lugares a que el público tenga libre entrada, como los hoteles y cafés, o en los casos de delitos flagrantes, o cuando urja practicar inmediatamente la diligencia. En este último caso, el decreto será fundado.
    Carabineros de Chile y la Policía de Investigaciones, en caso de delito flagrante y siempre que hubieren fundadas sospechas de que responsables del delito se encuentren en un determinado recinto cerrado, podrán, para los efectos de proceder a su detención, efectuar el registro de inmediato y sin previa orden judicial. El funcionario que practique el registro deberá individualizarse y cuidará que la diligencia se realice causando el menor daño y las menores molestias posibles a los ocupantes del recinto.
    Además, deberá otorgar al propietario o encargado del local, un certificado que acredite el hecho del registro, la individualización de los funcionarios que lo practicaron y de quien lo ordenó. Copia de este certificado deberá adjuntarse al parte policial respectivo, el cual deberá enviarse al tribunal competente, dentro de las 24 horas siguientes de efectuada la diligencia.
    La infracción a las obligaciones establecidas en este artículo, serán sancionada con la pena de reclusión menor en sus grados mínimo a medio.
NOTA:  9
    Véanse la "Convención de Viena sobre relaciones diplomáticas" y la "Convención de Viena sobre relaciones consulares", publicadas en el Diario Oficial los días 4 y 5 de marzo de 1968, respectivamente.
    Art. 157. (178) Salvo en los casos expresados en el tercer inciso del artículo precedente, el registro no se verificará sino después de interrogar al individuo cuya casa o persona hubiere de ser registrada, y sólo si se negare a entregar voluntariamente la cosa que es objeto de la pesquisa o no desvaneciere los motivos que hayan aconsejado la medida.
    En este caso el auto en que el tribunal ordene la diligencia será siempre fundado, debiendo expresar con toda claridad cuál es el edificio o lugar cerrado en que haya de verificarse el registro.
    En ausencia del dueño, se interrogará a las personas indicadas en el artículo 161.
    Art. 158. (179) Para proceder al examen y registro de lugares religiosos, de edificios en que funciona alguna autoridad pública, el juez hará pasar recado de atención a la autoridad o persona a cuyo cargo estuvieren, quien podrá asistir a la operación o nombrar a alguna persona que asista.
    Tratándose de recintos militares o policiales, las diligencias a que se refiere el inciso anterior deberán cumplirse por intermedio de los Tribunales Militares de la correspondiente jurisdicción.

    Artículo 159.- Para la entrada y registro de las casas y naves que, conforme al Derecho Internacional, gozan de inviolabilidad, el juez pedirá su consentimiento al respectivo agente diplomático por oficio, en el cual le rogará que conteste dentro de veinticuatro horas. Este será remitido por conducto del Ministerio de Relaciones Exteriores.
    Si el agente diplomático negare su consentimiento o no contestare en el término indicado, el juez lo comunicará al Ministeroi de Relaciones Exteriores. Mientras el Ministro no conteste manifestando el resultado de las gestiones que practique el juez se abstendrá de entrar al lugar indicado; pero adoptará las medidas de vigilancia que se expresan en el artículo 162.
    En casos urgentes y graves, podrá el juez solicitar la autorización al agente diplomático directamente o por intermedio del secretario, que certificará el hecho de haberse concedido.

    Artículo 160.- Para el registro de los locales consulares o partes de ellos que se utilicen exclusivamente para el trabajo de la oficina consular, se deberá recabar el consentimiento del jefe de la oficina consular o de una persona que él designe, o del jefe de la misión diplomática del mismo Estado.
    Regirá, en lo demás, lo dispuesto en el artículo precedente.

    Art. 161. (182) El auto de entrada y registro se notificará al dueño o arrendatario del lugar o edificio en que hubiere de practicarse la diligencia o al encargado de su conservación o custodia.
    Si no fuere habida alguna de las personas expresadas, la notificación se hará a cualquiera persona mayor de edad que se hallare en dicho lugar o edificio.
    Si no se hallare a nadie, se hará constar esta circunstancia en el acta de la diligencia.


    Artículo 162.- Desde el momento en que el juez decrete la entrada y registro en cualquier edificio o lugar cerrado, adoptará las medidas de vigilancia convenientes para evitar la fuga del inculpado o procesado o la substracción de instrumentos, efectos del delito, libros, papeles o cualesquiera otras cosas que hubieren de ser objeto del registro.


NOTA 1.1
    Las modificaciones introducidas al presente Código por la Ley N° 18.857, publicada en el Diario Oficial de 6 de Diciembre de 1989, rigen, según lo dispone su artículo vigésimo, noventa días después de su publicación en el Diario Oficial.
    Art. 163. (184) Practicadas las diligencias prescritas en los artículos anteriores, se procederá a la entrada y registro, empleando para ello, si fuere necesario, el auxilio de la fuerza.
    Art. 164. (185) En los registros deben evitarse las inspecciones inútiles, procurando no perjudicar ni molestar al interesado más de lo estrictamente necesario. El que lo practique adoptará las precauciones convenientes para no comprometer la reputación de aquél, y respetará sus secretos en cuanto esta reserva no dañe a la investigación.
    Art. 165. (186) El propietario, arrendatario o persona a cuyo cargo esté el local que se registra, será invitado para presenciar el acto; y, si estuviere impedido o ausente, la invitación se hará a un miembro adulto de la familia, o en su defecto, a una persona de la casa o a un vecino.
    INCISO DEROGADO.-

      Todos los concurrentes que pudieren, firmarán el acta que al efecto se levante, y si nada se descubriere de sospechoso en el local registrado, se dará testimonio de ella al interesado, si lo pidiere.

    Art. 166. (187) De los objetos que se recojan durante el registro se formará inventario, que se agregará al proceso, y se dará copia autorizada de dicho inventario al interesado que la pidiere.
    Art. 167. (188) El registro se practicará en un solo acto; pero podrá suspenderse cuando no fuere posible continuarlo, debiendo reanudarse apenas cese el impedimento.
    Suspendido el registro, se cerrarán y sellarán la parte del local y de los muebles en que hubiere de continuarse, en cuanto esta precaución se considere necesaria para evitar la fuga de la persona o la substracción de las cosas que se buscaren. Se adoptarán además en este caso las medidas indicadas en el artículo 162.

    Art. 168. (189) En la diligencia de entrada y registro en lugar cerrado, se expresarán los nombres del juez o funcionario que la practicare y de las demás personas que intervinieren en ella, los incidentes ocurridos, la hora en que hubiere principiado y aquella en que concluyere, la relación del registro en el mismo orden en que se hubiere efectuado y los resultados obtenidos.
    Art. 169. (190) No se practicará el registro de los libros y papeles de contabilidad del procesado o de otra persona sino por el mismo juez y en el único caso de aparecer indicios graves de que esta diligencia haya de resultar el descubrimiento o la comprobación de algún hecho o circunstancia importante en la causa.
    Art. 170. (191) El juez o funcionario comisionado para practicar la diligencia, recogerá los instrumentos y efectos del delito y podrá recoger también los libros, papeles o cualesquiera otros objetos que se hubieren encontrado, si lo estimare conducente para el adelanto de la investigación.
    Los libros y papeles que se recojan serán foliados, sellados y rubricados en todas sus hojas por el juez y el secretario. En un certificado autorizado por este último funcionario y firmado por los asistentes al acto se consignará el número de fojas útiles que se contienen en dichos libros o papeles y de él se dará copia al interesado, si la pidiere.
    Art. 171. (192) Toda persona que tenga objetos o papeles que puedan servir para la investigación será obligada a exhibirlos y entregarlos.
    Si la persona que los tenga o bajo cuya custodia o autoridad estén, rehúsa la exhibición, podrá ser apremiada del mismo modo que el testigo que se niega a prestar declaración, salvo que fuere de aquéllas a quienes la ley autoriza para negarse a declarar como testigo.
    Tratándose de documentos que tengan el carácter de secretos de acuerdo a las disposiciones del Código de Justicia Militar, se aplicará lo dispuesto en el artículo 53 bis de este Código.
    No serán aplicables respecto de los documentos a que se refiere este artículo, los N°s. 3° y 4° del artículo 455 del Código Orgánico de Tribunales.

    Artículo 172.- Podrá el juez, siempre que sus ocupaciones no le permitan proceder por sí mismo, encargar al secretario, acompañado de la fuerza pública si fuere necesario, la entrada y registro en lugar cerrado de que se trata en el presente párrafo.
    Los papeles objeto del registro sólo podrá examinarlos el juez y no el secretario comisionado, a menos que sea expresamente autorizado.
    En casos calificados, el juez, además, podrá encargar a Carabineros de Chile o a la Policía de Investigaciones la entrada y registro en lugar cerrado, conforme a lo establecido en el artículo 156. La orden respectiva deberá señalar el lugar preciso del registro, su finalidad y las especies que se ordena incautar, en su caso. En el evento que disponga el retiro de libros, papeles, registros o documentación mercantil o privada, el funcionario que realice la diligencia, sin perjuicio de estar autorizado para identificarlos, no podrá imponerse de su contenido y se limitará a su retiro en paquetes que sellará. Deberá dar recibo detallado de lo incautado al propietario o encargado del lugar. Los paquetes sólo podrán ser abiertos por el juez, en presencia del secretario, levantándose acta de lo obrado.

    Art. 173. (194) Cuando se tratare sólo de aprehender a una persona, el juez podrá comisionar para esta diligencia a un agente de policía, autorizándolo para entrar en edificio o lugar cerrado. El juez observará previamente, en su caso, las prescripciones de los artículos 158, 159 y 160.
    Art. 174. (195) El ejecutor de la orden de aprehensión presentará copia autorizada de ella al dueño de casa o, a falta de éste, a cualquiera de las personas que ahí se encuentren; y si ninguna persona apareciere en ella, la leerá en alta voz y la fijará en la puerta de calle.
    Acto continuo procederá al registro, sin emplear fuerza sino para abrir las puertas o ventanas en los lugares que se le resistieren, respetando las personas a quienes no se refiera el mandamiento.
    Terminado el registro, el ejecutor tomará las precauciones convenientes para evitar perjuicios al dueño de la casa allanada.
    Art. 175. (196) Podrá el juez ordenar el registro de los vestidos que actualmente lleven personas respecto de quienes haya indicios para creer que ocultan en ellos objetos importantes para la investigación o comprobación de un delito.
    Para practicar este registro se comisionará a personas del mismo sexo de la registrada y se guardarán a éstas las consideraciones compatibles con la correcta ejecución del acto.
    Artículo 176.- Podrá el juez ordenar la retención de la correspondencia privada, sea postal, telegráfica o de otra clase, que el procesado o inculpado remitiere o recibiere, y la de aquella que, por razón de especial circunstancias, se presuma que emana de él o le está dirigida, aún bajo nombre supuesto, siempre que se pueda presumir que su contenido tiene importancia para la investigación.  También podrá emitir  esta orden el juez respecto de cualquier otro objeto que remitiere o recibiere el procesado.
    El decreto del juez se hará saber a los jefes de los respectivos servicios de comunicaciones para que lleven a efecto la retención de la correspondencia, que entregarán bajo recibo al secretario del juzgado.

    Artículo 177.- El juez podrá, asimismo, ordenar que por cualquiera empresa de telégrafos o cables, o de otros sistemas de comunicación semejantes, se le faciliten copias de los telegramas, cablegramas o comunicaciones transmitidos o recibidos por ella, si lo estimare conveniente para el descubrimiento o comprobación de algún hecho de la causa. Podrá, además, exigir las versiones que existieren de las transmisiones por radio o televisión.

    Artículo 178.- La apertura y registro de la correspondencia de que tratan los dos artículos precedentes, se decretará por resolución fundada, en la cual se determinará con la precisión posible la correspondencia que debe ser objeto de esta medida.

    Artículo 179.- DEROGADO.-


    Art. 180. (201) El juez abrirá por sí mismo la correspondencia y, después de leerla para sí, apartará la que haga referencia a los hechos de la causa y cuya conservación considerare necesaria.
    En seguida tomará las notas que convenga para practicar las investigaciones a que la correspondencia diere lugar, rubricará y hará que los asistentes rubriquen los sobres y hojas, los sellará con el sello del juzgado, y, encerrándolo todo en otro sobre, al cual pondrá un rótulo para su reconocimiento, lo conservará en su poder durante el sumario bajo su responsabilidad.
    Este cierro podrá abrirse cuantas veces el juez lo estime necesario, y cada vez que se habra se citará al interesado para que presencie la operación en la forma dispuesta en los artículos precedentes.
    Art. 181. (202) La correspondencia que no se relacionare con la causa será entregada en el acto a la persona a quien pertenezca o a la que ésta comisionare al efecto o, a falta de ambas, a algún miembro inmediato de su familia.
    En todo caso, será devuelta la correspondencia una vez terminado el sumario.
    La que hubiere sido extraída de servicios de comunicaciones será devuelta a ellos después de cerrada y escrita nuevamente la dirección que antes tenía, dándoseles para su resguardo el certificado correspondiente.

    Art. 182. (203) Si durante la pesquisa de que se trata en el presente párrafo, se descubrieren objetos o datos que permitan sospechar la existencia de un delito distinto del que es materia del sumario y del cual nazca acción pública, el juez, si tuviere competencia para el juzgamiento del nuevo delito, extenderá a él sus investigaciones, formando o no proceso separado según las reglas legales; pero si careciere de competencia, se limitará a recoger los datos u objetos encontrados y a ponerlos a disposición del juez correspondiente con una relación de los antecedentes del caso.
    En tal evento, se observarán las disposiciones establecidas en los artículos precedentes.
    Artículo 183.- Las disposiciones de este párrafo no obstan a las atribuciones que las leyes confieren a la autoridad administrativa en materia de allanamientos; pero desde que la autoridad judicial comience a obrar en un proceso, aquella se abstendrá de toda medida que con él se relacione, a menos que sea expresamente requerida por el juez de la causa.

    4. De los documentos

    Artículo 184.- Para que los instrumentos públicos sean eficaces en juicio, se requiere:
    1° Que los traídos al proceso, sean puestos en conocimiento de la otra parte y cotejados con los originales, si los hubiere, si alguno de los interesados solicitare esta diligencia.
    Tratándose de documentos originales o que carecen de matriz, podrá pedirse, para que sean eficaces, que se reconozcan por el funcionario autorizante y, si no pudieren ser reconocidos por éste, el cotejo de firma o letra en la forma prevista en el artículo 188. El juez decretará esta diligencia si no la ha ordenado de oficio y sólo cuando los instrumentos tengan trascendencia para el resultado del proceso;
    2° Que los testimonios o certificados sean expedidos por el funcionario determinado por la ley, por el encargado del archivo o registro en que se hallen los originales, o por el secretario de la causa, y
    3° Que, si no se presenta íntegro el instrumento, se le adicione, a petición de cualquiera de los interesados o de oficio, con las otras partes de él que tengan relación con el proceso.

    Artículo 185.- El cotejo de instrumentos públicos se hará por el juez o por el secretario de la causa, cuando el primero lo ordene.


    Art. 186. (207) Los instrumentos extendidos en idioma diverso del castellano, que las partes presentaren al juicio, serán acompañadas de su respectiva traducción.
    El juez, de oficio o a petición de parte, podrá ordenar que la traducción sea revisada por un perito, que designará al efecto.
    Si el documento fuere agregado por orden del juez expedida de oficio, será mandado traducir por un perito; y se agregarán a los autos el original y la traducción.
    Los instrumentos públicos otorgados fuera de Chile se legalizarán en la forma expresada en el artículo 345 del Código de Procedimiento Civil.
    Artículo 187.- Los instrumentos privados deben ser reconocidos por las personas que los han escrito o firmado. Este reconocimiento se efectuará en la forma de una confesión o de una declaración de testigos, según emanare de alguna de las partes o de otra persona.
    Empero, si pareciere que la exhibición de estos instrumentos a tales personas hubiere de frustrar las diligencias del sumario, se podrá entretanto establecer la procedencia u origen de dichos instrumentos por declaraciones de testigos o por cualquier otro medio probatorio.

    Art. 188. (209) Si se negare o pusiere en duda la autenticidad de un instrumento privado, el juez nombrará dos peritos calígrafos para que cotejen la letra o firma del documento con la de otra que sea realmente de la persona a quien se atribuya.
  5. De las declaraciones de testigos

    Art. 189. (210) Toda persona que resida en el territorio chileno y que se hallare legalmente exceptuada, tiene obligación de concurrir al llamamiento judicial para declarar en causa criminal cuando supiere sobre lo que el juez le preguntare, si para ello ha sido citada con las formalidades prescritas por la ley.
    Todo testigo consignado en el parte policial, o que se presente voluntariamente a Carabineros de Chile, a la Policía de Investigaciones, o al tribunal, podrá requerir de éstos la reserva de su identidad respecto de terceros.
    Las autoridades referidas deberán dar a conocer este derecho al testigo y dejar constancia escrita de su decisión, quedando de inmediato afectas a la prohibición que se establece en el inciso siguiente.
    Si el testigo hiciere uso de este derecho, queda prohibida la divulgación, en cualquier forma, de su identidad o de antecedentes que conduzcan a ella. El tribunal deberá decretar esta prohibición. La infracción a esta norma será sancionada con la pena que establece el inciso segundo del artículo 240 del Código de Procedimiento Civil, tratándose de quien proporcione la información. En caso que la información sea difundida por algún medio de comunicación social, su director será castigado con una multa de diez a cincuenta ingresos mínimos mensuales.
    Esta prohibición regirá hasta el término del secreto del sumario.
    Sin perjuicio de lo anterior, el juez, en casos graves y calificados, podrá disponer medidas especiales destinadas a proteger la seguridad del testigo que lo solicite. Dichas medidas durarán el tiempo razonable que el tribunal disponga y podrán ser renovadas cuantas veces fueren necesarias.

    Art. 190. (211) El testigo que legalmente citado no compareciere podrá ser compelido por medio de la fuerza a presentarse ante el tribunal que haya expedido la citación, a menos que compruebe que ha estado en la imposibilidad de concurrir.
    Si compareciendo se negare sin justa causa a declarar, podrá ser mantenido en arresto hasta que preste su declaración.
    Todo lo cual se entiende sin perjuicio de la responsabilidad penal que pueda afectar al testigo rebelde.
    Ninguna de estas medidas se hará efectiva contra el testigo que comprobare su imposibilidad para asistir al llamamiento del juez.
    Art. 191. (212) No están obligados a concurrir al llamamiento judicial de que se trata en los artículos precedentes:
    1° El Presidente de la República y los ex Presidentes; los Ministros de Estado;
los Subsecretarios; los senadores y diputados; el Contralor General de la República y los ex Contralores Generales; los intendentes y los gobernadores, dentro del territorio de su jurisdicción; los miembros de la Corte Suprema o de alguna Corte de Apelaciones; los fiscales de estos tribunales; los ex Ministros de la Corte Suprema; los jueces letrados; los Oficiales Generales en servicio activo o en retiro; el arzobispo y los obispos; los vicarios generales y los vicarios capitulares;
    2° Las personas que gozan en el país de inmunidades diplomáticas;
    3° Las religiosas y las mujeres que por su estado o posición no pueden concurrir sin grave molestia; y
    4° Los que, por enfermedad u otro impedimento calificado por el juez, se hallen en imposibilidad de hacerlo.

    Artículo 192.- Las personas comprendidas en el número 1° del artículo precedente, prestarán su declaración por medio de informe, expresando que lo hacen bajo el juramento o promesa que la ley exige a los testigos; pero los miembros y fiscales de los tribunales superiores no declararán sin permiso de la Corte respectiva, quien no lo concederá si juzga que sólo se trata de establecer una causal de recusación contra el miembro o fiscal de ella, cuya declaración se solicita.
    Podrán también ser examinados en su domicilio o en su residencia oficial, previo aviso y fijación de día y hora, siempre que el juez de la causa lo estimare necesario para los efectos a que se refiere el inciso segundo del artículo 198, para lo cual dictará un auto motivado ordenando la práctica de la diligencia.
    Las personas comprendidas en el número 2° declararán también por medio de informe, si se prestan a ello voluntariamente y al efecto, se les dirigirá oficio respetuoso por intermedio del Ministerio respectivo. El chileno que ejerza en el país funciones diplomáticas por encargo de un Gobierno extranjero, no podrá negarse a informar.
    Los comprendidos en los números 3° y 4° serán examinados en su propia morada por el juez de la causa, acompañado del secretario.
    El privilegio a que se refieren este artículo y el anterior es esencialmente renunciable, compareciendo a declarar ante el tribunal.  Si informaren, deberán hacerlo en el término de diez días contados desde la remisión del oficio correspondiente, hecho que se certificará en el proceso. Si no lo hicieren dentro de dicho plazo, deberán comparecer a declarar, previa citación.
    Si en concepto del juez no fuere necesaria la comparecencia personal de un jefe de servicio de la Administración del Estado a declarar sobre hechos relativos a esas instituciones, podrá limitarse a recabar de ellos un informe escrito prestado bajo juramento o promesa, el que se apreciará conforme a las reglas de la prueba testimonial.
    Las personas a que se refiere el inciso precedente y aquellas no comprendidas en el artículo 191, respecto de quienes leyes especiales prescriban que prestarán su declaración por oficio, estarán obligadas a concurrir al tribunal, si éste estima que la declaración que hubieren hecho mediante informe es insuficiente para los fines de la investigación.

NOTA:  10
    Ver artículo 192 de la ley 10336 sobre Organización y Atribuciones de la Contraloría General de la República.
    Art. 193. (214) El juez hará concurrir a su presencia y examinará por sí mismo a los testigos indicados en la denuncia, querella o auto cabeza de procesos, o en cualesquiera otras declaraciones o diligencias y a todos los demás que supieren hechos o circunstancias, o poseyendo datos convenientes para la comprobación o averiguación del delito y del delincuente.
    Art. 194. (215) El juez mandará extender orden de citación para cada persona designada como testigo que, residiendo en el territorio de su jurisdicción, no fuere de las exceptuadas por el artículo 191.
    Esta orden será firmada por el secretario y en ella se expresarán el día, hora y lugar que el testigo debe presentarse.
    Cuando sea urgente el examen de un testigo, podrá citársele verbalmente para que comparezca en el acto sin esperar que se expida orden escrita de citación; pero se hará constar en los autos el motivo de la urgencia.
    Artículo 195.- La citación se notificará por carta certificada, dejándose testimonio en el expediente de la fecha de entrega de la carta a la oficina de correos, la individualización de dicha oficina y el número del comprobante emitido por ella, el cual se adherirá al expediente a continuación del testimonio.
    La notificación se entenderá practicada al quinto día hábil siguiente a la fecha recién aludida. Si la carta certificada fuera devuelta por la oficina de correos por no haberse podido entregar al destinatario, se adherirá al expediente.
    Con todo, en casos excepcionales y por resolución fundada el juez podrá ordenar que la notificación se practique por cédula, en los términos que contempla el inciso segundo del artículo 196.

    Artículo 196.- El testigo que no compareciere a la citación, notificada en la forma prevista en los incisos primero y segundo del artículo 195, será nuevamente notificado, esta vez por cédula, previo decreto judicial.
    La notificación la efectuará cualquier ministro de fe o empleado del tribunal comisionado para ello. El encargado de practicar la diligencia certificará el día y hora en que hubiera ejecutado la orden recibida o el inconveniente que haya impedido darle cumplimiento, de lo cual pondrá el secretario testimonio en autos.

    Art. 197. (218) Si la persona llamada a declarar ejerce funciones de servicio público que no puedan ser desamparadas, conjuntamente con citarla el juez dará aviso de la citación al jefe respectivo. Este tomará inmediatamente las providencias necesarias para que, sin daño del servicio, sea cumplida la orden del juez.
    Tratándose de personal de las Fuerzas Armadas y Carabineros de Chile, que no esté exento de la obligación de concurrir, el juez de la causa podrá encomendar la práctica de la diligencia al juez militar de instrucción que corresponda en virtud del exhorto que se regirá por las disposiciones señaladas en el artículo 198.
    Si el tribunal no hiciere uso de la facultad contemplada en el inciso anterior, deberá comunicar la citación al jefe respectivo quien hará que se cumpla la orden del tribunal. No obstante, si razones impostergables de servicio lo hicieren necesario, la autoridad militar o de carabineros correspondiente podrá solicitar al juez de la causa que proceda en la forma establecida en el inciso segundo del artículo 192. Si el juez denegare esta solicitud, deberá hacerlo en un auto motivado.

    Art. 198. (219) Si el testigo reside en un lugar distinto de aquel en que tiene su territorio jurisdiccional el tribunal que instruye el sumario, será examinado por el juez de letras o de policía local a quien se cometa la diligencia, en virtud del exhorto en que se expresen los hechos, citas y preguntas acerca de los cuales deba ser interrogado.
    Pero si el juez de la causa estime necesario oír por sí mismo al testigo, para la comprobación del delito, para el reconocimiento de la persona del delincuente o para otro objeto, igualmente importante, puede ordenar en auto motivado que el testigo comparezca ante él.
    El exhorto, una vez cumplido, será devuelto cerrado y sellado al tribunal de su origen, quien mandará agregarlo al sumario.

    Art. 199. (220) Si el testigo se encontrare en el extranjero, se dirigirá por la vía diplomática una carta rogatoria al tribunal del lugar en que aquél residiere o se hallare actualmente, a fin de que le tome su declaración. Dicha carta contendrá los antecedentes necesarios e indicará las preguntas que deban hacerse al testigo, sin perjuicio de que dicho juez las amplíe según le sugieran su discreción y prudencia.
    La carta contendrá la promesa de reciprocidad, y será examinada por la Corte Suprema antes de que este tribunal la remita al Ministerio de Relaciones Exteriores para hacerla llegar al tribunal a quien va dirigida.
    No obstante, los funcionarios del servicio diplomático o consular chileno que se encuentren en el extranjero, deberán declarar por oficio, si así lo ordena el juez. El oficio se dirigirá por intermedio del Ministerio de Relaciones Exteriores al jefe de aquellos servicios, según corresponda, para que proceda a darle cumplimiento en un breve plazo y devolver lo obrado por la misma vía.

    Art. 200. (221) Si el testigo no tuviere domicilio conocido o si se ignorare su paradero, el juez dictará las órdenes convenientes para que la policía lo averigüe.
    Art. 201. (222) No están obligados a declarar:
    1° El cónyuge del procesado, sus ascendientes o descendientes legítimos o ilegítimos reconocidos, sus parientes colaterales legítimos dentro del cuarto grado de consanguinidad o segundo de afinidad, sus hermanos naturales, su pupilo o su guardador; y
    2° Aquellas personas que, por su estado, profesión o función legal, como el abogado, médico o confesor, tienen el deber de guardar el secreto que se les haya confiado, pero únicamente en lo que se refiere a dicho secreto.



    Art. 202. (223) El juez advertirá al testigo que se halle comprendido en el número 1° del artículo anterior que no tiene la obligación de declarar en contra del procesado; pero que puede hacer las manifestaciones que considere oportunas, y se consignará la contestación que diere a esta advertencia. El testigo podrá retractar cuando quisiere el consentimiento que hubiere dado para prestar su declaración.
    Los testigos comprendidos, tanto en el número 1° como en el 2° del artículo precedente, estarán obligados a declarar respecto de los demás procesados a quienes no estén ligados con las relaciones que en dichos números se expresan, a no ser que su declaración pueda comprometer a aquellos con quienes tienen esa relación.
    Artículo 203.- Todo testigo, antes de ser interrogado, prestará juramento o promesa de decir la verdad sobre lo que fuere preguntado, sin añadir ni ocultar nada de lo que conduzca al esclarecimiento de los hechos.
    No se tomará juramento ni promesa a los menores de dieciséis años, a las personas indicadas en el número 1° del artículo 201, ni a aquellos de quienes se sospeche que hayan tomado parte en los delitos que se investigan, en calidad de autores, cómplices o encubridores.

    Art. 204.(225) Inmediatamente después de haber prestado juramento o promesa el testigo, el juez le instruirá de la obligación que tiene de ser veraz y de las penas con que la ley castiga el delito de falso testimonio en causa criminal.
    Podrá omitirse esta instrucción respecto de aquellos testigos que manifiestamente no la necesiten.

    Art. 205. (226) Salvo los casos exceptuados por la ley, los testigos serán examinados separada y secretamente por el juez en presencia del secretario.
    Art. 206. (227) Se comenzará el examen por aquellos a quienes se presuma sabedores del hecho, entre los que deben contarse el ofendido, las personas de su familia y aquellas que dieron parte del delito.
    Art. 207. (228) Todo testigo comenzará por manifestar su nombre y apellidos paterno y materno, su edad, lugar de su nacimiento, su estado, profesión, industria o empleo, la casa en que vive, si conoce al ofendido y al procesado, y si tiene con ellos parentesco, amistad o relaciones de cualquier clase.
    Cuando lo estime necesario, podrá también el juez interrogar al testigo sobre si ha estado alguna vez preso y cuál ha sido el resultado del proceso a que se le hubiere sometido.
    Art. 208. (229) El juez dejará que el testigo que no sea de referencia, narre sin interrupción los hechos sobre los cuales declara, y solamente le exigirá las explicaciones complementarias que sirvan para desvanecer la obscuridad o contradicción de que pudieran adolecer algunos conceptos.
    Después le dirigirá las preguntas que crea oportunas para el esclarecimiento de los hechos.
    Art. 209. (230) Los testigos de referencia serán examinados al tenor de las citas que de ellos se hubieren hecho.
    Al efecto, el juez les manifestará cuál es el punto sobre que versa la cita y les dará todas las explicaciones convenientes para la recta inteligencia del negocio. Instruidos de esta manera, contestarán afirmativa o negativamente sobre los hechos, y si agregaren algunos esclarecimientos, se expresarán en la diligencia en que se consigne la declaración.
    Art. 210. (231) Todo testigo debe explicar circunstanciadamente los hechos sobre que declara y dar razón de su dicho, expresando si los ha presenciado, si los deduce de antecedentes que conoce, o si los ha oído referir a otras personas, cuyo nombre y residencia determinará en cuanto le sea posible.
    Art. 211. (232) La declaración se prestará de viva voz y sólo se permitirá que el testigo consulte apuntes o memorias escritas cuando se trate de consignar datos minuciosos o complicados, que sea difícil retener en la memoria.
    Art. 212. (233) El juez podrá ordenar que se conduzca al testigo al lugar en que hubieren ocurrido los hechos, a fin de examinar allí, o poner a su presencia los objetos sobre los cuales hubiere de versar la declaración.
    Podrá también hacer que el testigo describa circunstanciadamente dichos objetos y que los reconozca entre otros semejantes, o adoptar las medidas que su prudencia le sugiera para asegurarse de la exactitud de la declaración.
    Art. 213. (234) No se harán al testigo preguntas capciosas ni sugestivas, ni se empleará coacción, promesa, engaño ni artificio alguno para obligarlo o inducirlo a declarar en determinado sentido.
    Art. 214. (235) Si el testigo no supiere el idioma castellano, será examinado por medio de un intérprete mayor de dieciocho años, quien prometerá bajo juramento desempeñar bien y fielmente el cargo.
    Por conducto del intérprete se interrogará al testigo y se recibirán sus contestaciones, las cuales serán consignadas en el idioma del testigo, si éste no entendiere absolutamente el castellano. En tal caso, se pondrá al pie de la declaración la traducción que de ella haga el intérprete.
    Art. 215. (236) Si el testigo fuere sordo, las preguntas le serán dirigidas por escrito; y si fuere mudo, dará por escrito sus contestaciones.
    Si no fuere posible proceder de esta manera, la declaración del testigo será recibida por intermedio de una o más personas que puedan entenderse con él por signos, o que comprendan a los sordo-mudos. Estas personas prestarán previamente el juramento de que se trata en el primer inciso del artículo precedente.
    Art. 216. (237) Terminada la declaración, se la extenderá por escrito en el proceso; y el testigo podrá, bajo la dirección del juez, dictar por sí mismo sus contestaciones.
    La diligencia comenzará por expresar la fecha en que se hubiere practicado y los demás pormenores indicados en el artículo 207.
    Redactada la declaración, será leída por el testigo, o por el secretario o por el intérprete en su caso, si aquél no pudiere o no quisiere hacerlo después de advertido de su derecho para leer la declaración por sí mismo, de todo lo cual se pondrá testimonio al pie de ella.
    El testigo podrá hacer las enmiendas, adiciones o aclaraciones que tenga a bien, todo lo cual se expresará con claridad a la conclusión de la diligencia, sin enmendar por eso lo que en ella se hubiere escrito.
    La diligencia será firmada por el juez, y demás personas que hubieren intervenido en ella si pudieren hacerlo, y autorizada por el secretario. Si alguno se negare a firmar, se hará mención de esta circunstancia.
    Art. 217. (238) No se consignarán en el proceso las declaraciones de los testigos que, en concepto del juez, fueren manifiestamene inconducentes para la comprobación de los hechos objeto del sumario. Tampoco se consignarán en cada declaración las manifestaciones del testigo que se hallaren en el mismo caso, pero sí todo lo que pueda servir tanto de cargo como de descargo.
    En el primer caso se pondrá testimonio de la comparecencia del testigo y del motivo de no escribirse su declaración.
    Art. 218. (239) El juez hará saber al testigo la obligación que tiene de comparecer a declarar cada vez que se le cite y a poner en conocimiento de dicho juez cualquier cambio de domicilio o de morada que efectúe dentro de los cuatro meses subsiguientes a su declaración, o hasta que se ratifique durante el plenario, en el caso de que se pida esta diligencia.
    De estas prevenciones se dejará testimonio al final de la declaración, y al margen de la misma se anotarán los cambios de domicilio o de morada del testigo.
    Cesará la obligación que el inciso 1° impone al testigo si, pedida su ratificación, se practica la diligencia antes del vencimiento del plazo señalado en dicho inciso.
    Art. 219. (240) Si al hacérsele las prevenciones de que habla el artículo precedente, manifestare el testigo la imposibilidad de concurrir durante el plenario, por tener que ausentarse a larga distancia, o si hubiere motivo para temer que le sobrevengan la muerte o una incapacidad física o moral que le impida ratificarse en tiempo oportuno, o si, por no tener el testigo residencia fija, sea probable que no se le encuentre más adelante, el juez inmediatamente o con el intervalo que estime oportuno para no frustrar los fines del sumario, pondrá en conocimiento del procesado la declaración del testigo, a fin de que exprese si exige o no que se lleve a efecto la diligencia de la ratificación.
    En caso afirmativo, se procederá a practicar dicha diligencia, con citación del procesado, del Ministerio Público y del querellante, todos los cuales y además los abogados del primero y del último, podrán presenciar la diligencia y hacer al testigo, por conducto del juez, cuantas preguntas tengan por conveniente, excepto las que el juez desestime como manifiestamente impertinentes.


    Art. 220. (241) El testigo que viviere solamente de su jornal diario, tendrá derecho a que la persona que lo presente le indemnice de la pérdida de tiempo que le ocasionare su comparecencia para prestar declaración o para practicar otra diligencia de interés en el juicio.
    Se entenderá renunciado este derecho si no se ejerciere en el plazo de veinte días contados desde la fecha en que se presta la declaración o se practica la diligencia.
    En caso de desacuerdo, estos gastos serán regulados por el tribunal sin forma de juicio y sin ulterior recurso.
    Si la diligencia ha sido practicada de oficio o a petición del Ministerio Público o de una parte que goce del beneficio de pobreza, la indemnización será pagada por la respectiva Municipalidad, que podrá repetir contra el civilmente responsable, en el caso de que alguno fuere declarado tal.
    6. Del informe pericial

    Artículo 221.- El juez pedirá informe de peritos en los casos determinados por la ley, y siempre que para apreciar algún hecho o circunstancia importante, fueren necesarios o convenientes conocimientos especiales de alguna ciencia, arte u oficio.
    En las comunas o agrupaciones de comunas en que exista un servicio público costeado con fondos fiscales o municipales destinado a practicar actuaciones o diligencias periciales de la naturaleza de las requeridas por el tribunal para el proceso, deberá encargarse de preferencia a dicho servicio evacuar el respectivo dictamen pericial y en caso de que alguno de los empleados de esa oficina sea designado nominativamente para efectuar la diligencia, no tendrá remuneración especial por esta labor.
    Si en la comuna o agrupación de comunas no existiere dicho servicio, los peritajes se encargarán de preferencia a los servicio o empresas fiscales, semifiscales, de administración autónoma o municipales que tengan idoneidad para practicarlos; pero si alguno de sus funcionarios fuere designado nominativamente para llevar a cabo la pericia, tendrá derecho a honorarios.
    No obstante lo dispuesto en los incisos anteriores, el tribunal podrá, cuando fuere necesario, designar peritos que figuren a las listas a que se refiere el inciso siguiente.
    Las listas de peritos serán propuestas cada dos años por la Corte de Apelaciones respectiva, previa determinación del número de peritos que en su concepto deban figurar en cada especialidad.
    En el mes de octubre del final del bienio correspondiente, se elevarán estas listas a la Corte Suprema, la cual formará las listas definitivas, pudiendo suprimir o agregar nombres sin expresar causa.
    Estas listas definitivas de peritos serán publicadas en el Diario Oficial en la primera quincena del mes de enero y regirán durante dos años desde la fecha de su publicación.
    Se entenderá que pertenecen de pleno derecho a las listas los institutos científicos de las universidades, las personas que los integren y las que profesen docencia universitaria, aunque no figuren en ellas.
    Con todo, el juez podrá designar como peritos a otras personas.
    El Presidente de la República, por intermedio del Ministerio de Justicia, fijará, periódicamente y cuando lo estime conveniente, el arancel que deberán cobrar los peritos nombrados en virtud de lo dispuesto en el inciso cuarto de este artículo.

    Artículo 221 bis.- En los casos de autopsia o examen médico, el juez deberá designar al legista correspondiente, a menos que existan razones que aconsejen el nombramiento de un perito diverso en determinado caso, las que se expondrán en auto motivado.
    En los lugares en que los médicos legistas no tuvieren la especialidad precisa que requiera el informe, se designará otro u otros peritos, según las reglas establecidas en el artículo 221.

    Art. 222. (243) Sólo en defecto de personas que tengan título profesional conferido conforme a la ley, podrán ser nombrados en el carácter de peritos personas no tituladas, pero que tengan competencia especial en la materia sobre que debe versar el informe.
    En los delitos contra la honestidad se hará recaer el nombramiento, siempre que fuere posible sin contrariar la disposición del inciso precedente, en persona del mismo sexo de aquella que debiere ser reconocida.
    Art. 223. (244) El tribunal determinará, en cada caso, si el reconocimiento debe hacerse por uno o más peritos.
    Artículo 224.-  En los juicios en que se ejercite la acción pública, el nombramiento de perito corresponde al juez, quien habrá de designar como tal, en los casos de autopsia o examen médico, al facultativo que lo sea de la ciudad o, a falta de éste, al de la Municipalidad respectiva, a menos de existir razones especiales que aconsejen la designación de otro y que se expondrán en auto motivado.
    Sin perjuicio de lo dispuesto en el inciso anterior, puede cada parte nombrar a su costa, un perito que se asocie al designado por el juez, salvo que, en concepto del tribunal, la intervención de estos peritos pudiera perjudicar al éxito de las investigaciones. Los trámites de nombramiento y aceptación del cargo no retardarán en este caso el reconocimiento, y sólo podrá nombrarse un perito por todos los querellantes y otro por las demás partes, y cuando aquéllos o éstas sean varios. No obstante, el juez, por motivos calificados, podrá aceptar un mayor número de peritos por cada parte y determinar si deben actuar en forma separada o asociados.

    Artículo 225.- Si las partes hicieren uso de la facultad que les concede el inciso segundo del artículo anterior, manifestarán al juez el nombre del perito, y ofrecerán, al hacer esta manifestación, los comprobantes de tener dichos peritos título profesional conferido conforme a la ley, salvo el caso de excepción indicado en el artículo 222.
    En ningún caso podrán hacer uso de dicha facultad después de iniciada la diligencia pericial.
    El juez resolverá sobre la admisión de estos peritos breve y sumariamente, procediendo en la forma determinada para las recusaciones en el artículo 234.

    Artículo 226.- El nombramiento se hará saber a los peritos por medio de oficio o de notificación. La notificación podrá hacerse en casos urgentes, y previo decreto del juez, por un oficial del tribunal o por un agente de policía.

    Artículo 227.- Toda persona designada como perito está obligada a aceptar el encargo que se le confía, siempre que esté oficialmente comisionada para este objeto, como el médico legista, o que tenga título oficial, o que ejerza públicamente la ciencia, arte, u oficio cuyo conocimiento se juzga necesario para el informe pericial.
    Las personas que no se hallaren en ese caso o que tengan algún impedimento, deberán exponer su excusa al juez, dejándose constancia en el acto de la notificación o manifestándola por escrito presentado en el mismo día.

    Art. 228. (250) El perito que, sin alegar excusa, o cuya excusa sea desechada por el juez, se negare a desempeñar el encargo podrá ser apremiado en la forma establecida para los testigos en el artículo 190.
    Art. 229. (251) No podrán prestar informe pericial acerca del delito las personas a quienes el artículo 201 exime de la obligación de declarar como testigos.
    Art. 230. (252) El nombramiento de peritos se notificará inmediatamente a las partes.
    Artículo 231.- Los peritos nombrados por el tribunal podrán ser recusados por las partes en virtud de una causa legal. Los que fueren designados por las partes podrán ser tachados del mismo modo que los testigos, durante el plenario.

    Art. 232. (254) Son causas de recusación de los peritos:
    1a. El parentesco de consanguinidad dentro del cuarto grado o de afinidad dentro del segundo con el querellante o con el querellado o el procesado;
    2a. El interés directo o indirecto en la causa o en otra semejante; y
    3a. La amistad íntima con la parte contraria o la enemistad manifiesta con el que recusa.




NOTA 1.1
    Las modificaciones introducidas al presente Código por la Ley N° 18.857, publicada en el Diario Oficial de 6 de Diciembre de 1989, rigen, según lo dispone su artículo vigésimo, noventa días después de su publicación en el Diario Oficial.
    Art. 233. (255) La parte que intente recusar a un perito, deberá hacerlo por escrito antes de empezar la diligencia pericial, expresando la causa de la recusación, el nombre y residencia de los testigo de que piensa valerse, y acompañando la prueba documental, o designando el lugar en que ésta se halle, si no la tuviere a su disposición.
    Artículo 234.- Cuando la causa alegada fuere una de las señaladas en el artículo 232 y el perito la reconozca como cierta, el tribunal podrá tenerlo por recusado y designar otro perito de inmediato, si estima que el fundamento reconocido es suficiente para configurarla.
    Si el perito no la reconoce o el juez estima que el motivo aceptado es insuficiente, ordenará que se agregue a los autos la prueba documental de que haya hecho mención el recusante y mandará que comparezcan los testigos indicados, notificando previamente a las partes la resolución correspondiente y fijando para ello un plazo de hasta diez días.
    En el día y hora fijados y en presencia de las partes que concurran, examinará en forma legal a los testigos, acerca de la existencia de la causa de recusación. Con el mérito  de estas declaraciones o el de la prueba instrumental rendida, se pronunciará sin más trámites sobre la recusación; y si da lugar a ella, procederá a nombrar nuevo perito.
    Si la causa no fuere una de las designadas en el artículo 232, o si no se ofreciere prueba para acreditarla, el juez rechazará de plano la recusación.  Las resoluciones a que se refiere este artículo no son apelables.

    Art. 235. (257) Si la diligencia de reconocimiento encomendada a peritos no pudiere retardarse, deberá procederse a efectuarla, no obstante la recusación.
    Artículo 236.- Las personas que por razón de su cargo están llamadas a desempeñar ordinariamente las funciones de perito, prestarán una sola vez juramento o promesa de buen desempeño del encargo en la forma que se indica en el inciso siguiente, ante el juez del crímen, y si hubiere dos o más en la comuna o agrupación de comunas, ante el del Primer Juzgado.  De este juramento o promesa se pondrá testimonio en el libro de decretos económicos y será comunicado a los otros jueces para su debida constancia en el libro indicado del respectivo tribunal.
    Los demás peritos prestarán juramento o promesa ante un ministro de fe, de que emitirán su parecer con imparcialidad, en el menor tiempo posible y conforme a los principios de la ciencia o reglas del arte u oficio que profesan, y de que guardarán reserva sobre los datos y conclusiones de su informe.

    Art. 237. (259) El informe pericial se presentará por escrito y contendrá:
    1° La descripción de la persona o cosa que sea objeto de él, del estado y del modo en que se hallare;
    2° La relación circunstanciada de todas las operaciones practicadas y de su resultado; y
    3° Las conclusiones que, en vista de tales datos, formulen los peritos conforme a los principios o reglas de su ciencia, arte u oficio.
    Si las circunstancias lo exigieren, y sin perjuicio del informe definitivo, el juez podrá pedir a los peritos un preinforme provisorio, del que se dejará constancia en autos en la forma de una declaración.

    Art. 238. (260) Si para verificar el reconocimiento fuere necesario alterar o destruir la cosa que ha de reconocerse, se la dividirá, si fuere posible, y se reservará una parte, la cual se conservará intacta y en seguridad bajo el sello del juzgado para reiterar la operación, si llegare a ser necesario.
    Art. 239. (261) Podrán las partes asistir a los reconocimientos y someter a los peritos las observaciones que estimaren convenientes, salvo que el tribunal estime que la presencia de ella es ofensiva a la moral o perjudicial a la investigación.
    Si concurrieren las partes, deberá también asistir el juez o cometer la diligencia al secretario, quien pondrá testimonio en autos de las observaciones que hicieren.
    Art. 240. (262) El juez, de oficio o a instancia de las partes presentes en el reconocimiento, podrá hacer a los peritos, cuando produzcan de palabra sus conclusiones, las preguntas que estimare pertinentes; o pedirles, cuando las produzcan por escrito, las aclaraciones necesarias.
    Las contestaciones o aclaraciones de los peritos se considerarán como parte de su informe.

    Art. 241. (263) Si las opiniones de los peritos nombrados estuvieren en desacuerdo, el juez designará uno o más, según los casos, para que en compañía de los primeros procedan a practicar de nuevo la operación y a emitir otro informe.
    Si no fuere posible repetir la operación, los nuevos peritos se limitarán a deliberar con los demás en vista de los reconocimientos practicados, y a formular, de acuerdo o separadamente, sus conclusiones motivadas.
    Podrá además el juez, si lo creyere indispensable, remitir los informes periciales a alguna corporación científica del Estado o particular para que, examinando detenidamente las diversas conclusiones formuladas, emita su parecer acerca de las cuestiones debatidas, con arreglo a los principios de la ciencia que les sean aplicables.

    Art. 242. (264) El juez facilitará a los peritos los medios materiales necesarios para practicar las diligencias que les encomiende y los pedirá a la autoridad local, si no los tuviere a su disposición.
    Los peritos podrán también pedir que se les proporcionen los datos que juzguen indispensables para formar su opinión, ya sea por la lectura de algunas piezas del sumario, ya por la interrogación de las personas que figuran en el proceso como partes o testigos, acerca de puntos determinados. El juez dará lugar a estas diligencias, siempre que no existan motivos especiales que lo impidan.
    Artículo 243.- DEROGADO.-


    Artículo 244.- El informe pericial deberá ser presentado al juez dentro del quinto día a contar desde que se notifique a los peritos su nombramiento; pero si se necesitare de más tiempo para preparar el informe, el juez señalará un plazo razonable para que le sea presentado. Con todo, podrá exigir que en un plazo menor se le presente un informe provisorio.
    En caso de desobediencia, podrá el juez aplicarles una multa de medio a dos ingresos mínimos, junto con prescindir de su informe, y decretará el nombramiento de nuevos peritos.  Las medidas aplicadas serán comunicadas a la Corte de Apelaciones para los efectos de la formación de las listas a que se refiere el artículo 221.

    Artículo 245.- Cuando los peritos nombrados por el juez en los juicios en que se ejercita la acción pública, no desempeñaren el encargo según lo dispuesto en el inciso segundo del artículo 221, tendrán derecho por los servicios que se les encomienden, a un honorario que será tasado por el juez de la causa y pagado por el Fisco, quien podrá repetir contra la parte que sea condenada en las costas del juicio.
    De la solicitud de cobro de honorarios de los peritos, se dará traslado al Fisco por el término de diez días. Dicha solicitud, que se presentará por separado y no necesitará cumplir con los requisitos de designación de abogado patrocinante y apoderado, deberá ir acompañada de una copia del respectivo informe pericial.
    En las comunas o agrupaciones de comunas en que no haya abogado-procurador fiscal, se notificará al presidente del Consejo de Defensa Fiscal o al abogado- procurador fiscal de la jurisdicción correspondiente. En este caso, el plazo indicado en el inciso anterior se aumentará con el emplazamiento a que se refiere el artículo 259 del Código de Procedimiento Civil.
    Las reglas del inciso anterior se aplicarán a las notificaciones de la resolución judicial que regule los honorarios y al plazo para interponer la apelación.
    Sólo será necesario el trámite de la consulta para las resoluciones que ordenen el pago de honorarios por una cantidad superior a diez unidades tributarias mensuales para cada perito.  De la consulta y de la apelación conocerá la Corte de Apelaciones, en cuenta, salvo que el tribunal autorice alegatos, en cuyo caso ordenará traer los autos en relación.
    Cuando se cobren honorarios por pericias medico- legales de acuerdo con los montos determinados en el arancel a que se refiere el inciso final del artículo 221, las solicitudes deberán presentarse directamente al Servicio Médico Legal, acompañadas de dos copias del informe evacuado y de una certificación del tribunal que las decretó, que acredite el hecho de haberse evacuado el informe en los términos indicados en la respectiva solicitud.
    El Servicio Médico Legal, en representación del Fisco, procederá a pagar los honorarios devengados y comunicará tal circunstancia por oficio al Consejo de Defensa del Estado y al juzgado respectivo, para los efectos previstos en el inciso primero.

  Título IV
  DE LA CITACION, DETENCION, PRISION PREVENTIVA Y DEL
ARRAIGO

    Art. 246. (268) Todo individuo contra quien las diligencias del sumario arrojen datos que hagan presumir su responsabilidad penal, quedará sujeto a la obligación de comparecer ante el juez de la causa o a la restricción de su libertad personal en la forma determinada en este título.
    Si la obligación de comparecer fuere para prestar declaración y ésta afectare a alguna de las personas señaladas en los números 1°, 3° y 4° del artículo 191, el juez de la causa concurrirá personalmente a interrogarla en su morada o domicilio, previo aviso y fijación del día y hora en que se llevará a efecto la diligencia, salvo que el citado comparezca voluntariamente ante el Tribunal. En lo demás, serán aplicables las normas del párrafo siguiente.

  Párrafo 1°
  De la citación

    Artículo 247.- Para el efecto de que el inculpado preste declaración y para que, sometido a proceso comparezca a los demás actos del juicio, el juez se limitará a citarlo cuando tenga domicilio conocido y el delito que se le imputa fuere alguno de los siguientes:
    1° Cualquiera infracción sancionada con pena de falta;
    2° Delitos que la ley pene únicamente con inhabilitación para cargos u oficios públicos o profesiones titulares, o con suspensión de ellos, o con multa, y
    3° Simples delitos que la ley pene con una sanción privativa o restrictiva de libertad no superior a la de una temporal menor en su grado mínimo.
    Lo dicho en los dos últimos números no se aplicará a los casos en que la detención o prisión, en vista de lo que aparece en el sumario, se considere indispensable para la seguridad personal del ofendido o para que no se frustren las investigaciones que deban practicarse, según las circunstancias del delito o las condiciones personales del imputado; mas, llenados estos fines, el inculpado o procesado será puesto en libertad.





NOTA 1.1
    Las modificaciones introducidas al presente Código por la Ley N° 18.857, publicada en el Diario Oficial de 6 de Diciembre de 1989, rigen, según lo dispone su artículo vigésimo, noventa días después de su publicación en el Diario Oficial.
    Artículo 248.- La citación a que se refiere el artículo precedente se hará en la forma prevenida en los artículos 194 y 195. Si el inculpado o procesado no compareciere, ésta se practicará en forma personal por cualquier ministro de fe o empleado del tribunal y contendrá el apercibimiento de que, si no comparece, se librará contra él orden de detención o de prisión, según los casos.

    Artículo 249.- Si el citado con arreglo a lo prevenido en el artículo anterior, no compareciere ni justificare causa legítima que se lo impida, el juez hará efectivo el apercibimiento despachando la orden correspondiente.

    Art. 250. (272) Sometido a proceso el individuo a quien se imputa alguno de los delitos expresado en el artículo 247, quedará obligado a presentarse a todos los actos del juicio y a la ejecución de la sentencia, bajo apercibimiento de decretarse en su contra orden de prisión, si pasare más de dos días sin concurrir al juzgado cuando sea necesario.
    Rindiendo fianza bastante en concepto del juez, podrá hacerse representar por un procurador del número en todos los actos del juicio en que no fuere indispensable su comparecencia personal.

  2. De la detención
    I. Régimen General



    Art. 251. (273) Para asegurar la acción de la justicia, podrán los jueces decretar la detención de una persona en la forma y en los casos determinados por la ley.

    Art. 252. (274) Por la detención se priva la libertad por breve tiempo a un individuo contra quien aparecen fundadas sospechas de ser responsable de un delito, o a aquel contra quien aparece motivo que induzca a creer que no ha de prestar a la justicia la cooperación oportuna a que lo obliga la ley, para la investigación de un heho punible.
    Art. 253. (275) Ningún habitante de la República puede ser detenido sino por orden de funcionario público expresamente facultado por la ley y después de que dicha orden le sea intimada en forma legal, a menos de ser sorprendido en delito flagrante, y, en este caso, para el único objeto de ser conducido ante el juez competente.
    Art. 254. (276) La detención podrá verificarse:
    1° Por orden del juez que instruye un sumario o conoce del delito;
    2° Por orden de un Intendente Regional o Gobernador Provincial en los casos que designe la ley;
    3° Por un agente de policía en los casos expresamente determinados por la ley; y
    4° Por cualquiera persona, cuando se trate de un delincuente sorprendido in fraganti, para el solo efecto de conducirlo ella misma o por medio de la policía, ante el juez competente.

    Art. 255. (277) El juez que instruye un sumario podrá decretar la detención:
    1° Cuando, estando establecida la existencia de un hecho que presente los caracteres de delito, tenga el juez fundadas sospechas para reputar autor, cómplice o encubridor a aquel cuya detención se ordene;
    2° Cuando en el lugar de la ejecución de un delito se encontraren reunidas varias personas en los momentos en que haya sido perpetrado y el juez cree necesario o conveniente practicar las diligencias indagatorias que correspondan;
    3° Cuando la indagación del delito exigiere la concurrencia de alguna persona para prestar informe o declaración y ésta se negare a comparecer; y
    4° Cuando haya temor fundado de que el testigo se oculte, se fugue o se ausente y su deposición se considerare necesaria para el esclarecimiento del delito y averiguación de los culpables.
    Art. 256. (278) Todo tribunal, aunque no ejerza jurisdicción en lo criminal, podrá dictar orden de detención contra las personas que, dentro de la sala de su despacho, cometieren algún crimen o simple delito.
    Los jueces de letras y los jueces que practican las primeras diligencias del sumario podrán también dictar órdenes de detención en los casos expresados en los artículos 6° y 7° de este Código, conformándose a las disposiciones del presente Título.

    Art. 257. (279) DEROGADO.-


    Artículo 258.- Los Intendentes Regionales y Gobernadores Provinciales podrán dictar orden de detención, siempre que estimen fundadamente que hay verdadero peligro en dejar burlada la acción de la justicia por la demora en recabarla de la autoridad judicial, para aprehender y poner de inmediato a disposición de dicha autoridad a los presuntos culpables de los siguentes delitos:
    1° Crímenes y simples delitos contra la seguridad exterior y soberanía del Estado, o contra la seguridad interior, previstos en los Títulos I y II del Libro II del Código Penal y en la Ley de Seguridad del Estado;
    2° Falsificación de monedas, papel moneda, instrumentos de crédito del Estado, de establecimientos públicos y sociedades anónimas o de bancos e instituciones financieras legalmente autorizadas;
    3° Crímenes o simples delitos de tráfico de estupefacientes;
    4° Crímenes o simples delitos que la ley tipifique como conductas terroristas;
    5° Crímenes y simples delitos de sustracción y secuestro de personas, y
    6° Cualquier crímen o simple delito perseguible de oficio cometido en la sala o recinto en que el Intendente Regional o Gobernador Provincial desempeña sus funciones y en los momentos en que las ejerce;

    Artículo 259.- Los alcaldes podrán dictar orden de detención contra los responsables de los delitos señalados en los números 4°, 5° y 6° del artículo precedente, cuando la demora en recabarla de la autoridad competente pueda dejar burlada la acción de la justicia.
    Las personas aprehendidas por estos funcionarios serán puestas inmediatamente a disposición del tribunal que corresponda.

    Art. 260. (282) Los agentes de policía están obligados a detener:
    1º. A todo delincuente de crimen o simple delito a quien se sorprenda in fraganti;
    2º. Al sentenciado a las penas de presidio, reclusión o prisión que hubiere quebrantado su condena, y
    3º. Al detenido o preso que se fugare.
    En los casos señalados, la detención podrá hacerse en los lugares o establecimientos a que tenga acceso el público, como los locales de espectáculos, cafés, restaurantes, hoteles, prostíbulos y otros semejantes, sin la necesidad de la orden correspondiente para la entrada a dichos sitios.
    La detención del que se encuentre en los casos previstos en el párrafo segundo del número 6° del artículo 10 del Código Penal, se hará efectiva en su casa. Carabineros o la Policía de Investigaciones, según el caso, deberá dar inmediata cuenta de los hechos al juez del crimen, para los efectos de lo previsto en el Título IX del Libro II de este Código.
    Si el detenido tuviese su casa fuera de la ciudad donde funciona el tribunal competente, la detención se hará efectiva en la casa que aquél señale dentro del territorio jurisdiccional de dicho tribunal.

    Artículo 260 bis.- La policía deberá solicitar la identificación de cualquier persona en casos fundados, tales como la existencia de un indicio de que ella hubiere cometido o intentado cometer un crimen, simple delito o falta, de que se dispusiere a cometerlo, o de que pudiere suministrar informaciones útiles para la indagación de un crimen, simple delito o falta. La identificación se realizará en el lugar en que la persona se encontrare, por medio de documentos de identificación expedidos por la autoridad pública, como cédula de identidad, licencia de conducir o pasaporte.
    El funcionario policial deberá otorgar a la persona facilidades para encontrar y exhibir estos instrumentos.
    Durante este procedimiento, la policía podrá proceder al registro de las vestimentas, equipaje o vehículo de la persona cuya identidad se controla.
    En caso de negativa de una persona a acreditar su identidad, o si habiendo recibido las facilidades del caso no le fuere posible hacerlo, la policía la conducirá a la unidad policial más cercana para fines de identificación. En dicha unidad se le darán facilidades para procurar una identificación satisfactoria por otros medios distintos de los ya mencionados, dejándola en libertad en caso de obtenerse dicho resultado. Si no resultare posible acreditar su identidad, se le tomarán huellas digitales, las que sólo podrán ser usadas para fines de identificación y, cumplido dicho propósito, serán destruidas.
    En cualquier caso que hubiere sido necesario conducir a la persona a la unidad policial, el funcionario que practique el traslado deberá informarle verbalmente de su derecho a que se comunique a su familia, o a la persona que indique, de su permanencia en el cuartel policial. Asimismo, no podrá ser ingresada en celdas o calabozos.
    El conjunto de procedimientos detallados en los incisos precedentes no podrá extenderse por un plazo superior a seis horas, transcurridas las cuales la persona que ha estado sujeta a ellos deberá ser puesta en libertad, salvo que existan indicios de que ha ocultado su verdadera identidad o ha proporcionado una falsa, caso en el cual se estará a lo dispuesto en el inciso siguiente.
    Si la persona se niega a acreditar su identidad o se encuentra en la situación indicada en el inciso anterior, se procederá a su detención, debiendo ser puesta a disposición del tribunal como autora de la falta prevista y sancionada en el Nº5 del artículo 496 del Código Penal.
    Los procedimientos dirigidos a obtener la identificación de una persona en conformidad a los incisos precedentes deberán realizarse en la forma más expedita posible, y el abuso en su ejercicio podrá ser constitutivo del delito previsto y sancionado en el artículo 255 del Código Penal.

    Artículo 261.- La policía podrá detener al que sorprenda infraganti cometiendo una falta, si no tuviere un domicilio conocido ni rindiere caución en la forma prevista por el artículo 266, de que comparecerá a la presencia judicial en la audiencia inmediata sin necesidad de otra citación.

    Artículo 262.- Cualquiera persona puede detener a un delincuente a quien sorprenda infraganti, para el efecto de ponerlo de inmediato y directamente o por medio de la policía, a disposición del juez a quien corresponda el el conocimiento del negocio.

    Art. 263. (285) Se reputa delincuente flagrante:
    1° Al que actualmente está cometiendo un delito;
    2° Al que acaba de cometerlo;
    3° Al que en los momentos en que acaba de cometerse huye del lugar en que se cometió y es designado por el ofendido u otra persona como autor o cómplice;
    4° Al que, en un tiempo inmediato a la perpetración del delito, fuere encontrado con objetos procedentes del delito o con señales en sí mismo o en sus vestidos que induzcan a sospechar su participación en él, o con las armas o instrumentos que se emplearon para cometerlo; y
    5° Al que personas asaltadas o heridas, o víctimas de un robo o hurto, que reclaman auxilio, señalen como autor o cómplice de un delito que acaba de cometerse.
    Art. 264. (286) Si el aprehendido en delito flagrante es presentado inmediatamente al juez competente, éste procederá a tomar declaración al aprehensor, a los testigos presenciales que concurran y a interrogar al detenido; y en vista de estas investigaciones lo dejará en libertad o mantendrá la detención, o la convertirá en prisión preventiva, según proceda de derecho.
    Si el aprehensor es un agente de policía, se tendrán como testimonios legalmente prestados sus declaraciones contenidas en las comunicaciones o partes que se envíen al tribunal, con la firma del funcionario aprehensor y la de su superior jerárquico. Si el juez estima estrictamente necesaria la comparecencia personal del funcionario policial, deberá adoptar las medidas para que sea atendido con preferencia a los demás citados y a primera hora de la audiencia respectiva.

    Art. 265. (287) Si, por no ser hora de despacho o por otra causa, el detenido por delito flagrante no fuere conducido en el acto ante el juez, el jefe de la unidad policial o de la casa de detención que recibiere al detenido, hará que la persona que lo conduzca le deje por escrito y bajo su firma una exposición del hecho que motivó la aprehensión  y la designación de su propio domicilio.
    Si esta persona no supiere firmar, lo harán por ella dos testigos llamados al efecto.

    Artículo 266.- Si el delito flagrante que se imputa a la persona detenida fuere alguno de los mencionados en el artículo 247, el funcionario encargado del recinto policial al que sea conducida deberá ponerla en libertad, intimándola para que comparezca ante el juez competente a la primera hora de la audiencia inmediata, cumpliéndose uno de los siguientes requisitos: a) que el detenido acredite tener domicilio conocido, o b) que rinda en dinero efectivo una fianza de comparecencia ascendente a media unidad tributaria mensual si se tratare de una falta y a una unidad tributaria mensual si se tratare de un delito o cuasidelito. La fianza será recibida por el mismo funcionario y podrá ser depositada por el propio detenido o por cualquiera persona a su nombre.
    Se darán al detenido las facilidades pertinentes para que pueda cumplir con cualquiera de estos requisitos.
    Art. 267. (289) Cuando el delincuente flagrante o el detenido conforme al artículo 260 no fuere presentado inmediatamente al juez respectivo, el funcionario que lo reciba en calidad de detenido lo pondrá a disposición del juez con los antecedentes del caso, a primera hora de la audiencia más próxima o antes si éste así lo ordena.
    El juez procederá a practicar en el acto las diligencias indicadas en el artículo 264.

    Art. 268. (290) El particular que detuviere a otro justificará, si éste lo exige, haber obrado en virtud de motivos bastantes para estimar que el aprehendido era en realidad un delincuente flagrante.
    Art. 269. (291) La detención decretada por otra autoridad que no sea el juez, no durará sino hasta que el detenido sea puesto a disposición del juez competente; lo cual se verificará en el acto o, si no fuere hora de despacho, a primera hora de la audiencia inmediata.
    En todo caso, el juez podrá ordenar que en cualquier momento se le ponga a su disposición.
    Este magistrado apreciará las piezas o antecedentes que se le hubieren transmitido, y mantendrá el decreto de detención o lo suspenderá según el mérito que ellos arrojen.

    Artículo 270.- DEROGADO

    Artículo 270 bis.- El plazo máximo de la detención en los casos de los artículos 267 y 269 será de veinticuatro horas para el del delito flagrante y de cuarenta y ocho horas en los demás.
    Antes de vencer estos plazos, deberá ponerse al detenido a disposición del juez, sin perjuicio de que pueda tener lugar, en su caso, el régimen especial establecido en el artículo 272 bis.

    Art. 271. (293) Si el juez a cuya disposición se pusiere a los individuos detenidos por la policía, con arreglo a los números 1° y 2° del artículo 260, fuere el propio de la causa, procederá según corresponda a la situación y estado de ésta.
    Si aquél no fuere el juez de la causa, extenderá una diligencia en que se expresen la persona que hubiere hecho la detención, su domicilio y demás circunstancias bastantes para buscarla e identificarla, los motivos que ella manifestare haber tenido para la detención, y el nombre, apellido y circunstancia del detenido; interrogará a éste y asentará su contestación.
    Inmediatamente después remitirá la diligencia y la persona del detenido al juez a quien correspondiere conocer la causa.
    Pero si el detenido comprobare sumariamente que no es la persona condenada o procesada a quien se trataba de aprehender, será puesto en libertad.
    Art. 272. (294) La detención no podrá durar en ningún caso más de cinco días, contados desde que el aprehendido sea puesto a disposición del tribunal, y terminará, aún antes de ese plazo, en los casos siguientes:
    1° Cuando el inculpado fuere procesado o cuando, por no existir mérito suficiente para hacer esta declaración, el juez ordenare que sea puesto en libertad;
    2° Tratándose del caso previsto en el número 2° del artículo 255, la detención terminará en el acto de recibirse las declaraciones o informaciones de las personas allí expresadas, siempre que no resulten complicadas en el hecho que la ha motivado; y
    3° En los casos de los números 3° y 4° del mismo artículo, la detención se limitará al tiempo necesario para tomar declaración al testigo, o para que preste el informe si fuere un perito.
    El juez deberá recibir esa declaración o ese informe inmediatamente después de encontrarse el testigo o el perito a su disposición.

  II.- Plazos excepcionales de detención.

  Artículo 272 bis.- El juez  podrá, por resolución fundada, ampliar hasta un total de cinco días el plazo de cuarenta y ocho horas de detención ordenada o practicada por cualquiera otra autoridad.
    Cuando se investiguen hechos calificados por la ley como conductas terroristas, el juez podrá ampliar el plazo de cuarenta y ocho horas hasta un total de diez días.
    En la misma resolución que amplíe el plazo, en cualquiera de los casos señalados en los incisos precedentes, el Tribunal ordenará que el detenido sea examinado por el médico que el Juez designe, el cual deberá practicar el examen e informar al Tribunal el mismo día de la resolución, debiendo en todo caso cumplirse con lo preceptuado en el artículo 290. El nombramiento en ningún caso podrá recaer en un funcionario del organismo policial que hubiere efectuado la detención o en cuyo poder se encontrare el detenido.
    Tanto la petición de ampliación que en su caso hiciere la Policía como la orden del juez, deberán constar por escrito.
    La petición de ampliación será considerada denuncia para todos los efectos legales.
    El juez podrá denegar la ampliación sin expresar causa; o concederla por resolución fundada, si estima que la ampliación es útil para el éxito de las indagaciones, mediante orden escrita y firmada en que se mencione el nombre del detenido, el período que durará la detención y el nombre y grado del Jefe de Policía bajo cuya responsabilidad quedará el detenido debiendo el Juez velar en todo momento por la debida protección de éste.
    El juez podrá revocar en cualquier momento la ampliación que hubiere concedido y ordenar que se le envíe el detenido inmediatamente a su disposición sin expresar motivo; determinar el lugar de la detención; disponer que se lleve a su despacho para interrogarlo; decretar exámenes médicos, pudiendo recaer la designación en cualquier facultativo de su confianza; levantar la incomunicación o establecer un régimen de incomunicación determinado; visitar al detenido; y hacerse informar acerca de las pesquisas realizadas y las que se verifiquen durante la detención.
    Expirada la autorización del juez, el detenido será siempre y de inmediato llevado a su presencia y disposición.
    Estos plazos se contarán desde la detención ordenada o practicada por otra autoridad, y sólo podrán prorrogarse hasta el número de días que falten para completar los cinco o los diez a que se refiere este artículo.
    Tanto la petición como una copia de la orden deberá siempre agregarse al proceso; otra copia expedida por el juez, deberá ser entregada al detenido tan pronto como sea recibida la orden por la policía.
    La negligencia grave del juez en la debida protección del detenido será considerada como infracción a sus deberes, de acuerdo con el artículo 324 del Código Orgánica de Tribunales.

    Art. 273. (295) Cuando se ejercite la acción privada que procede de los delitos de injuria o calumnia y no se recibiere información previa, no habrá lugar a detención, sino sólo a prisión preventiva, si ésta fuere procedente con arreglo a la ley.
  3. DEL PROCESAMIENTO Y LA PRISION PREVENTIVA


    Artículo 274.- Después que el juez haya interrogado al inculpado, lo someterá a proceso, si de los antecedentes resultare:
      1° Que está justificada la existencia del delito que se investiga, y
    2° Que aparecen presunciones fundadas para estimar que el inculpado ha tenido participación en el delito como autor, cómplice o encubridor.
    El juez procesará al inculpado por cada uno de los hechos punibles que se le imputen, cuando concurran las circunstancias señaladas.

    Artículo 275.- La resolución en que el inculpado sea sometido a proceso o mandado poner en libertad será fundada y expresará si se han reunido o no las condiciones determinadas en el  artículo 274.
    La que lo somete a proceso enunciará, además, los antecedentes tenidos en consideración y describirá sucintamente los hechos que constituyan las infracciones penales imputadas.
    En la misma resolución, el juez ordenará la filiación del procesado por el servicio correspondiente y concederá la excarcelación al procesado, fijando en su caso la cuantía de la fianza, cuando el delito por el cual se le enjuicia haga procedente ese beneficio en alguna de las formas previstas en los artículos 357 ó 359, a menos que exista motivo para mantenerlo en prisión preventiva, el que deberá expresar.
    Si fuere necesario, las decisiones a que se refiere el inciso precedente podrán ser dictadas en resoluciones separadas.




NOTA 1.1
      Las modificaciones introducidas al presente Código por la Ley N° 18.857, publicada en el Diario Oficial de 6 de Diciembre de 1989, rigen, según lo dispone su artículo vigésimo, noventa días después de su publicación en el Diario Oficial.
    Artículo 276.- La resolución que somete a proceso al imputado será notificada al privado de libertad en la forma establecida en el artículo 66.
    Si el procesado se encontraré en libertad y tuviere apoderado o mandatario constituido en el proceso, se notificará a éste por cédula. De no tenerlo, el tribunal arbitrará las medidas para su más pronta notificación personal.

    Artículo 277.- Por el procesamiento la detención se convierte en prisión preventiva.


    Artículo 278.- El procesado es parte en el proceso penal y deben entenderse con él todas las diligencias del juicio. Su defensa es obligatoria.
    En el acto de ser notificado de la resolución que lo somete a proceso, el encausado debe indicar el nombre del abogado y del procurador a quienes confía su defensa y representación, si antes no lo hubiere hecho, bajo apercibimiento de quedarle designados el abogado y el procurador de turno.
    La designación por el procesado de abogado y procurador particulares en el acto de la notificación se comunicará a éstos mediante notificación personal, por cédula o por carta certificada, debiendo entenderse que han aceptado el mandato si no lo rechazan dentro de segundo día.
    Una vez aceptada expresa o tácitamente, la defensa por el abogado particular es obligatoria para él y no podrá abandonarla. En caso de renuncia, deberá no obstante evacuar todos los actos de defensa mientras esté vigente el término de emplazamiento desde la notificación de su renuncia, a menos que antes se haya designado otro defensor.
    En la misma forma indicada en el inciso tercero se comunicará su designación al abogado y procurador de turno, quienes se regirán por las disposiciones del Código Orgánico de Tribunales y demás leyes que rijan la comparecencia en juicio, y estarán obligados a actuar aunque el procesado se encuentre libre, debiendo ser remunerados por él si no gozare del beneficio de pobreza.
    Se permitirá la defensa personal del procesado por el solo hecho de tener el título de abogado.
    De las diligencias de que tratan los incisos precedentes se pondrá testimonio en el proceso y se expresará el nombre del abogado y del procurador que el procesado haya escogido o que le sean designados de oficio.



NOTA 1.1
      Las modificaciones introducidas al presente Código por la Ley N° 18.857, publicada en el Diario Oficial de 6 de Diciembre de 1989, rigen, según lo dispone su artículo vigésimo, noventa días después de su publicación en el Diario Oficial.
    Artículo 278 bis.- El auto de procesamiento puede ser dejado sin efecto o modificado durante todo el sumario, de oficio o a petición de parte; pero el juez no podrá hacerlo desde que se ha concedido apelación en contra de él, ni sin nuevos antecedentes probatorios cuando haya sido revisado por la vía de ese recurso.

    Art. 279. (301) Si el procesado se encontrare en territorio extranjero, y el delito es de aquellos que autorizan la extradición con arreglo al Derecho Internacional, el juez procederá a pedirla en la forma que se determina en el párrafo 1 del Título VI del Libro III de este Código.



    Artículo 279 bis.- Podrá el juez no someter a proceso al inculpado y disponer su libertad aunque aparezcan reunidos los requisitos para procesarlo, cuando al tiempo de cumplirse el plazo de la detención judicial o al pronunciarse sobre la respectiva solicitud, hubiere adquirido la convicción de que con los antecedentes hasta entonces acumulados se encuentra establecido alguno de los motivos que dan lugar al sobreseimiento definitivo previstos en los números 4° a 7° del artículo 408, sin perjuicio de continuar las indagaciones del sumario hasta agotarlas.
    Con el mérito de nuevos antecedentes, podrá el juez durante todo el sumario dejar sin efecto el auto fundado que haya dictado y procesar al inculpado, a petición de parte o de oficio. Si no ocurriere así hasta el término del suamrio, el juez dictará sobreseimiento en favor del imputado, ordenando su consulta cuando fuere procedente.
    El que no fuere sometido a proceso en virtud de esta disposición, conservará su calidad de inculpado y podrá hacer uso de los derechos que a éste se le confieren. Antes de ser puesto en libertad, deberá designar domicilio y quedará obligado a presentarse a todos los actos del sumario para que fuere llamado.
    Además, el juez podrá decretar su arraigo en el territorio nacional mientra dure el proceso y ordenar que se presente periódicamente al tribunal.

    4. Disposiciones comunes a la detención y a la
prisión preventiva
    Art. 280. (302) Toda orden de detención o de prisión será expedida por escrito, y para llevarla a efecto, el juez o la autoridad que la dictare despachará un mandamiento firmado, en que dicha orden se encuentre transcrita literalmente.
    Art. 281. (303) El mandamiento de detención o de prisión contendrá:
    1° La designación del funcionario que lo expide;
    2° El nombre de la persona a quien se encarga su ejecución, si el encargo no se hiciere de un modo genérico a la fuerza pública representada por la policía de seguridad o por algún cuerpo de ejército, o de otro modo;
    3° El nombre y apellido de la persona que debe ser aprehendida o, en su defecto, las circunstancias que la individualicen o determinen;
    4° El motivo de la detención o prisión siempre que alguna causa grave no aconseje omitirlo;
    5° La determinación de la cárcel o lugar público de detención a donde deba conducirse al aprehendido, o de su casa cuando así se hubiere decretado.
      6° La circunstancia de si debe o no mantenérsele en incomunicación; y
    7° La firma entera del funcionario y del secretario, si lo tuviere.

    Art. 282. (304) Cuando la ejecución del mandamiento sea cometida a la fuerza pública, el jefe de ella designará al individuo o individuos que hayan de darle cumplimiento.
    Si el mandamiento se dictare para la aprehensión de malhechores que anduvieren en cuadrilla, bastará que designe determinadamente a uno o varios, para que se pueda aprehender a los demás que se encontraren en su compañía.
    Art. 283. (305) Los autos en que se decrete o deniegue la detención o prisión o la excarcelación serán apelables en el solo efecto devolutivo.
    El mandamiento de detención o prisión será ejecutorio en todo el territorio de la República.
    Art. 284. (306) El mandamiento debe intimarse, al tiempo de ejecutarlo, a la persona en quien debe cumplirse; se le exhibirá en el mismo momento de su detención y se le entregará copia de él.
    Antes de conducir a la persona detenida a la unidad policial, el funcionario público a cargo del procedimiento de detención o de aprehensión deberá informarle verbalmente la razón de su detención o aprehensión y de los derechos a que se refiere el inciso siguiente. Igual información deberá prestar al detenido o aprehendido, el encargado de la primera casa de detención policial hasta la que sea conducido, inmediatamente de ser ingresado a ella. Se dejará constancia en el libro de guardia respectivo, del hecho de haberse proporcionado la información señalada, de la forma en que se prestó la información, del nombre de los funcionarios que la proporcionaron y de aquellos ante los cuales se entregó. Sin perjuicio de lo anterior, cuando por las circunstancias que rodean la detención o aprehensión no se pueda informar al sujeto de sus derechos al momento de practicarla, la información se proporcionará inmediatamente de ser ingresado a la unidad policial o casa de detención. En los casos previstos en los incisos cuarto y quinto del artículo 260, la referida información se entregará en la casa del detenido, o en la que él señale, cuando la tuviere fuera de la ciudad. La observancia de las exigencias de este inciso no exime al funcionario de dar cumplimiento a lo establecido en el inciso anterior.
    En todo recinto de detención policial y casa de detención, en lugar claramente visible del público, deberá existir un cartel destacado en el cual se consignen los derechos de los detenidos, cuyo texto y formato serán fijados por decreto supremo del Ministerio de Justicia.
    El juez, al interrogar al detenido o preso, deberá comprobar si se dio o no cumplimiento a lo dispuesto en los dos incisos anteriores. En caso de comprobarse que ello no ocurrió, remitirá oficio con los antecedentes respectivos a la autoridad competente, para que ésta aplique las sanciones disciplinarias correspondientes y tendrá por no prestadas las declaraciones que el detenido o preso hubiere formulado ante sus aprehensores.

    Art. 285. (307) Si el juez que hubiere expedido el mandamiento sabe que la persona cuya aprehensión ordena se encuentra gravemente enferma, de tal manera que no pueda trasladársele a la cárcel sin peligro, adoptará las medidas que estime convenientes para evitar la fuga.
    Si la enfermedad no fuere conocida del juez, el encargado de cumplir la orden no la llevará a efecto hasta darle parte; pero tomará entre tanto las precauciones convenientes para impedir la fuga del que debe ser capturado.
    Artículo 286.- Siempre que se trate de aprehender a un empleado público o a un individuo de las Fuerzas Armadas o Carabineros, se dará aviso , al tiempo de expedirse el mandamiento, al jefe de la persona que se manda a aprehender.
    Si esta persona tuviere a su cargo caudales o efectos públicos, se llenarán además las formalidades prescritas por las leyes del ramo para asegurar dichos caudales y la formación de la correspondiente cuenta.

    Art. 287. (309) Todo el que aprehendiere a un presunto delincuente tomará las precauciones necesarias para impedir que haga en su persona o en su traje alteraciones que puedan dificultar su reconocimiento.
    Art. 288. (310) Cualquier resistencia para que se lleve a efecto el mandamiento de detención o de prisión expedido conforme a la ley, autoriza el empleo de la fuerza con el solo objeto de asegurar la persona que deba ser aprehendida.
    Podrá el juez, además, decretar el allanamiento de la casa en que haya sospecha fundada de que se encuentre; y se procederá entonces con arreglo a lo dispuesto en los artículos 173 y 174.

    Art. 289. (311) Si se temiere fundadamente la fuga o la resistencia de aquel a quien se trata de aprehender, se podrá, con el objeto de asegurar su persona, emplear la fuerza antes de intimar el mandamiento; pero en tal caso deberá hacerse la intimación tan pronto como cese el peligro de la fuga o resistencia.
    La fuerza pública civil o militar, estará obligada a prestar su auxilo inmediatamente que sea requerida al efecto por cualquier persona que le presente el mandamiento expedido por el juez.
    Art. 290. (312) Todo individuo aprehendido por orden de autoridad competente, será conducido en el acto a la cárcel o al lugar público de detención que el respectivo mandamiento señalare.
    Al recibirlo, el alcaide o el encargado del lugar de detención o prisión copiará en su registro la orden transcrita en dicho mandamiento o el mandamiento mismo, y hará mención de la persona que ha conducido o aprehendido al individuo.
    Art. 291. (313) El jefe de un establecimiento que recibiere a una persona en calidad de detenido o preso, dará parte del hecho al juez competente inmediatamente después del ingreso o, si no fuere hora de despacho, en la primera hora de la audiencia próxima.
    Si la aprehensión hubiere sido efectuada sin orden judicial por un particular, por un Alcalde o por la policía, el detenido será puesto a disposición del juez al tiempo de comunicarle la detención. Si ésta hubiere sido decretada por un Intendente o Gobernador, el el inculpado será puesto a disposición del juez, con todos los antecedentes relativos a la detención, en el el menor plazo posible; el cual nunca podrá exceder de cuarenta y ocho horas.
    El individuo detenido o preso por orden judicial, queda por el mismo hecho a dispoción del juez de la causa.

    Art. 292. (314) Los detenidos y los presos estarán, en cuanto sea posible, separados los unos de los otros.
    Si la separación no fuere posible, dispondrá el juez de que no se reúnan en un mismo departamento personas de diferente sexo, ni los procesados de un mismo proceso, y de que los jóvenes y los no reincidentes se hallen separados de los de edad madura y de los reincidentes.
    Para la distribución de los detenidos y presos se tendrá en cuenta el grado de educación de los mismos, su edad y la naturaleza del delito que se les imputa.



NOTA 1.1
      Las modificaciones introducidas al presente Código por la Ley N° 18.857, publicada en el Diario Oficial de 6 de Diciembre de 1989, rigen, según lo dispone su artículo vigésimo, noventa días después de su publicación en el Diario Oficial.
    Art. 293. (315) La detención, así como la prisión preventiva, debe efectuarse de modo que se moleste la persona o se dañe la reputación del procesado lo menos posible. La libertad de éste será restringida en los límites estrictamente necesarios para mantener el orden del establecimiento y para asegurar su persona e impedir las comunicaciones que puedan entorpecer la investigación.
    El detenido o preso, aunque se encuentre incomunicado, tiene derecho a que, en su presencia, a la mayor brevedad y por los medios más expeditos posibles se informe a su familia, a su abogado o a la persona que indique, del hecho y la causa de su detención o prisión. El aviso deberá darlo el encargado de la guardia del recinto policial al cual fue conducido, o el secretario del tribunal ante el cual fue puesto a disposición, si no se hubiere dado con anterioridad. Los funcionarios señalados dejarán constancia de haber dado el aviso.
    El funcionario encargado del establecimiento policial o carcelario en que se encuentre el detenido antes de ser puesto a disposición del tribunal, no podrá rehusar que éste conferencie con su abogado en presencia de aquél, hasta por treinta minutos cada día, exclusivamente sobre el trato recibido, las condiciones de su detención y sobre los derechos que puedan asistirle.
    La negativa o el retardo injustificado en el cumplimiento de lo establecido en los dos incisos precedentes serán sancionados disciplinariamente con la suspensión del cargo, en cualquiera de sus grados, por la respectiva superioridad de la institución a la cual pertenezca el funcionario infractor o por la autoridad judicial que corresponda.

    Art. 294. (316) El detenido o preso tendrá derecho para procurarse, a sus expensas, las comodidades y ocupaciones que sean compatibles con el objeto de su detención o prisión y con el régimen del establecimiento.
    Podrá, además, en el caso de no estar incomunicado por disposición del juez, recibir la visita de un ministro de su religión, de su abogado o de su procurador, o de aquellas personas con quienes esté en relación de familia, de intereses o que puedan darle consejos, observándose en este caso las prescripciones del reglamento de la casa. Si el juez lo estimare conveniente, podrá ordenar que las conferencias del detenido con dichas personas sean presenciadas por algunos de los empleados del establecimiento o del juzgado, o suspenderlas temporalmente mientras sea necesario para el éxito de la investigación.
    Art. 295. (317) El juez autorizará, en cuanto no se perjudique el éxito del sumario, los medios de correspondencia y comunicación de que pueda hacer uso el detenido o preso. Podrá ordenar que éste no reciba ni dirija cartas, telegramas ni mensajes de ninguna especie, sin que antes sean puestos en su conocimiento para ver si existe inconveniente en hacerlos llegar a su destino. En ningún caso se podrá impedir a los detenidos o presos que escriban a los funcionarios superiores del orden judicial, ni a los oficiales del Ministerio Público.
    5. De las medidas que agravan la detención o la prisión

    Art. 296. (318) No se pondrán prisiones al detenido o preso, ni se adoptará contra él ninguna otra medida extraordinaria de seguridad, sino en los casos de desobediencia, violencia o rebelión, o cuando esta medida parezca necesaria para la seguridad de los demás detenidos o para evitar el suicidio o la evasión, intentados de alguna manera.
    Art. 297. (319) Sólo el juez de la causa podrá ordenar la medida indicada en el artículo precedente, o autorizar la que otro funcionario hubiere dictado antes de poner al detenido o preso a disposición del juez.
    En casos urgentes y conforme al reglamento de la casa, podrá el alcaide o el jefe del establecimiento, o la persona que haga sus veces, disponer que se pongan prisiones al detenido o preso por alguno de los motivos expresados en el artículo anterior; pero dará parte por escrito al juez de la causa en la primera audiencia, para que se pronuncie sobre dicha medida.
    Art. 298. (320) El detenido o preso puede ser incomunicado por el juez cuando fuere indispensable para la averiguación y comprobación del delito.
    Art. 299. (321) La incomunicación podrá durar, si fuere necesario, todo el tiempo de la detención y, si ésta se convirtiere en prisión preventiva, podrá prolongarse hasta completar el término de diez días.
    INCISO SEGUNDO.- DEROGADO.-

    Art. 300. (322) El juez podrá decretar una nueva incomunicación del procesado cuando nuevos antecedentes traídos al sumario dieren mérito para ella; pero esta incomunicación no podrá exceder de cinco días.

    Art. 301. (323) El incomunicado podrá asistir, guardándose las precauciones necesarias, a las diligencias periciales en que la ley le dé intervención, siempre que su presencia no pueda desvirtuar el objeto de la incomunicación.
    Art. 302. (324) Podrá también el incomunicado tener los libros, recado de escribir y demás efectos que él se proporcione si a juicio del juez no hubiere peligro para el éxito de la investigación.
    Pero no podrá entregar ni recibir carta ni comunicación alguna sino con la venia del juez, quien se instruirá previamente de su contenido, salvo lo dispuesto en el 2° inciso del artículo 295.
    Art. 303. (325) Se permitirá que el incomunicado conferencie con su abogado en presencia del juez con el objeto de obtener medidas para hacer cesar la incomunicación. La solicitud oral o escrita en tal sentido no podrá ser denegada.

    Art. 304. (326) Ninguna incomunicación puede impedir que el funcionario encargado del establecimiento en que se halla el detenido o preso, lo visite.
    Este funcionario es obligado, siempre que el detenido o preso lo solicite, a transmitir al juez competente la copia del decreto de detención o prisión que se hubiere dado al detenido o preso, o a dar él mismo un certificado de hallarse detenido o preso aquel individuo.
    Artículo 305.- En el proceso se pondrá testimonio de toda medida con que se agrave la restricción de la libertad impuesta a un detenido o procesado, especificándose el día en que la medida hubiere comenzado a aplicarse y aquel en que hubiere sido suspendida.

  6. Del arraigo


    Art. 305 bis A. En casos graves y urgentes, el juez podrá prohibir la salida del territorio nacional al inculpado respecto de quien existan antecedentes que, apreciados en conciencia, sean bastantes para estimar que en el sumario podrá ser decretada su detención y que tratará de sustraerse a la acción de la justicia.
    Para este efecto, dictará orden de arraigo por un lapso no superior a sesenta días, el que no podrá prorrogarse en virtud del mismo hecho que motiva la orden. Transcurrido el plazo por el cual se decretó, el arraigo quedará sin efecto. El juez deberá comunicarlo de inmediato a la misma autoridad policial a quien impartió la orden sin más trámites.
    El juez, de oficio o a petición de parte, podrá poner término al arraigo durante su vigencia, si los antecedentes del proceso lo justifican.
    Si dentro del plazo fijado para el arraigo el inculpado es detenido y dejado luego en libertad por no existir méritos para someterlo a proceso, el juez deberá establecer en la misma resolución si se mantiene el arraigo o se le deja sin efecto.
    No se decretará arraigo tratándose de delitos que sólo hacen procedentes la citación, sin perjuicio que él pueda derivar de las resoluciones a que se refiere el artículo 305 bis C.
    Las resoluciones que den lugar al arraigo o lo denieguen, serán apelables en el solo efecto devolutivo, y la vista del recurso gozará de la preferencia establecida en el inciso quinto del artículo 69 del Código Orgánico de Tribunales.

    Art. 305 bis B. El arraigo podrá decretarse de oficio, a petición del Ministerio Público o del querellante particular y producirá efecto por el solo hecho de decretarse; no obstante, deberá ser comunicado personalmente al afectado por el organismo policial que el tribunal determine, sin perjuicio de su notificación judicial una vez que preste declaración indagatoria.

    Art. 305 bis C. No obstante lo dispuesto en el artículo 305 bis A, las órdenes de detención y la resolución que somete a proceso al inculpado llevan consigo el arraigo, mientras están vigentes en el proceso y aun cuando el inculpado o procesado se encuentre en libertad provisional.
    Producen también arraigo de pleno derecho las sentencias condenatorias que impongan penas privativas o restrictivas de libertad que deban cumplirse en el país mientras no se ejecuten o extingan y aun en los casos en que el condenado se encuentre en libertad condicional o cumpliendo alguna de las penas sustitutivas establecidas en la ley N° 18.216.



    Art. 305 bis D. Las personas afectadas por el arraigo, sea judicial o de pleno derecho, sólo podrán ausentarse del territorio nacional con autorización del juez que conozca o haya conocido de la causa, por el tiempo que en la misma resolución se fije, y sin que se paralice, en su caso, la marcha regular del proceso.
    El solicitante deberá rendir caución cuya naturaleza y monto fijará el juez en la misma resolución que autoriza la ausencia.
    Si el arraigado no regresa dentro del plazo señalado por el juez, se hará efectiva la caución sin más trámite, a beneficio de la Junta de Servicios Judiciales.
    El quebrantamiento del arraigo será sancionado con prisión en su grado máximo o presidio menor en su grado mínimo. Se entiende que este delito se comete en Chile, sea que se haya burlado el arraigo judicial o de pleno derecho ausentándose del territorio nacional, sea que el arraigado no haya retornado al país en el plazo debido.

    Art. 305 bis E.- La comunicación de arraigo al organismo que corresponda deberá contener todos los antecedentes que permitan individualizar correctamente al arraigado.

    Art. 305 bis F.- El querellante que a sabiendas solicite y obtenga una medida de arraigo infundada, será responsable de todos los perjuicios que con ella se causaren, con independencia de la responsabilidad criminal que pueda corresponderle con arreglo a la ley.
La acción civil para reclamar la indemnización de dichos perjuicios, deberá interponerse ante el tribunal que conoció del arraigo y se tramitará y resolverá como incidente conforme lo disponen los artículos 89 y siguientes del Código de Procedimiento Civil.

    Título V
    DEL PROCEDIMIENTO DE AMPARO


    Art. 306. (328) Todo individuo contra el cual existiere orden de arraigo, detención o prisión emanada de autoridad que no tenga facultad de disponerla, o expedida fuera de los casos previstos por la ley, o con infracción de cualquiera de las formalidades determinadas en este Código, o sin que haya mérito o antecedentes que lo justifiquen, sea que dicha orden se haya ejecutado o no, podrá, si no hubiere deducido los otros recursos legales, reclamar su inmediata libertad o que se subsanen los defectos denunciados.

    Art. 307. (329) Este recurso se deducirá ante la Corte de Apelaciones respectiva por el interesado o, en su nombre, por cualquiera persona capaz de parecer en juicio, aunque no tenga para ello mandato especial, y puede interponerse por telégrafo; y pedir el tribunal, en la misma forma, los datos e informes que considere necesarios.
    Art. 308. (330) El tribunal fallará el recurso en el término de veinticuatro horas.
    Sin embargo, si hubiere necesidad de practicar alguna investigación o esclarecimiento para establecer los antecedentes del recurso, fuera del lugar en que funcione el tribunal llamado a resolverlo, se aumentará dicho plazo a seis días, o con el término de emplazamiento que corresponda si éste excediere de seis días.
    Art. 309. (331) Podrá el tribunal comisionar a alguno de sus ministros para que, trasladándose al lugar en que se encuentra el detenido o preso, oiga a éste, y, en vista de los antecedentes que obtenga, disponga o no su libertad o subsane los defectos reclamados. El ministro dará cuenta inmediata al tribunal de las resoluciones que adoptare, acompañando los antecedentes que las hayan motivado.
    Art. 310. (332) El tribunal que conoce del recurso podrá ordenar que, dentro del plazo que fijará según la distancia, el detenido o preso sea traído a su presencia, siempre que lo creyera necesario y éste no se opusiere; o que sea puesto a disposición del ministro a quien hubiere comisionado, en el caso del artículo anterior.
    Este decreto será precisamente obedecido por todos los encargados de las cárceles o del lugar en que estuviere el detenido y la demora en darle cumplimiento o la negativa para cumplirlo sujetará al culpable a las penas determinadas por el artículo 149 del Código Penal.
    Art. 311. (333) Si el tribunal revocare la orden de detención o de prisión, o mandare subsanar sus defectos, ordenará que pasen los antecedentes al Ministerio Público y éste estará obligado a deducir querella contra el autor del abuso, dentro del plazo de diez días, y a acusarlo, a fin de hacer efectiva su responsabilidad civil y la criminal que corresponda en conformidad al artículo 148 del Código Penal.
    En uno y otro caso el funcionario culpable deberá indemnizar los perjuicios que haya ocasionado.
    El detenido o preso podrá igualmente deducir esta querella.
    Art. 312. (334) Cuando de los antecedentes apareciere que no hay motivo bastante para expedir la orden a que se refiere el artículo anterior, el tribunal lo declarará así en auto motivado. Esta declaración no exime al autor del abuso de la responsabilidad que pudiere afectarle conforme a las leyes.
    Artículo 313.- El oficial del Ministerio Público que no dedujere querella en el plazo indicado en el artículo 311, será objeto siempre de suspensión disciplinaria de su cargo hasta por treinta días, para cuyo efecto se elevarán los antecedentes en original o copia, al superior jerárquico correspondiente.

    Artículo 313 bis.- Cuando la Corte comprobare que el arresto, detención o prisión arbitraria o la irregularidad que dió lugar al recurso existió al momento de su interposición, pero que con posterioridad fue puesto en libertad el detenido o preso o se subsanaron los defectos reclamados, acogerá el amparo para los efectos de declarar la existencia de la infracción y hacer uso de sus facultades disciplinarias, o de las medidas que se indican en los artículos 311 y y 313.

    Art. 314. (336) Se considerará como prisión arbitraria y dará lugar al recurso de que trata este título, cualquiera demora en tomar su declaración al inculpado dentro del plazo que el artículo 319 establece.
    Art. 315. (337) El recurso a que se refiere este título no podrá deducirse cuando la privación de la libertad hubiere sido impuesta como pena por autoridad competente, ni contra la orden de detención o de prisión preventiva que dicha autoridad expidiere en la secuela de una causa criminal, siempre que hubiere sido confirmada por el tribunal correspondiente.
    Artículo 316.- La resolución que libre la Corte de Apelaciones en este recurso será apelable para ante la Corte Suprema, pero sólo en el efecto devolutivo cuando sea favorable al recurrente de amparo.
    La apelación deberá interponerse en el perentorio término de vienticuatro horas.

    Art. 317. (339) El que tuviere conocimiento de que una persona se encuentra detenida en un lugar que no sea de los destinados a servir de casa de detención o de prisión, estará obligado a denunciar el hecho, bajo la responsabilidad penal que pudiere afectarle, a cualquiera de los funcionarios indicados en el artículo 83, quienes deberán transmitir inmediatamente la denuncia al tribunal que juzguen competente.
    A virtud del aviso recibido o noticia adquirida de cualquier otro modo, se trasladará el juez, en el acto, al lugar en que se encuentre la persona detenida o secuestrada y la hará poner en libertad. Si se alegare algún motivo legal de detención, dispondrá que sea conducida a su presencia e investigará si efectivamente la medida de que se trata es de aquellas que en casos extraordinarios o especiales autorizan la Constitución o las leyes.
    Se levantará acta circunstanciada de todas estas diligencias en la forma ordinaria.
    Artículo 317 bis.- La negativa o demora injustificada de cualquiera autoridad en dar cumplimento a las órdenes dictadas por la Corte de Apelaciones en el conocimiento de un recurso de amparo, sujetarán al culpable a las penas determinadas en el artículo 149 del Código Penal. En todos estos casos el Ministerio Público está obligado a perseguir la responsabilidad de los infractores.

    Título VI
    DE LAS DECLARACIONES DEL INCULPADO
    Art. 318. (340) El juez que instruye el sumario tomará al sindicado del delito cuantas declaraciones considere convenientes para la averiguación de los hechos.
    Artículo 318 bis.- Es derecho del imputado libre presentarse ante el juez a declarar; en su ejercicio, nadie podrá impedirle el acceso al tribunal.
    El juez dejará constancia expresa en los autos de la presentación voluntaria, la cual no impedirá que se disponga su detención con posterioridad a la declaración.

    Artículo 319.- Todo detenido debe ser interrogado por el juez dentro de las veinticuatro horas siguientes a aquella en que hubiere sido puesto a su disposición. Lo dispuesto en el inciso precedente es sin perjuicio de lo que establece el artículo 272 bis.
    Si la detención ha tenido lugar con motivo de un delito flagrante, el juez procederá conforme lo prescribe el artículo 264.

    Art. 320. (342) La declaración del inculpado no podrá recibirse bajo juramento. El juez se limitará a exhortarlo a que diga la verdad, advirtiéndole que debe responder de una manera clara y precisa a las preguntas que le dirigiere.
    Artículo 321.- La primera declaración del inculpado o procesado comenzará con un interrogatorio de identificación, al cual deberá siempre responder. Se le preguntará su nombre, apellido paterno y materno, su apodo si lo tuviere, su edad, lugar de nacimiento y de su residencia actual, estado, profesión, oficio o modo de vivir, si ha sido procesado anteriormente, por qué delito, en qué juzgado, qué pena se le impuso, si la cumplió, si sabe leer y escribir y si conoce el motivo de su detención.  Se le interrogará también sobre los lugares donde trabaja y se dejará constancia de los números de teléfonos por medio de los cuales sea posible comunicarse con él y de los datos que arroje su cédula de identidad, la que deberá exhibir.
    Si es menor, deberá indicar el nombre de los padres o de las personas a cuyo cuidado se encuentre, y todos los datos necesarios para verificar su edad.

    Artículo 322.- Las demás preguntas que se dirijan al inculpado o procesado tendrán por objeto la averiguación de los hechos y de la participación que en ellos hubiere cabido a él u otras personas.
    Según la naturaleza y circunstancias del delito, se le preguntará también acerca de los bienes que tiene y de los ingresos que percibe; el nombre, estado y profesión de las personas con quienes vive, las labores específicas a que está dedicado y demás circunstancias personales y domésticas que puedan influir en la determinación de los móviles del delito.
    El juez informará al inculpado cual es el hecho que se le atribuye y podrá hacerle saber las pruebas que existieren en su contra, invitándole en seguida a manifestar cuanto tenga por conveniente para su descargo o aclaración de los hechos, según lo previsto en el artículo 329, y a indicar las pruebas que estime oportunas. Si las circunstancias exigieren explicaciones de su conducta que puedan establecer su inculpabilidad o culpabilidad o la de otras personas imputadas en el delito que se investiga, el juez procurará insertar literalmente las preguntas y respuestas que versaren sobre esta materia.

    Art. 323. (345) Es absolutamente prohibido no sólo el empleo de promesas, coacción o amenazas para obtener que el inculpado declare la verdad, sino también toda pregunta capciosa o sugestiva, como sería la que tienda a suponer reconocido un hecho que el inculpado no hubiere verdaderamente reconocido.
    A fin de asegurar el cumplimiento de lo establecido en la condición 2a. del artículo 481, el Juez deberá adoptar todas las medidas necesarias para cerciorarse de que el inculpado o procesado no haya sido objeto de tortura o de amenaza de ella antes de prestar su confesión,  debiendo especialmente comprobar el cumplimiento de lo dispuesto en el inciso tercero del artículo 272 bis. La negligencia grave del Juez en la debida protección del detenido será considerada como infracción a sus deberes, de conformidad con el artículo 324 del Código Orgánico de Tribunales.



    Art. 324. (346) Las relaciones que haga y las respuestas que dé el inculpado serán orales.
    Podrá, no obstante, el juez, en vista de las circunstancias del inculpado o de la naturaleza de la causa, permitirle que redacte a su presencia una contestación escrita sobre puntos difíciles de explicar, o que consulte, también a su presencia, apuntes o notas.
    Art. 325. (347) Se pondrán de manifiesto al inculpado todos los objetos que contribuyan a comprobar el cuerpo del delito a fin de que declare si los reconoce. Se le interrogará acerca de la procedencia y el destino de los objetos que reconociere y acerca de la razón de haberlos encontrado en su poder y, en general, sobre cualquiera otra circunstancia que conduzca al esclarecimiento de la verdad.
    Art. 326. (348) Cuando el juez considere conveniente el examen del inculpado en el lugar mismo en que ocurrieron los hechos sobre los cuales deba ser interrogado, o ante las personas o cosas con ellos relacionadas, procederá a practicar la diligencia en la forma dispuesta por el artículo 212.
    Artículo 327.- Si el inculpado, rehúsa contestar, o se finge loco, sordo o mudo, y el juez, en estos últimos casos, llegare a suponer como fundamento la simulación, sea por sus observaciones personales, sea por el testimonio de testigos o el dictamen de uno o más peritos, se limitará a hacer notar al inculpado que su actitud no impedirá la prosecución del proceso y que puede producir el resultado de privarle de algunos de sus medios de defensa.

    Art. 328. (350) El inculpado no podrá negarse a contestar a las preguntas del juez, fundándose en la incompetencia de este funcionario, pero se pondrá testimonio en autos de la protesta que formulare a este respecto.
    Art. 329 (351) Se permitirá al inculpado manifestar cuanto tenga por conveniente para demostrar su inocencia y para explicar los hechos, y se evacuarán con prontitud las citas que hiciere y las demás diligencias que propusiere y que sean conducentes para comprobar sus aseveraciones.
    Artículo 330.- El inculpado o procesado podrá dictar por sí mismo su declaración bajo la dirección del juez. Si no lo hiciere, la dictará éste, procurando en lo posible emplear las mismas palabras de que aquél se hubiere valido.
    El inculpado podrá, asimismo, leer la declaración una vez escrita, y el juez le advertirá que tiene este derecho. Si no usare de él la leerá en alta voz el secretario a su presencia. Si agrega o corrige alguna parte de su declaración, se consignará al final sin alterar lo escrito.
    El juez podrá ordenar que la declaración del inculpado se recoja mediante versión taquigráfica o en aparatos de estenotipia o fonograbadores.
    Los taquígrafos o estenotipistas prestarán juramento de ser veraces y de no revelar el secreto del sumario, y deberán traducir el texto inmediatamente.
    Si la versión es fonograbada, tendrá el inculpado derecho a oírla, y de ampliar o de aclarar su dichos de inmediato.  Se levantará un acta en que se transcriba la versión fonograbada bajo la vigilancia del secretario, salvo que el juez quisiere controlarla personalmente. El declarante, en ambos casos, tendrá derecho a cerciorarse del acta y firmarla.
    El texto taquigráfico se guardará en la custodia del secretario.  Lo mismo se hará con la versión fonograbada si el juez lo estima necesario; pero podrá hacerla desaparecer si se ha transcrito la declaración y el inculpado ha aceptado la transcripción.
    Igual sistema podrá adoptar el juez, si lo estima conveniente, con respecto a las declaraciones de los testigos.

    Art. 331. (353) La declaración será firmada por el juez, por todos los que hubieren intervenido en el acto, si pudieren hacerlo, y autorizada por el secretario.
    Si el inculpado se excusa de firmar, se consignará el motivo que alegare para ello; pero en ningún caso será esta negativa razón para anular la diligencia.
    Art. 332. (354) Si el inculpado no supiere la lengua castellana o si es sordo, o mudo, o sordo-mudo, se procederá a tomarle declaración en la forma preceptuada por los artículos 214 y 215.
    Artículo 333.- Si el examen del inculpado se prolongare mucho tiempo, o si se le ha hecho un número de preguntas tan considerable que llegare a perder la serenidad de juicio necesaria para contestar a lo demás que deba preguntársele, se suspenderá el examen y se le concederá el descanso prudente y necesario para recuperar la calma.
    Se hará constar en la diligencia el tiempo invertido en el interrogatorio.

    Art. 334. (356) Si en declaraciones posteriores se contradice el inculpado con lo declarado anteriormente, o retractare lo que ya había confesado, se le interrogará sobre el móvil de sus contradicciones y sobre las causas de su retractación.
    Art. 335. (357) Si son varios los inculpados, sus declaraciones serán tomadas una en pos de otra, sin permitirles que se comuniquen entre sí hasta la terminación de estas diligencias.
    Art. 336. (358) El inculpado podrá declarar cuantas veces quisiere, y el juez le recibirá inmediatamente la declaración si tuviere relación con la causa.
    Artículo 337.- DEROGADO.-


    Artículo 338.- DEROGADO.-


    Artículo 339.- DEROGADO.-


    Art. 340. (362) Si el inculpado reconociere francamente su participación en el hecho punible que se pesquisa, una vez comprobada la existencia del cuerpo del delito, podrá el juez someterlos a proceso.
    Esto no obstante, el juez continuará practicando las digilencias conducentes para adquirir el convencimiento de la verdad de la confesión y averiguar las circunstancias del delito; interrogará al procesado acerca de si hubo otros autores o cómplices, si conoce algunas personas que hubieran sido testigos o tuvieren conocimiento del hecho y, en general, sobre todo aquello que pueda aclarar o confirmar su confesión.

    Art. 341. (363) Se podrá asimismo, omitir la declaración previa del inculpado y proceder desde luego a procesarlo, cuando, al ponérsele a disposición del juez, estuvieren ya suficientemente comprobados el cuerpo del delito y la participación que en él haya cabido al inculpado.

    Título VII
    DE LA IDENTIFICACION DEL DELINCUENTE Y SUS
CIRCUNSTANCIAS PERSONALES
    Art. 342. (364) Todo aquel que acrimine a una persona determinada, deberá reconocerla judicialmente cuando el juez o las partes lo crean necesario, a fin de que no pueda dudarse cuál es la persona a quien se refiere.
    Art. 343. (365) La diligencia de reconocimiento se practicará poniendo a la vista del que hubiere de verificarlo, la persona que haya de ser reconocida, vestida, si fuere posible, con el mismo traje que llevaba en el momento en que se dice cometido el delito, y acompañada de otras seis o más personas de circunstancias exteriores semejantes.
    A presencia de todas ellas o desde un punto en que no pueda ser visto, según el juez lo estimare más conveniente, el que practicare el reconocimiento, juramentado de antemano, manifestará si se encuentra entre las personas que forman la rueda o grupo, aquella a quien se hubiere referido en sus declaraciones y, en caso afirmativo, cuál de ellas es.
    Antes del reconocimiento, el juez interrogará al testigo, preguntándole si conocía al inculpado y desde qué fecha, si lo había visto personalmente o en imágen, invitándolo a que lo describa en sus rasgos más característicos, y cuidará de que no reciba indicación alguna de que pueda deducir cuál es la persona a quién va a señalar.

    Art. 344. (366) Cuando fueren varios los que hubieren de reconocer a una persona, la diligencia de reconocimiento se practicará separadamente con cada uno de ellos, sin que puedan comunicarse entre sí hasta que se haya efectuado el último reconocimiento.
    Cuando fueren varios los que hubieren de ser reconocidos por una misma persona, podrá hacerse el reconocimiento de todos en un solo acto.
    Artículo 345.- Los alcaides de las cárceles y los jefes de los lugares de detención, tomarán las precauciones necesarias para que los presos o detenidos, no hagan en su persona o traje alteración alguna que pueda dificultar su reconocimiento; y si en los establecimientos expresados hubiere traje reglamentario, conservarán el que lleven dichos presos o detenidos al ingresar en ellos, a fin de que puedan vestirlo cuantas veces sea necesario para diligencias de reconocimiento.

    Artículo 346.- De la diligencia del reconocimiento se extenderá acta circunstanciada, que firmarán con el juez y secretario, el testigo y el inculpado o procesado si pudieren hacerlo.
     
     
   
 


     
     




NOTA 1.1
    Las modificaciones introducidas al presente Código por la Ley N° 18.857, publicada en el Diario Oficial de 6 de Diciembre de 1989, rigen, según lo dispone su artículo vigésimo, noventa días después de su publicación en el Diario Oficial.
    Art. 347. (369) Si se originare alguna duda acerca de la identidad del inculpado o procesado, el juez tratará de acreditar dicha identidad por cuantos medios fueren conducentes a ese objeto, en especial mediante un informe del Servicio de Registro Civil e Identificación.
    Hará, en consecuencia, contar con la minuciosidad posible las señas personales del inculpado o procesado, a fin de que la diligencia pueda servir oportunamente de prueba de su identidad.





NOTA 1.1
    Las modificaciones introducidas al presente Código por la Ley N° 18.857, publicada en el Diario Oficial de 6 de Diciembre de 1989, rigen, según lo dispone su artículo vigésimo, noventa días después de su publicación en el Diario Oficial.
    Artículo 347 bis.- Si el inculpado expresa ser menor de dieciocho años o esta circunstancia es conocida o presumida por otro medio, el juez mandará agregar al proceso su certificado de nacimiento, practicando, al efecto, las diligencias del caso.
    Los jueces podrán encomendar, aun telefónicamente, a otros jueces, a las autoridades de Investigaciones o de Carabineros del lugar donde haya sido inscrito el nacimiento del presunto menor, que se constituyan en las oficinas del Registro Civil para determinar la fecha del nacimiento y la comunique por la misma vía o por otra, y de la información así obtenida se dejará constancia en la causa.  Podrán también obtener esta información directamente del Oficial Civil, por las vías más rápidas, incluso por teléfono o radio.
    No encontrándose la inscripción, el juez hará lo posible por agregar, en los mismos términos que preceden, el certificado de parto o su fe de bautismo y oirá al Consejo Técnico de la casa de Menores correspondiente, o al funcionario que se haya designado en su lugar; en su defecto, pedirá dictamen a algún facultativo y recibirá las informaciones de parientes o conocidos del menor, a fin de determinar su edad, sobre el cual dictará resolución expresa.
    Si es ostensible que el inculpado es menor de dieciséis años, se lo pondrá de inmediato y provisionalmente a disposición del Juez de Menores, sin perjuicio de practicar las diligencias previstas en los incisos anteriores, y de proceder en consecuencia según fuere el resultado de las mismas.

    Artículo 347 bis A.- En ningún caso la declaración acerca de si el menor ha obrado o no con discernimiento prevista en la Ley de Menores, podrá ser demorada más de quince días. Si el Juez de Menores no ha recibido los informes técnicos correspondientes, prescindirá de ellos para formular la declaración.
    La internación del menor, cuando proceda con arreglo a la ley, será considerada privación de libertad para todos los efectos legales. El juez del crimen deberá otorgarle la excarcelación, si fuere procedente de acuerdo con las reglas generales, sin que constituya impedimento para hacerlo el hecho de no haberse efectuado o estar pendiente la declaración de discernimiento.

NOTA:  11
    El Artículo transitorio de la Ley N° 19.343, publicada en el "Diario Oficial" de 31 de Octubre de 1994, dispuso que la modificación introducida al presente artículo rige sesenta días después de su publicación en el Diario Oficial.
    Art. 348. (371) Si el juez advirtiere en el procesado indicios de enajenación mental, le someterá inmediatamente a la observación de facultativos en el establecimiento en que se hallare detenido, o en uno para enfermos mentales, si fuere más a propósito o si aquél está en libertad.
    Sin perjuicio de este reconocimiento, el juez recibirá información acerca del estado mental del procesado. En esta información serán oídas las personas que puedan deponer con acierto, en razón de sus circunstancias personales o de las relaciones que hayan tenido con el inculpado o procesado antes y después de haberse ejecutado el hecho.







NOTA 1.1
      Las modificaciones introducidas al presente Código por la Ley N° 18.857, publicada en el Diario Oficial de 6 de Diciembre de 1989, rigen, según lo dispone su artículo vigésimo, noventa días después de su publicación en el Diario Oficial.
    Artículo 349.- El inculpado o encausado será sometido a examen mental siempre que se le atribuya algún delito que la ley sanciones con presidio o reclusión mayor en grado máximo u otra superior; o cuando fuere sordomudo o mayor de setenta años, cualquiera sea la penalidad del delito que se le atribuye.

    Art. 350. (373). El juez podrá, cuando lo considere conveniente, practicar las indagaciones necesarias para apreciar el carácter y la conducta anterior del inculpado o procesado, y no podrá negarse a practicar esta investigación cuando el mismo inculpado o procesado la solicitare.



    Artículo 350 bis.- Si por la declaración indagatoria o por otro medio se supiere que el inculpado ha sido sometido a proceso en otra ocasión, se hará agregar a los autos un certificado del secretario del Juzgado que tuvo a su cargo el proceso, o del archivero judicial, en el que conste la fecha de comisión del delito, la fecha de la sentencia o del archivo judicial, en su caso, la  individualización de los procesados, la parte dispositiva del fallo y el hecho de encontrarse o no ejecutoriado y si ha sido o no cumplida. Podrá, no obstante, el tribunal ordenar expresamente que se agregue copia íntegra del fallo.
    Si el proceso anterior hubiere sido instruido en rebeldía del procesado, o si se hallare todavía pendiente se acumularán los juicios ante el juez a quien corresponda conocer de ellos, sin perjuicio de que pueda ordenarse su sustanciación por cuerda separada.





NOTA 1.1
    Las modificaciones introducidas al presente Código por la Ley N° 18.857, publicada en el Diario Oficial de 6 de Diciembre de 1989, rigen, según lo dispone su artículo vigésimo, noventa días después de su publicación en el Diario Oficial.
    Título VIII
    DEL CAREO

    Artículo 351.- Cuando los testigos o los procesados entre sí, o aquéllos con éstos, discordaren acerca de algún hecho o de alguna circunstancia que tenga interés en el sumario, podrá el juez confrontar a los discordantes a fin de que expliquen la contradicción o se pongan de acuerdo sobre la verdad de lo sucedido.
Procederá asimismo esta diligencia con respecto a los querellantes y meros inculpados.
    No será procedente el careo de las personas que no tienen obligación de prestar declaración como testigos, salvo que hubieren consentido en declarar ni lo será tampoco con respecto a aquellas que no están obligadas a concurrir.
    Tampoco procederá el careo entre inculpados o procesados y la víctima en los delitos contemplados en los artículos 361 a 367 bis del Código Penal y en el artículo 375 del mismo cuerpo legal. Si el juez lo estima indispensable para la comprobación del hecho o la identificación del delincuente, deberá emplear el procedimiento indicado en el inciso primero del artículo 355, reputándose a la víctima como testigo ausente, a menos que ella consienta expresamente en el careo.

    Art. 352 (375) Para verificar el careo, el juez hará comparecer ante él a las personas cuya declaración sea contradictoria, y juramentando o tomando promesa a los que sean testigos o querellantes y exhortando a todos a decir verdad, hará leer o leerá por sí mismo el punto en que las declaraciones se contradigan, y preguntará a cada uno de los discordantes si se ratifica en su dicho o si tiene algo que agregar o modificar a lo expuesto.
    Si alguno altera su declaración concordándola con la de otro, el juez indagará la razón que tenga para alterarla, y la que tuvo para haber declarado en los términos en que antes lo hizo.
    Si los discordantes se limitaren a ratificarse, el juez les manifestará la contradicción que existe entre sus respectivos dichos y les amonestará para que se pongan de acuerdo en la verdad, permitiendo al efecto que cada uno de los careados haga a cualquiera de los otros las preguntas que estime conducentes y las reconvenciones a que las respuestas dieren lugar, y cuidando de que no se desvíen del punto en cuestión, ni se insulten o amenacen.

    Art. 353. (376) Si fueren diversos los hechos o circunstancias acerca de los cuales ocurre la divergencia, el careo se referirá separada y sucesivamente a cada uno de ellos.
    Art. 354. (377) En el acta que se levantará para hacer constar la diligencia del careo, se pondrá testimonio con toda exactitud de las preguntas, reconvenciones y respuestas de las personas careadas, redactándolas el juez, en cuanto sea posible, con las mismas palabras con que hubieren sido expresadas.
    El careo podrá ser recogido mediante versión taquigráfica o en aparatos de estenotipia o fonograbadores y en tal caso tendrá aplicación lo dispuesto en el artículo 330.

    Art. 355. (378) Cuando apareciere contradicción entre la declaración de un testigo ausente y la del procesado o de otro testigo presente, y el juez creyere indispensable aclarar el punto en que ella ocurra, leerá al procesado o al testigo presente su declaración y las particularidades de la del ausente en que se note el desacuerdo; y las explicaciones que dé o las observaciones que haga para confirmar, variar o modificar sus anteriores asertos se consignarán en la diligencia.
    Subsistiendo la disconformidad, se librará exhorto al juez de la residencia del testigo ausente, en el cual se insertarán a la letra la declaración que haya prestado y la parte conducente de la diligencia a que se refiere el inciso anterior, a fin de que se complete esta diligencia con la de aquel testigo en la misma forma indicada en el precedente inciso.
    En casos graves, y juzgándolo el juez absolutamente necesario, ordenará la comparecencia del testigo ausente a fin de practicar el careo ante él y en la forma ordinaria. Procederá también esta diligencia con respecto a los denunciantes, querellantes y meros inculpados.

  Título IX
  DE LA LIBERTAD PROVISIONAL



    Art. 356. (379) La libertad provisional es un derecho de todo detenido o preso. Este Derecho podrá ser ejercido siempre, en la forma y condiciones previstas en este Título.
    La prisión preventiva sólo durará el tiempo necesario para el cumplimiento de sus fines. El juez, al resolver una solicitud de libertad, siempre tomará en especial consideración el tiempo que el detenido o preso haya estado sujeto a ella.
    El detenido o preso será puesto en libertad en cualquier estado de la causa en que aparezca su inocencia.
    Todos los funcionarios que intervengan en un proceso están obligados a dilatar lo menos posible la detención de los inculpados y la prisión preventiva de los procesados.


    Artículo 356 bis.- En los casos del artículo 10 números 4°, 5° y 6° del Código Penal y de los dos incisos finales del artículo 260 de este Código, la libertad provisional del detenido será resuelta inmediatamente por el juez de la causa, aun verbalmente, de oficio o a petición de parte, y con caución o sin ella, cualquiera que sea el daño causado al agresor. Esta resolución no requerirá del trámite de consulta, ni será necesario cumplir con los requisitos del artículo 361, en su caso y la apelación se concederá en el solo efecto devolutivo.

    Art. 357. (380) Una vez averiguado que el delito de que se trata está sancionado únicamente con penas pecuniarias o privativas de derechos, o con una pena privativa o restrictiva de la libertad de duración no superior a la de presidio menor en su grado mínimo, se decretará la libertad provisional del procesado, sin exigirle caución alguna.
    Pero éste deberá permanecer en el lugar del juicio hasta su terminación y presentarse a los actos del procedimiento y a la ejecución de la sentencia, inmediatamente que fuere requerido o citado conforme a los artículos 247, 249 y 250.

    Art. 358. (381) Si el delito imputado no mereciere pena aflictiva, se otorgará la libertad provisional sin necesidad de caución:
    1° Al procesado en cuyo favor se pronunciare en primera instancia sentencia de absolución o auto de sobreseimiento, aun cuando la sentencia o auto hayan de ser revisados por tribunal superior; y
    2° Al procesado condenado en primera instancia a una pena cuyo tiempo se hubiere completado durante la detención y la prisión preventiva.




    Art. 359. (382) Se suspenderá el decreto de detención o de prisión preventiva contra una persona sindicada de delito a que la ley no señale pena aflictiva, siempre que ella afiance suficientemente su comparecencia al juicio y a la ejecución de la sentencia que se pronuncie. Y si esa persona da previamente fianza, no se librarán aquellos decretos.
    En consecuencia, y sin perjuicio de lo dispuesto en los artículos 247 y 357, se concederá, de oficio o a petición de parte, bajo fianza suficiente, la libertad provisional:
    1° A los autores de delito a que la ley impone una pena menor que las de presidio, reclusión,  confinamiento, extrañamiento y relegación menores en su grado máximo;
    2° A los cómplices o a los encubridores de delitos a que la ley señale una pena mayor que las del número precedente, cuando según la ley haya de reducirse la pena a una menor que las designadas en dicho número;
    3° A los procesados de delito frustrado o de tentativa que se hallen en el caso del número 1°; y
    4° A los procesados como autores o cómplices o encubridores de cualquier delito, siempre que, por las circunstancias atenuantes que concurran o por las que resten una vez compensadas ellas con las agravantes del caso, la pena sea menor que las expresadas en el mismo número 1°.




    Art. 360. (383) Los procesados por delito que merezca pena aflictiva que sean absueltos, o respecto de los cuales se dicte auto de sobreseimiento en primera instancia, serán puestos en libertad, bajo fianza, mientras la causa fuere revisada por el tribunal superior.
    También se concederá la excarcelación bajo fianza a los procesados que han cumplido la pena que les aplica la sentencia de primera instancia.

    Art. 361. (384) Si el delito tiene asignada por ley pena aflictiva, el detenido o preso tendrá derecho a que se le conceda la excarcelación, salvo en los casos a que se refiere el artículo 363.
    En este caso, la resolución que otorgue la libertad provisional será fundada, sobre la base de los antecedentes de hecho y de derecho que existan en el proceso, y deberá consultarse al tribunal de alzada que corresponda. Dicho tribunal resolverá la respectiva consulta, o apelación en su caso, por resolución también fundada.
    Para los efectos de este artículo no se aceptará otra caución que hipoteca o depósito de dinero o de efectos públicos de un valor equivalente. Estas cauciones podrán ser constituidas también por terceros.

    Art. 362. (385) Al acordar la libertad provisional en cualquiera de sus formas, podrá el juez, cuando las circunstancias lo exijan, disponer que el inculpado o procesado se presente a la secretaría, en los días que le determine, bajo apercibimiento de dejar sin efecto la libertad provisional, y del pago de la caución.




NOTA 1.1
      Las modificaciones introducidas al presente Código por la Ley N° 18.857, publicada en el Diario Oficial de 6 de Diciembre de 1989, rigen, según lo dispone su artículo vigésimo, noventa días después de su publicación en el Diario Oficial.
    Art. 363. (386) Sólo podrá denegarse la libertad provisional, por resolución fundada, basada en antecedentes calificados del proceso, cuando la detención o prisión sea estimada por el Juez como necesaria para el éxito de las investigaciones del sumario, o cuando la libertad del detenido o preso sea peligrosa para la seguridad de la sociedad o del ofendido.
    Se entenderá que la detención o prisión preventiva es necesaria para el éxito de las investigaciones, sólo cuando el juez considerare que existe sospecha grave y fundada de que el imputado pudiere obstaculizar la investigación, mediante conductas tales como la destrucción, modificación, ocultación o falsificación de elementos de prueba; o cuando pudiere inducir a coimputados, testigos, peritos o terceros para que informen falsamente o se comporten de manera desleal o reticente.
    Para estimar si la libertad del imputado resulta o no peligrosa para la seguridad de la sociedad, el juez deberá considerar especialmente alguna de las siguientes circunstancias: la gravedad de la pena asignada al delito; el número de delitos que se le imputare y el carácter de los mismos; la existencia de procesos pendientes; el hecho de encontrarse sujeto a alguna medida cautelar personal, en libertad condicional o cumpliendo alguna de las penas sustitutivas contempladas en la ley N° 18.216; la existencia de condenas anteriores cuyo cumplimiento se encontrare pendiente, atendiendo a la gravedad de los delitos de que trataren, y el haber actuado en grupo o pandilla.
    Se entenderá que la seguridad de la víctima del delito se encuentra en peligro por la libertad del detenido o preso cuando existan antecedentes calificados que permitan presumir que éste pueda realizar atentados en contra de ella o de su grupo familiar. Para la aplicación de esta norma, bastará que esos antecedentes le consten al juez por cualquier medio.
    El tribunal deberá dejar constancia en el proceso, en forma pormenorizada, de los antecedentes calificados que hayan obstado a la libertad provisional, cuando no pueda mencionarlos en la resolución, por afectar el éxito de la investigación.
    Para conceder la libertad provisional en los casos a que se refiere este artículo, el tribunal deberá requerir los antecedentes del detenido o preso al Servicio de Registro Civil e Identificación por el medio escrito u oral que estime más conveniente y expedito. El Servicio de Registro Civil e Identificación estará obligado a proporcionar de inmediato la información pertinente, usando el medio más expedito y rápido para ello, sin perjuicio de remitir con posterioridad los antecedentes correspondientes.
    Sólo estarán autorizados a solicitar oralmente la información mencionada el juez o el secretario letrado del tribunal, dejándose testimonio en el proceso de la fecha y forma en que se requirió el informe respectivo y, si la respuesta es oral, señalará además su fecha de recepción, la individualización de la persona que la emitió y su tenor.
    Lo dispuesto en los dos incisos precedentes se entenderá sin perjuicio de las actuaciones que se deban efectuar para prontuariar al procesado.


    Art. 364. (387) La libertad provisional se puede pedir y otorgar en cualquier estado del juicio.
    INCISO SEGUNDO.- DEROGADO.-

    Cuando el juez de la causa oficie a otro juez para la aprehensión de una persona, expresará en el oficio, si puede concederse o no la libertad provisional, con caución o sin ella; y el juez exhortado la otorgará o no en conformidad a esa expresión.
    Si otorga la libertad, exigirá a la persona fianza de presentarse al juez de la causa en un plazo breve que fijará.

    Art. 365. (388) La solicitud sobre libertad provisional será resuelta, a más tardar, veinticuatro horas después de presentada.
    INCISO SEGUNDO.- DEROGADO.-


    Art. 366. (389) El auto que decrete o deniegue la libertad provisional y el que fije la cuantía de la caución, si hubiere lugar a ella, serán reformables de oficio o a instancia de parte durante todo el curso de la causa.
    Pedida la reconsideración, el juez podrá desecharla de plano.
    Si el procesado quiere apelar de alguno de los autos expresados en el inciso 1°, deberá deducir el recurso en el acto de la notificación y le será concedido solamente en el efecto devolutivo. El ministro de fe que practique la notificación interrogará al procesado sobre si se conforma o apela y de su respuesta pondrá testimonio en la diligencia.

    Art. 367. (390) La caución tiene por objeto asegurar la presentación del inculpado o procesado cuando el juez, estimando necesaria su comparecencia personal, lo citare, o cuando se tratare de llevar a efecto la ejecución de la sentencia.


NOTA 1.1
      Las modificaciones introducidas al presente Código por la Ley N° 18.857, publicada en el Diario Oficial de 6 de Diciembre de 1989, rigen, según lo dispone su artículo vigésimo, noventa días después de su publicación en el Diario Oficial.
    Art. 368. (391) La cuantía de la caución será determinada por el juez, tomando en consideración la naturaleza del delito, el estado social y antecedentes del procesado y las demás circunstancias que pudieran influir en el mayor o menor interés de éste para ponerse fuera del alcance de la justicia.

    Art. 369. (392) La fianza podrá constituirse por escritura pública o por un acta firmada ante el juez por el procesado y el fiador.
    El fiador deberá ser vecino del lugar, tener la solvencia determinada por el artículo 2350 del Código Civil, y no encontrarse comprendido entre las personas a quienes prohíbe obligarse como fiador el artículo 2342 del mismo Código.
    Art. 370. (393) Una misma persona no podrá estar obligada por más de dos fianzas a la vez, a menos que se trate de procesados de un mismo proceso.


    Art. 371. (394) Podrá substituirse la fianza por un depósito de dinero, prenda de efectos públicos o hipoteca suficiente.
    Art. 372. (395) El procesado y el fiador deberán designar casa para el efecto de las notificaciones y citaciones que ocurrierén y que sea menester hacerles personalmente, aun cuando hayan constituido apoderado.
    Las notificaciones que se hagan al procesado o a su procurador, deberán ser hechas también al fiador cuando se relacionen con la obligación de éste.
    El procesado y su fiador darán aviso de todo cambio de morada al secretario de la causa, quien dejará de ello testimonio en el proceso.
    Artículo 372 bis.- Al encarcelado sin caución que fuere notificado de una citación personalmente en el domicilio señalado, o por cédula si no se le encontrare en él, y no compareciere en el término fijado por el juez, se le dejará sin efecto la libertad provisional.
Si comparece o es aprehendido después, para ser excarcelado por segunda vez podrá exigírsele que rinda caución.

    Art. 373. (396) Si el procesado mandado citar por el juez y notificado personalmente en el domicilio que tuviere señalado, o por cédula si no se le encontrare allí, no compareciere en el término que se le fijare, mandará el juez que se notifique a su fiador para que lo presente en el plazo de cinco días.
    Este plazo podrá ampliarse a solicitud del fiador siempre que hubiere motivo fundado.
    Se notificará al fiador en el domicilio que hubiere señalado, o por cédula si no fuere encontrado en él. No compareciendo el procesado en el término de los cinco días o en el que se fijare, se procederá a hacer efectiva la fianza, para lo cual dictará el juez el auto respectivo, que quedará ejecutoriado sin necesidad de más trámite.
    Art. 374. (397) Si el procesado hubiere constituido prenda o hipoteca para su libertad provisional y no compareciere a la primera citación, se le hará una segunda para que se le presente al juzgado dentro del quinto día. Si no compareciere en este plazo, el juez ordenará vender la prenda por un corredor de comercio o embargar la finca hipotecada. Sin perjuicio, el juez dictará las órdenes convenientes para la aprehensión del procesado.


    Art. 375. (398) El procedimiento ejecutivo se seguirá de oficio y sin dilación alguna, en cuaderno separado y conforme a las reglas del Título siguiente, hasta enterar en alguna institución bancaria a la orden de la Junta de Servicios Judiciales, o en dinero en la secretaría del juzgado, la suma a que ascienda la cuantía de la fianza, depósito, hipoteca o prenda mandados constituir.

    Artículo 375 bis.- Si la caución ha consistido en un depósito de dinero, y no comparece el procesado en el tiempo que se le fijare por el juez, éste hará efectiva la caución girando la cantidad consignada a la orden de la Junta de Servicios Judiciales.

    Art. 376. (399) Si el procesado compareciere o fuere aprehendido dentro del mes siguiente a la fecha en que se deposite en la Junta de Servicios Judiciales la suma a que asciende la cuantía de la fianza, depósito, prenda o hipoteca constituidos, se adjudicará a la expresada institución la cuarta parte de dicha suma y se devolverá al fiador o al procesado, según corresponda, la cantidad restante.
    De otro modo, será adjudicada a la Junta la totalidad de dicha suma.
    En caso de imposibilidad para comparecer, debidamente justificada durante el incidente de adjudicación, será devuelta toda la suma al fiador o al procesado, según corresponda.



    Art. 377. (400) Podrá el juez poner término a la libertad provisional por resolución fundada, cuando aparezcan nuevos antecedentes que así lo justifiquen, al tenor de lo dispuesto en los incisos primero y segundo del artículo 363, y procediendo en lo demás en conformidad a lo establecido en el inciso tercero del mismo artículo.

    Artículo 378.-  Terminará la responsabilidad del tercero que ha constituido la caución y ésta quedará cancelada:
    1° Cuando lo pidiere, presentando al procesado;
    2° Cuando éste fuere reducido a prisión;
    3° Cuando denunciare que el procesado intenta fugarse, siempre que la denuncia se hiciere con la oportunidad necesaria para que pueda llevarse a efecto la aprehensión;
    4° Cuando recayere en el juicio sentencia firme de sobreseimiento o de absolución, o cuando, siendo ésta condenatoria, se presentare el procesado a cumplir condena, y
    5° Cuando falleciere el procesado estando pendiente la causa.



    Artículo 379.-  Hecha efectiva en todo o parte la caución, no tendrá acción el que la hubiere constituido para pedir la devolución a título de pago indebido, pero le quedará a salvo su derecho para reclamar la indemnización que corresponda, del procesado o de sus causas-habientes, en conformidad a las reglas legales.

  Título X
  DEL EMBARGO Y DE LAS DEMAS MEDIDAS PARA ASEGURAR LA
RESPONSABILIDAD PECUNIARIA DEL PROCESADO Y DE LOS TERCEROS
CIVILMENTE RESPONSABLES




  Artículo 380.-  En la resolución que someta a proceso al inculpado, el juez ordenará de oficio que, si tiene bienes, se le embarguen los que sean suficientes para cubrir las costas y gastos que pueda ocasionar el juicio al Estado y el máximo de la multa señalada por la ley al delito, fijando el monto hasta el cual deba calcularse el embargo.
    Para fijar esa cantidad, el juez no tomará en cuenta las responsabilidades civiles provenientes del delito, sino cuando ellas cedan en favor del Fisco.
    Podrá también considerarlas a petición fundada de parte.
    Cuando el delito por el cual se ordene procesar al procesado sea violación, rapto, homicidio o lesiones, el juez podrá también decretar de oficio el embargo de los bienes del procesado, para asegurar todas las responsabilidades pecuniarias que se puedan pronunciar contra él, si estima que de otra manera la víctima o sus herederos no podrán hacer efectivos sus derechos.
    En cualquier estado del proceso, el querellante o el actor civil podrán pedir el embargo de bienes del procesado o del tercero civilmente responsable para el aseguramiento de todas las responsabilidades civiles provenientes de cualquier delito, y el juez lo decretará de acuerdo con los antecedentes que se hayan producido, determinando el monto hasta el cual ha de recaer el embargo.
    La circunstancia de no encontrarse ejecutoriado el auto de procesamiento no obstará para que el embargo se decrete y se lleve a efecto.




NOTA 1.1
      Las modificaciones introducidas al presente Código por la Ley N° 18.857, publicada en el Diario Oficial de 6 de Diciembre de 1989, rigen, según lo dispone su artículo vigésimo, noventa días después de su publicación en el Diario Oficial.
    Artículo 381.-  En casos graves y urgentes, o cuando sea de temer que el inculpado o el responsable civil oculten sus bienes o se desprendan de ellos, o si la persona a la cual deba afectar no es de conocida solvencia, el embargo podrá ordenarse de oficio o a petición de parte desde que aparezcan contra el inculpado fundadas sospechas de su participación en un hecho que presente caracteres de delito.

    Artículo 382.-  Por el embargo quedan afectados bienes del procesado o del inculpado, o de terceros civiles responsables, para asegurar las responsabilidades pecuniarias que contra ellos puedan declararse. En tanto estas responsabilidades no se pronuncien por sentencia firme, el embargo tendrá carácter cautelar, pero ejecutoriada la sentencia los bienes embargados serán realizados para la satisfacción de aquéllas.

    Artículo 383.-  Tan pronto ordene el embargo, el juez despachará un mandamiento que contendrá:
    1° La orden de embargar bienes de una o más personas, a quienes se individualizará, por las cantidades que se indicarán;
    2° Los bienes que deberán embargarse, si el juez estima convenientes señalarlos determinadamente;
    3° La designación de un depositario provisional, que podrá ser el propio inculpado o procesado, y
    4° La orden de prestar el auxilio de la fuerza pública al ministro de fe o al depositario, en caso de que la soliciten.
    Con el mandamiento despachado por el juez, el ministro de fe procederá al embargo de los bienes determinados en él y en seguida notificará al afectado, personalmente si es habido, por cédula si, no siéndolo, se conoce su domicilio o morada, y si ésta es desconocida, mediante un aviso que se insertará en el estado diario.
    El mandamiento de embargo y las demás diligencias a que se refiere este Título deberán ser practicados en cuaderno separado por el receptor, o por el funcionario del tribunal o de policía que el juez designe como ministro de fe para estos efectos en el proceso.

    Artículo 384.-  El mandamiento de embargo decretado contra los bienes de la mujer casada, no divorciada ni separada de bienes, se trabará en sus bienes propios, en los de la sociedad conyugal o en los de ambos.

    Art. 385. (408) Si los bienes embargados consistieren en dinero efectivo, efectos públicos, créditos realizables en el acto, alhajas de oro, plata o pedrería, se depositarán en un banco o en poder de la persona que el juez designe y quedarán dichos bienes a disposición de éste.
    Art. 386. (409) Si el embargo se trabare en otros bienes muebles, no semovientes, o en frutos y rentas embargables, el ministro de fe encargado del embargo los entregará bajo inventario al depositario.
    El depositario firmará la diligencia de recibo, obligándose a conservar los bienes a disposición del juez que conozca de la causa y, en caso de pérdida, a pagar la cantidad a que ascendiere el valor de lo depositado, sin perjuicio de la responsabilidad criminal en que pudiere incurrir.
    El depositario podrá recoger y conservar en su poder los bienes embargados, o dejarlos bajo su responsabilidad en poder del procesado.
    El juez determinará, bajo su responsabilidad, si el depositario ha de afianzar el buen cumplimiento del cargo, y el importe de la fianza, en su caso.

    Art. 387. (410) Si se embargaren sementeras, plantíos o, en general, frutos pendientes o algún establecimiento industrial o mercantil, podrá el juez decretar, cuando, atendidas las cirunstancias, lo creyere conveniente, que continúe administrándolos el procesado por sí o por medio de la persona que designe.
    Si el procesado conservare la administración, el juez le nombrará un interventor que lleve cuenta y razón de los frutos que se perciban y consuman. Si el juez determinare nombrar un administrador, éste afianzará el buen desempeño de su cargo, y el procesado podrá nombrar un interventor.
    Art. 388. (411) En los casos de los dos artículos anteriores, cesará el embargo tan pronto como los frutos percibidos alcancen a una suma equivalente a la cantidad fijada por el juez en conformidad al artículo 380.
    Art. 389. (412) El embargo de un inmueble no comprende el de sus frutos o rentas; salvo el caso de que, no siendo suficiente el valor del inmueble, el juez determine expresamente que se extienda a todos o a una parte de ellos.
    La misma regla se aplicará al embargo de vehículos de la locomoción colectiva o taxis, cuando no se dispusiere su retiro de la circulación.
    El embargo será inscrito sin dilación en el Registro Conservatorio de Bienes Raíces o de Vehículos Motorizados, según corresponda, y no podrá exigirse pago de derechos por esta diligencia, sino cuando el procesado fuere condenado.



NOTA 1.1
      Las modificaciones introducidas al presente Código por la Ley N° 18.857, publicada en el Diario Oficial de 6 de Diciembre de 1989, rigen, según lo dispone su artículo vigésimo, noventa días después de su publicación en el Diario Oficial.
    Art. 390. (413) DEROGADO.-


    Art. 391. (414) El depositario cuidará de que los semovientes den los productos propios de su clase con arreglo a las circunstancias y procurará su conservación y aumento.
    Si creyere conveniente la enajenación de todos o de algunos de ellos, pedirá al juez la correspondiente autorización.
    El juez autorizará la enajenación siempre que el inculpado, procesado o tercero civilmente responsable, según corresponda, convenga en ello. Lo decretará contra la voluntad de éstos cuando no hubiere depositario que acepte el cargo y aun sin previa petición del depositario, cuando los gastos de administración y conservación excedieran de los productos, a menos que el pago de dichos gastos se aseguren suficientemente por el inculpado, procesado, tercero civilmente responsable o por otra persona.
    Igualmente, el juez decretará la enajenación de los bienes muebles sujetos a corrupción o susceptibles de próximo deterioro, y de los muebles y semovientes cuya conservación sea difícil o dispendiosa.



NOTA 1.1
      Las modificaciones introducidas al presente Código por la Ley N° 18.857, publicada en el Diario Oficial de 6 de Diciembre de 1989, rigen, según lo dispone su artículo vigésimo, noventa días después de su publicación en el Diario Oficial.
    Art. 392. (415) Durante el juicio podrá el tribunal que actualmente conociere de él, ampliar o reducir el embargo y demás medidas, según los motivos que sobrevinieren para estimar que han aumentado o disminuido las responsabilidades pecuniarias del procesado.

    Art. 393. (416) El juez podrá también, para los fines de que trata este Título y de oficio o a petición de parte, decretar en lugar del embargo o junto con él cualesquiera de las medidas precautorias previstas en el Título V del Libro II del Código de Procedimiento Civil en la forma allí regulada.

    Art. 394. (417) Se omitirán o alzarán el embargo o la prohibición de enajenar o gravar, siempre que el procesado caucione confianza o hipoteca suficiente las resposabilidades pecuniarias que pudieran imponérsele en definitiva.
    Art. 395. (418) Asimismo se omitirá el embargo siempre que no hubiere bienes suficientes y conocidos en que hacerlo efectivo.
    Art. 396. (419) En cualquier estado del juicio en que fuere reconocida la inocencia del procesado, se procederá a alzar inmediatamente el embargo trabado en sus bienes, o a cancelar las fianzas o levantar la prohibición de enajenar, que le hubieren sido impuestas.
    El Conservador no podrá exigir pago de derechos por estas diligencias.

    Art. 397. (420) Las tramitaciones a que dieren origen las diligencias prescritas en este Título, se instruirán en cuaderno separado, y la medidas que el juez adoptare serán apelables sólo en el efecto devolutivo, salvo que se refieran a la realización de los bienes embargados, en cuyo caso la apelación se concederá en ambos efectos. En lo demás, el proceso sigue su curso legal.

    Art. 398. (421) Cuando la reponsabilidad civil recaiga sobre terceras personas, el embargo y las medidas cautelares se trabará sobre bienes de éstas, y se procederá en todo de conformidad con las disposiciones de este Título.
    Las terceras personas que aparecieren como civilmente responsables, tendrán derecho para intervenir en todo lo relativo a las diligencias ordenadas en este título, y podrán sostener su irresponsabilidad y comprobarla por los medios que determina la ley.
    Esta intervención no suspenderá en ningún caso la substanciación del juicio criminal; y el juicio a que diere lugar se tramitará en la forma de un incidente.

    Art. 399. (422) Declaradas por sentencia firme las responsabilidades civiles que deban satisfacer los procesados o los civilmente responsables, se procederá a la realización de los bienes embargados o afectos a medidas precautorias, en cuanto sea procedente, de conformidad con las reglas generales.
    En todo lo que no estuviere previsto en este título, se aplicarán las reglas que el Código de Procedimiento Civil establece sobre embargo, administración y procedimiento de apremio y sobre medidas precautorias, en su caso.

    Art. 400. (423) En los casos de quiebra del procesado o del tercero civilmente responsable, el representante del Fisco, el querellante particular y el actor civil, en su caso, figurarán como acreedores por las cantidades que haya fijado el juez que conoce del proceso, con arreglo a los artículos 380 y 381, y con la prelación que les corresponda, según las reglas generales.

    Título XI
    DE LA CONCLUSION DEL SUMARIO

    Art. 401. (432) Practicadas las diligencias que se hayan considerado necesarias para la averiguación del hecho punible y sus autores, cómplices y encubridores, el juez declarará cerrado el sumario. Las partes tendrán el plazo común de cinco días para pedir que se deje sin efecto esta resolución y se practiquen las diligencias que se consideren omitidas, las que deberán mencionar concretamente.
    El término se considerará ampliado, cuando el sumario constare de más de cien fojas, con un día más por cada veinticinco fojas que excedan del número indicado, pero en ningún caso podrá ser mayor de quince días.
    Vencido el término, el juez resolverá de plano todas las solicitudes conjuntamente y ordenará practicar las diligencias que estime necesarias, reabriendo el sumario en tal caso. Es inapelable esta resolución en cuanto ordena diligencias.
    Una vez cumplidas las actuaciones ordenadas cerrará nuevamente el sumario.

    Art. 402. (433) DEROGADO.-


    Art. 403. No podrá elevarse a plenario un proceso por crimen o simple delito sino en contra de las personas que están sometidas a proceso.

    Art. 404. (434) DEROGADO.-


    Art. 405. (435) Si durante el sumario, el procesado dedujere algunas de las excepciones de previo y especial pronunciamiento enunciadas en el artículo 433, se la tramitará en cuaderno separado y no se suspenderá la investigación, ni aun por apelación pendiente.
    Título XII
    DEL SOBRESEIMIENTO
    Art. 406. (436) Por el sobreseimiento se termina o se suspende el procedimiento judicial en lo criminal.
    El sobreseimiento es definitivo o temporal, total o parcial.
    Art. 407. (437) Puede decretarse auto de sobreseimiento en cualquier estado del juicio, haya o no querellante particular, y puede pedirse por cualquiera de las partes o por el Ministerio Público, y decretarse de oficio por el juez.
    Art. 408. (438) El sobreseimiento definitivo se decretará:
    1° Cuando, en el sumario, no aparezcan presunciones de que se haya verificado el hecho que dio motivo a formar la causa;
    2° Cuando el hecho investigado no sea constitutivo de delito;
    3° Cuando aparezca claramente establecida la inocencia del procesado;
    4° Cuando el procesado esté exento de responsabilidad en conformidad al artículo 10 del Código Penal o en virtud de otra disposición legal;
    5° Cuando se haya extinguido la responsabilidad penal del procesado por alguno de los motivos establecidos en los números 1°, 3°, 5° y 6° del artículo 93 del mismo Código;
    6° Cuando sobrevenga un hecho que, con arreglo a la ley, ponga fin a dicha responsabilidad; y
    7° Cuando el hecho punible de que se trata haya sido ya materia de un proceso en que haya recaído sentencia firme que afecte al actual procesado.


    Art. 409. (439) Se dará lugar al sobreseimiento temporal:
    1° Cuando no resulte completamente justificada la perpetración del delito que hubiere dado motivo a la formación del sumario;
    2° Cuando, resultando del sumario haberse cometido el delito, no hubiere indicios suficientes para acusar a determinada persona como autor, cómplice o encubridor;
    3° Cuando el procesado caiga en demencia o locura, y mientras ésta dure;
    4° Cuando para el juzgamiento criminal se requiera la resolución previa de una cuestión civil de que deba conocer otro tribunal; y entonces se observará lo prevenido en los artículos 4° de este Código y 173 del Código Orgánico de Tribunales; y
    5° Cuando el procesado ausente no comparezca al juicio y haya sido declarado rebelde, siempre que haya mérito bastante para formular acusación en su contra, y sin perjuicio de lo prevenido en el artículo 604.


    Art. 410. (440) El sobreseimiento es total cuando se refiere a todos los delitos y a todos los procesados; y parcial cuando se refiere a algún delito o a algún procesado, de los varios a que se hubiere extendido la averiguación.
    Si el sobreseimiento es parcial, se continuará el juicio respecto de aquellos delitos o de aquellos procesados a que no se hubiere extendido aquél.
    Art. 411. (441) DEROGADO.-


    Art. 412. (442) DEROGADO.-


    Art. 413 (443) El sobreseimiento definitivo no podrá decretarse sino cuando esté agotada la investigación con que se haya tratado de comprobar el cuerpo del delito y de determinar la persona del delincuente.
    Si en el sumario no estuvieren plenamente probadas las circunstancias que eximen de responsabilidad o los hechos de que dependa la extinción de ella, no se decretará el sobreseimiento sino que se esperará la sentencia definitiva.
    Art. 414. (444) El auto de sobreseimiento definitivo deberá consultarse cuando el juicio versare sobre delito que la ley castiga con pena aflictiva.
    Deberá también consultarse siempre que hubiere sido dictado contra la opinión del Ministerio Público.
    Si el sobreseimiento definitivo fuere parcial, no se llevará a efecto la consulta sino cuando se eleven los autos por alguna apelación o en consulta de la sentencia definitiva. Pero si hubiere procesado sometido a prisión preventiva no procesado por otro delito, respecto de quien se hubiere mandado sobreseer, se hará inmediatamente la consulta y se elevará copia de los antecedentes que se refieran a ese procesado.


    Art. 415. (445) La Corte de Apelaciones, una vez elevados los autos en apelación o en consulta de la sentencia en que se manda sobreseer o seguir adelante el juicio, oirá la opinión de su fiscal y, sin más trámite, pondrá la causa en tabla para pronunciarse acerca de las conclusiones que éste formule.
    En la vista de la causa, las partes podrán exponer verbalmente lo que convenga a su derecho.
    INCISO DEROGADO
    Art. 416. (446) Si el fiscal se conformare con el sobreseimiento, propondrá la aprobación del auto consultado; pero, si creyere que el sumario arroja mérito para continuar la causa, pedirá que se la siga adelante, elevándola a plenario.
    Podrá pedir también que se la reponga al estado de sumario, cuando creyere que deban evacuarse algunas diligencias además de las que han sido practicadas, e indicará con precisión cuáles deban ser esas diligencias.
    Art. 417. (447) Si el tribunal advirtiere que la causa se ha seguido, ante juez incompetente, devolverá los autos al juez competente para que se pronuncie acerca del sobreseimiento o adelante la investigación, si lo creyere necesario; pero no por eso dejarán de ser válidas las demás diligencias practicadas.
    Art. 418. (448) El sobreseimiento total y definitivo pone término al juicio y tiene la autoridad de cosa juzgada.
    La misma autoridad tiene el parcial definitivo respecto de aquellos a quienes afecta.
    El temporal suspende el procedimiento hasta que se presenten mejores datos de investigación o cese el inconveniente legal que haya detenido la prosecución del juicio.
    Art. 419. (449) Terminado el proceso por auto firme de sobreseimiento definitivo, se pondrá en libertad a los procesados que no estén presos por otra causa, y se entregarán a quien pertenezcan los libros, papeles y correspondencia que se haya recogido, y las piezas de convicción que tuvieren dueño conocido.
    Si existieren piezas de convicción de algún valor que no tengan dueño conocido, el juez de la causa procederá como si se tratase de una especie mueble al parecer perdida, y dará cumplimiento a lo dispuesto en los artículos 629 y 630 del Código Civil.
    Art. 420. (450) Si se pronunciare auto firme de sobreseimiento temporal, el juez mandará poner en libertad a los procesados que no estén presos por otra causa, y hará archivar, junto con los autos, los libros, papeles, correspondencia y piezas de convicción que hubiere recogido, si creyere necesario conservarlos para evitar que se frustre la investigación que pueda intentarse más adelante.
    En caso de no estimar necesaria su conservación, serán devueltos o realizados, en la forma indicada en el artículo precedente.
    Art. 421. (451) Si el sobreseimiento definitivo o temporal afectare a un procesado loco o demente, éste será puesto en libertad; pero si se le ha imputado un hecho que la ley califique de crimen, se adoptarán las medidas de precaución indicadas en el número 1° del artículo 10 del Código Penal.
    El sobreseimiento por amnistía del procesado no obsta a la continuación, en el mismo juicio criminal, de la acción civil ya entablada.




    Art. 422. (452) Cuando el sumario manifieste que el hecho punible consiste en una mera falta, el juez procederá en la forma establecida en el Título I del Libro III de este Código; y le servirán de base las diligencias practicadas.
    Art. 423. (453) Cuando en el curso del proceso a que se refiere el artículo anterior aparecieren hechos que importen un crimen o un simple delito, se le tramitará en conformidad a las disposiciones del presente libro.
  Segunda Parte
  DEL PLENARIO
  Título I
  DE LA ACUSACION

    Art. 424.(455) Cuando, ejecutoriada la resolución que declara cerrado el sumario, el juez no encontrare mérito para decretar el sobreseimiento, dictará un auto motivado en el cual dejará testimonio de los hechos que constituyen el delito o los delitos que resultan haberse cometido y la participación que ha cabido en él, o en cada uno de ellos, al procesado o a los procesados de la causa, con expresión de los medios de prueba que obran en el sumario para acreditar unos y otras. Este auto será la acusación de oficio y deberá dictarse en el plazo de quince días, contado desde la ejecutoria aludida al comienzo de este artículo.



    Art. 425. (456) Si en el sumario hubieren obrado querellantes o actores civiles, que no se hubieren desistido, el juez les dará traslado de la acusación por el término fatal y común de diez días, que se aumentará en un día por cada doscientas fojas de que consten los autos, no pudiendo exceder de veinte días. Dentro de este plazo, el querellante podrá adherir a la acusación de oficio o presentar otra por su parte y deducir las acciones civiles que le correspondan. El actor civil podrá interponer formalmente las suyas, en igual término.
    Se entenderá abandonada la acción por el querellante que no hubiere presentado su adhesión o su acusación dentro de plazo.

    Art. 426. (457) Los autos y los libros y piezas de convicción podrán ser examinados en la secretaría del tribunal, a menos que el juez, por motivo calificado, permita que sean llevados a otro lugar por un procurador, con las debidas garantías, por un plazo determinado. Vencido este plazo, podrá ser apremiado con arresto el procurador que ho hubiere devuelto al secretario del tribunal los autos, libros o piezas que se le hayan confiado.

    Art. 427. (458) La acusación del querellante particular contendrá las mismas enunciaciones del auto de acusación de oficio, deberá calificar con toda claridad el o los delitos que pretende cometidos, la participación del procesado o de cada uno de los procesados, y las circunstancias que deben influir en la aplicación de las penas y concluirá solicitando la imposición de éstas, expresa y determinadamente.


    Art. 428. El ejercicio de las acciones civiles en el plenario se efectúa por medio de una demanda, que deberá cumplir los requisitos exigidos por el artículo 254 del Código de Procedimiento Civil.
    El querellante deberá interponer su demanda civil conjuntamente con su acusación o adhesión, en un mismo escrito. Podrá, también, abandonar la acciòn penal e interponer, dentro del plazo del articulo 425, únicamente su demanda civil.
    La falta de ejercicio de la acción civil en el proceso penal, sea que se abandone la acción penal o no, no obsta a su ejercicio ante el juez civil competente.

    Art. 429. En sus escritos de adhesión, de acusación y de demanda civil, el querellante y el actor civil deberán expresar los medios probatorios de que intentan valerse, o si se atienen al mérito del sumario, renunciando a la prueba y al derecho de pedir que se ratifiquen los testigos. Si ofrecen rendir prueba de testigos presentarán una lista, individualizándolos con nombre, apellidos, profesión y domicilio o residencia y una minuta de interrogatorio.
    En los mismos escritos deberán, individualizar de igual modo, al perito o a los peritos que propongan, para los efectos del artículo 471, indicando, además, sus títulos o calidades.

    Art. 430. (459) De la acusación de oficio, de la adhesión o de la acusación del querellante particular y de la demanda del actor civil, o de todas ellas, cuando fueren dos o más, se dará traslado al procesado o procesados y a los civilmente demandados.
    El procesado o procesados serán representados y defendidos por el abogado y procurador que hubieren designado o por los que hubieren estado de turno al practicarse la notificación de que se trata en el artículo 276.
    Si las defensas de dos o más procesados de un mismo proceso fueren incompatibles entre sí, el que el juez designare será representado y defendido por el procurador y el abogado de turno y los demás lo serán por los procuradores y abogados que el juez respectivamente les señalare, salvo el caso que, en conformidad al artículo 278, hubieren nombrado otro abogado o procurador.



    Art. 431. (460) Cualquier ofendido que no haya figurado como actor civil en el sumario, podrá presentar demanda civil, en la forma dispuesta por los artículos 428 y 429 hasta antes que se notifique al procesado, o a uno cualquiera de ellos, si fueren varios, el traslado dispuesto por el artículo 430. En este caso, el juez extenderá el traslado respecto de la nueva demanda.


    Art. 432. Los mandatarios judiciales ya constituidos en el proceso, cuyos mandatos no hubieren expirado, se entienden facultados para interponer demandas civiles y para ser notificados de ellas, en su caso, a menos que se les haya negado expresamente tal facultad. En este caso y respecto de quienes no tengan mandatario en el proceso, regirán las reglas generales.

    Artículo 432 bis.- El traslado de la acusación de oficio al querellante particular y a los actores civiles deberá notificarse personalmente o por cédula.
    En la misma forma deberá notificarse al mandatario del procesado el traslado referido en los artículos 430 y 431.

  Título II
  DE LAS EXCEPCIONES DE PREVIO Y ESPECIAL
PRONUNCIAMIENTO

    Art. 433. (461) El procesado sólo podrá oponer como excepciones de previo y especial pronunciamiento las siguientes:
    1a. Declinatoria de jurisdicción;
    2a. Falta de personería del acusador;
    3a. Litis pendencia;
    4a. Cosa juzgada;
    5a. Perdón de la parte ofendida, el cual ha de ser otorgado antes de iniciarse el procedimiento respecto de los delitos que no pueden ser perseguidos sin previa denuncia o consentimiento del agraviado;
    6a. Admnistía o indulto;
    7a. Prescripción de la acción penal; y
    8a. Falta de autorización para procesar, en los casos en que sea necesaria con arreglo a la Constitución o a las leyes.



    Art. 434. (462) Durante el plenario las excepciones de previo y especial pronunciamiento se deducirán conjuntamente con la contestación a la acusación, la cual se formulará en carácter de subsidiaria.
    Sin perjuicio de lo dispuesto en el inciso anterior, el procesado también alegar las excepciones de los números 4°, 5°, 6°, 7° y 8° del artículo 433, como defensas de fondo para el caso de que no se acojan como artículo de previo y especial pronunciamiento.
    Si el procesado no las alega como defensas de fondo, el juez podrá renovar su examen en la sentencia definitiva y resolverlas, aunque las hubiere desechado como excepciones previas.
   




NOTA 1.1
    Las modificaciones introducidas al presente Código por la Ley N° 18.857, publicada en el Diario Oficial de 6 de Diciembre de 1989, rigen, según lo dispone su artículo vigésimo, noventa días después de su publicación en el Diario Oficial.
    Art. 435. (463) El procesado que dedujere artículo de previo y especial pronunciamiento acompañará a su petición los documentos justificativos de los hechos a que se refiere o manifestará las diligencias del sumario en que estén acreditados esos hechos. Si no tuviere a su disposición los documentos necesarios, designará claramente y con la posible determinación, el archivo u oficina donde se encontraren y pedirá al juez que mande agregar copia de ellos.


    Art. 436. (464) Del escrito en que el procesado o acusado deduzca el artículo se dará traslado por el término común de seis días al querellante o acusador particular, según corresponda.

    Art. 437. (465) Si el querellante o el acusador particular, en su caso, intentaren desvirtuar con otros documentos el mérito de los presentados por el procesado, los acompañarán o expresarán claramente y con la posible determinación, el archivo u oficina donde se encuentran y pedirán al juez que mande agregar copia de ellos.

    Art. 438. (466) El juez decretará la agregación de las copias que se expresan en los artículos 435 y 437, con citación de las demás partes del juicio. En virtud de este decreto, quedarán las partes autorizadas para personarse en el archivo u oficina a fin de señalar la parte del documento que deba compulsarse, si no les fuere necesaria la compulsa de todo él, y para presenciar el cotejo. Cada interesado pagará los gastos de la parte del compulsa que solicite, si no goza del privilegio de pobreza.
    El procesado podrá hacer, en el término de veinticuatro horas contadas desde que las copias pedidas por las otras partes se pusieren en su conocimiento, las observaciones que tenga a bien.


    Art. 439. (437) Los artículos de previo y especial pronunciamiento se substanciarán y fallarán como incidentes.
    Art. 440. (468) Si alguna de las excepciones opuestas fuere la de declinatoria de jurisdicción o la de litis pendencia, el juez la resolverá antes de las demás. Cuando considere procedente alguna de éstas, y la litis anterior no pendiere ante él, mandará remitir los autos al juez que considere competente, absteniéndose de resolver sobre las otras excepciones.
    Art. 441. (469) Cuando se declare haber lugar a cualquiera de las excepciones comprendidas en los números 4°, 5°, 6° y 7° del artículo 433, se sobreserá definitivamente en la causa, y se mandará que se ponga en libertad al procesado o procesados que no estén presos por otro motivo.




    Art. 442. (470) Si se declarare haber lugar al artículo de falta de autorización para procesar, el juez mandará inmediatamente subsanar este defecto.
    La causa quedará, entre tanto, en suspenso y se continuará según su estado, una vez obtenida la autorización.
    Si ésta fuere denegada, todo lo actuado quedará nulo y se sobreseerá definitivamente en la causa.
    Todo lo cual se entiende sin perjuicio de lo dispuesto en el artículo 618.
    Art. 443. (471) La resolución que deseche las excepciones 4a, 5a, 6a, 7a y 8a de las enumeradas en el artículo 433, no es apelable.
    La resolución que deseche las excepciones 1a, 2a, y 3a del citado precepto, es apelable en el solo efecto devolutivo.
    Art. 444. (472) Admitida alguna de las excepciones de que se trata en el inciso 1° del artículo precedente, el auto será consultado a la Corte de Apelaciones en los mismos casos en que la ley prescriba la consulta de una sentencia definitiva.
    Art. 445. (473) Cuando se suscitare durante el sumario un artículo de previo y especial pronunciamiento, se le substanciará y fallará en pieza separada, sin perjuicio de tomarse en cuenta para el fallo los antecedentes que arroja el sumario.
    Si durante el plenario se opusieren excepciones de previo pronunciamiento, se suspenderá el juicio principal.
    Art. 446. (474) Cuando fueren admitidas las excepciones perentorias opuestas por alguno o algunos de los procesados, el sobreseimiento será parcial; y la causa seguirá su curso respecto de los procesados restantes, o de los delitos no comprendidos en el sobreseimiento.
    Desechadas las excepciones del previo y especial pronunciamiento, el juez dará curso a la contestación a la acusación subsidiaria formulada.


    Título III
    DE LA CONTESTACION A LA ACUSACION

    Art. 447. (475) El acusado y el civilmente responsable tienen para contestar el plazo de seis días. Si son varios los acusados, o varios los demandados civiles, tendrán el término común de diez días, sin perjuicio del aumento que señala el artículo 425.
    Respecto del examen del expediente, regirán en cuanto sean aplicables, las reglas establecidas en el artículo 426.

    Art. 448. (476) En la contestación, el procesado expondrá con claridad los hechos, las circunstancias y las consideraciones que acrediten su inocencia o atenúen su culpabilidad.
    Podrá presentar una o más conclusiones con tal que sean compatibles entre sí o con tal que, si fueren incompatibles, las presente subsidiariamente, para el caso en que la sentencia deniegue la otra u otras.
    La contestación de la acusación por el acusado constituye un trámite esencial que no puede darse por evacuado en su rebeldía. Vencido que sea el plazo, si no se evacuare el trámite, el juez arbitrará las medidas para que se conteste la acusación, ya sea por el abogado que el inculpado hubiere nombrado, o por el de turno, o por el que le señalare, o por la Corporación de Asistencia Judicial o por la institución que cumpla sus finalidades, pudiendo aplicar la sanción establecida en el artículo 598 del Código Orgánico de Tribunales, en caso de contravención.


    Art. 449. (477) Puede el procesado renunciar a la práctica de las diligencias del juicio plenario y consentir en que el juez pronuncie sentencia sin más trámite que la acusación y su contestación. El juez accederá a la petición formulada a este respecto, siempre que el Ministerio Público o el querellante particular no se opongan, alegando que tienen prueba que producir durante el plenario.





    Art. 450. (478) Los acusados y el responsable civilmente manifestarán en su escrito de contestación, cuáles son los medios probatorios de que intentan valerse, y presentarán listas de testigos que hubieren de declarar a su instancia.
    En dichas listas se expresarán el nombre y apellido de los testigos, su apodo si por él son conocidos, y su domicilio o residencia. La parte que los presentare manifestará además si se encarga de hacerlos comparecer o si pide que sean citados judicialmente.
    En los mismos escritos, deberán proponer el perito o peritos para los efectos del artículo 471. Los individualizarán por su nombre y apellidos, señalarán su domicilio, las calidades y títulos que justifiquen su designación e indicarán si solicitan su informe escrito u oral, y en este último caso, si deberán ser citados o los harán comparecer.

    Artículo 450 bis.- Los demandados civiles deberán oponer todas sus excepciones en el escrito de contestación y el juez las fallará en la sentencia definitiva.
    Si se rechaza la demanda por vicios formales, sin resolver el fondo de la acción deducida, podrá renovarse ante el juez de letras en lo civil, entendiéndose suspendida la prescripción en favor del demandante civil, desde que interpuso la demanda o, en su caso, desde que se constituyó en parte civil.

  Título IV
  DE LA PRUEBA Y DE LA MANERA DE APRECIARLA
  1. De la prueba en general

    Art. 451. (479) A la contestación del procesado o del demandado civil, y si son varias, a la última, el juez proveerá recibiendo la causa a prueba, si las partes la han ofrecido en sus escritos respectivos, y en caso contrario, ordenará que se agregue a los autos la contestación a la acusación, y notificada la correspondiente resolución regirá lo dispuesto en el artículo 499.
   
   
   
     


    Art. 452 (480) No se llevará a efecto ninguna diligencia probatoria si no está ordenada por decreto judicial notificado a las partes.
    El juez no permitirá que se practiquen diligencias probatorias que no sean conducentes a demostrar los hechos materia del juicio.
    Art. 453. (481) En la resolución que recibe la causa a prueba, el juez fijará una o más audiencias dentro del probatorio para recibir las declaraciones de las partes, de testigos o de peritos. La recepción de estas pruebas continuará en las audiencias consecutivas que fueren necesarias, hasta su término.
    Las partes del juicio deberán ser notificadas con un día de anticipación a lo menos, para que por sí o por procurador puedan concurrir al acto.
    Cuando la prueba deba recibirse fuera del lugar de asiento del tribunal, las órdenes y exhortos serán librados dentro de veinticuatro horas, a más tardar.

    Art. 454. (482) Las diferentes actuaciones de prueba se practicarán en audiencia pública, excepto cuando la publicidad fuere peligrosa para las buenas costumbres; lo cual declarará en auto especial el juez de la causa.
    Los jueces durante el plenario cuidarán de señalar horas diversas para recibir la prueba en cada causa.


    Art. 455. (483) Contra la resolución que decrete o rechace una diligencia, sólo cabe reposición dentro de tercero día, sin perjuicio de que pueda ordenarse en la segunda instancia del juicio, si se renueva la petición, o de oficio.

    Art. 456. (483) Se levantará acta de lo actuado, pudiendo el juez dejar constancia de las preguntas y respuestas o de un resumen de ellas, según convenga al caso.
    El acta será suscrita por todos los asistentes; el juez podrá permitir la firma inmediata de las personas a quienes autorice para retirarse antes del término de la audiencia.

    Art. 456. bis (484) Nadie puede ser condenado por delito sino cuando el tribunal que lo juzgue haya adquirido, por los medios de prueba legal, la convicción de que realmente se ha cometido un hecho punible y que en él ha correspondido al procesado una participación culpable y penada por la ley.


    Art. 457. (485) Los medios por los cuales se acreditan los hechos en un juicio criminal, son:
    1° Los testigos;
    2° El informe de peritos;
    3° La inspección personal del juez;
    4° Los instrumentos públicos o privados;
    5° La confesión; y
    6° Las presunciones o indicios.
    Sobre cada uno de estos medios de prueba rigen las disposiciones dictadas a su respecto en la parte primera de este libro y en los párrafos siguientes.
    2. De la prueba de testigos

    Art. 458. (486) Cada parte podrá presentar, durante el plenario, hasta seis testigos para probar cada uno de los hechos que le convengan.
    INCISO SEGUNDO.- DEROGADO.-


    Art. 459. (487) La declaración de dos testigos hábiles, contestes en el hecho, lugar y tiempo en que acaeció, y no contradicha por otro u otros igualmente hábiles, podrá ser estimada por los tribunales como demostración suficiente de que ha existido el hecho, siempre que dicha declaración se haya prestado bajo juramento, que el hecho haya podido caer directamente bajo la acción de los sentidos del testigo que declara y que éste dé razón suficiente, expresando por qué y de qué manera sabe lo que ha aseverado.
    Art. 460. (488) No son testigos hábiles:
    1° Los menores de dieciséis años;
    2° Los procesados por crimen o por simple delito, y los condenados por crimen o simple delito mientras cumplen la condena, a menos de tratarse de un delito perpetrado en el establecimiento en que el testigo se halle preso;
    3° Los que hubieren sido condenados por falso testimonio; y aquellos respecto de quienes se probare que han incurrido en falsedad al prestar una declaración jurada o que se ocupen habitualmente en testificar en juicio;
    4° Los vagabundos, los de malas costumbres y los de ocupación deshonesta;
    5° Los ebrios consuetudinarios, o los que al tiempo de deponer se encontraban en estado de ebriedad;
    6° Los que tuvieren enemistad con alguna de las partes, si es de tal naturaleza que haya podido inducir al testigo a faltar a la verdad;
    7° Los amigos íntimos del procesado o de su acusador particular, los socios, dependientes o sirvientes de uno u otro y los cómplices y los encubridores del delito.
    La amistad o enemistad deberán manifestarse por hechos graves que el tribunal calificará según las circunstancias;
    8° Los que, a juicio del tribunal, carezcan de la imparcialidad necesaria para declarar por tener en el proceso interés directo o indirecto;
    9° Los que tuvieren pleito pendiente con una de las partes, con su cónyuge, hijos, padres o hermanos, o lo hubieren tenido con resultados desfavorables en los cuatro años anteriores a la declaración;
    10. Los que tuvieren con alguna de las partes parentesco de consanguinidad en línea recta o dentro del cuarto grado de la colateral; o parentesco de afinidad en línea recta o dentro del segundo grado de la colateral;
    11. Los denunciantes a quienes afecte directamente el hecho sobre que declaren, a menos de prestar la declaración a solicitud del reo y en interés de su defensa;
    12. Los que hubieren recibido de la parte que lo presenta dádivas o beneficios de tal importancia que, a juicio del tribunal, hagan presumir que no tienen la imparcialiad necesaria para declarar;
    13. Los que declaren de ciencia propia sobre hechos que no puedan apreciar, sea por la carencia de facultades o aptitudes, sea por imposibilidad material que resulte comprobada;
    14. Los que, no pudiendo exponer sus ideas de palabra o por escrito, no puedan tampoco darse a entender con perfecta claridad por medio de signos.






NOTA
    El artículo 9° de la Ley 19047, publicada el 14.02.1991, modificado por la Ley 19158, otorga facultad para mantener la palabra reo por estar empleada en sentido genérico.
    Art. 461. (489) El testimonio del mayor de dieciocho años valdrá, aun cuando se refiera a hechos ocurridos en los cuatro años anteriores a la fecha en que cumplió aquella edad.
    Art. 462. (490) Se presumirá ebrio consuetudinario al testigo que hubiere sido condenado tres veces por ebriedad, dentro de los últimos cinco años.
    Art. 463. (491) Las inhabilidades que se fundan en las circunstancias de parentesco, amistad, enemistad, vínculo social o dependencia del testigo con relación a algunas de las partes, sólo se considerarán como tales en cuanto los testigos puedan ser inspirados por el interés, afecto u odio que pudieran nacer de aquellas relaciones.
    Art. 463 bis. Tratándose de los delitos contemplados en los artículos 361 a 367 bis y 375 del Código Penal, no regirán las normas sobre inhabilidad de los testigos, contempladas en el artículo 460, que se funden en razones de edad, parentesco, convivencia o dependencia.

    Art. 464. (492) Los jueces apreciarán la fuerza probatoria de las declaraciones de testigos que no reúnan los requisitos exigidos por el artículo 459.
    Tales declaraciones pueden constituir presunciones judiciales.
    Igualmente las de testigos de oídas, sea que declaren haber oído al procesado, o a otra persona.


    Art. 465. (493) El juez examinará a los testigos acerca de los hechos pertinentes expuestos por el que los presentare, en los escritos de acusación y de contestación.
    Art. 466. (494) Los interrogatorios o contra-interrogatorios que presentaren las partes, los mandará poner el juez en conocimiento de las otras partes; quienes podrán objetarlos dentro de veinticuatro horas; y el juez resolverá dentro de las veinticuatro horas siguientes.
    El juez interrogará a los testigos al tenor de las preguntas que hubiere declarado pertinentes.
    Podrá también interrogarlos sobre otros hechos conducentes y hacerles preguntas para aclarar las formuladas por las partes.
    Estas podrán también interrogarlos con permiso del juez; quien lo concederá para hechos pertinentes sobre los cuales no hubiere declarado antes el testigo. Tampoco podrá negarlo cuando las preguntas se dirijan a establecer causales de inhabilidad de los testigos.
    Art. 467. (495) No serán apelables las resoluciones que el juez dicte en virtud de lo dispuesto en el artículo precedente; pero el tribunal podrá, al rever la sentencia, mandar adelantar las diligencias que estimare incompletas, comisionando a uno de sus miembros o al juez de primera instancia.
    Art. 468. (496) Durante el término probatorio, el juez ratificará a los testigos del sumario o a alguno de ellos, cuando lo considere conveniente o cuando lo pida alguna de las partes.
    Estas pueden asistir a la diligencia de ratificación y hacer a los testigos las preguntas que el juez estime conducentes con arreglo a lo dispuesto en el artículo 466.
    Art. 469. (497) No es necesario ratificar en el juicio plenario a los testigos del sumario, para la validez de sus declaraciones; pero, si alguna de las partes lo solicitare, se ratificará a los testigos que sean habidos y que no se hayan ratificado conforme a lo establecido en el artículo 219.
    Art. 470. (498) Si alguno de los testigos del sumario hubiere fallecido, o se ha ausentado o no pudiere ser habido, y se objeta su declaración por no estar ratificado en el plenario, el juez interrogará bajo juramento a dos personas dignas de crédito que hayan conocido a aquel testigo, acerca del concepto que tengan de la veracidad de él; y el dicho favorable de estas personas producirá el efecto de la ratificación.
    Si no pudiere practicarse esta diligencia o si las personas indicadas no abonaren al testigo, se aplicará a su declaración la regla del artículo 464.
  3. Del informe de peritos

    Art. 471. (499) Cuando se hubiere presentado informe pericial durante el sumario, las partes podrán pedir en los respectivos escritos de acusación y contestación, que el perito amplíe su informe con el solo objeto de aclarar o desvanecer dudas, o subsanar errores de que éste pueda adolecer, los que deberán ser señalados determinadamente.
    Además, podrán pedir un nuevo informe pericial y el juez resolverá de acuerdo con el inciso primero del artículo 221.
    Si en el sumario no se hubiere practicado examen pericial y las partes piden alguno durante el término probatorio, el juez lo ordenará cuando lo estimare conducente.
    En uno u otro caso, se observarán las prescripciones del Párrafo VI, Título III del Libro II.
    Los peritos pueden ser tachados por las mismas causales que los testigos.

    Art. 472. (500) El dictamen de dos peritos perfectamente acordes, que afirmen con seguridad la existencia de un hecho que han observado o deducido con arreglo a los principios de la ciencia, arte u oficio que profesan, podrá ser considerado como prueba suficiente de la existencia de aquel hecho, si dicho dictamen no estuviere contradicho por el de otro u otros peritos.
    Art. 473. (501) Fuera del caso expresado en el artículo anterior, la fuerza probatoria del dictamen pericial será estimada por el juez como una presunción más o menos fundada, según sean la competencia de los peritos, la uniformidad o disconformidad de sus opiniones, los principios científicos en que se apoyen, la concordancia de su aplicación con las leyes de la sana lógica y las demás pruebas y elementos de convicción que ofrezca el proceso.
    4. De la inspección personal del juez

    Art. 474. (502) Acerca de la existencia de los rastros, huellas y señales que dejare un delito y acerca de las armas, instrumentos y efectos relacionados con él, hará completa prueba la diligencia de la inspección ocular que haya practicado el juez asistido por el secretario, asentada en el acta correspondiente.


    Art. 475. (503) Acerca de los hechos que hubieren pasado en presencia del juez y ante el respectivo secretario, hará completa prueba la diligencia que, con las debidas formalidades, se hubiere asentado sobre el particular.
    Art. 476. (504) Se tendrá asimismo como prueba completa toda diligencia en que se hicieren constar las observaciones que el juez haya hecho por sí mismo con asistencia del secretario, en los lugares que hubiere visitado con motivo del suceso, o los hechos que hubieren pasado ante uno y otro funcionario.
    5. De la prueba instrumental
    Art. 477. (505) Todo instrumento público constituye pueba completa de haber sido otorgado, de su fecha y de que las partes han hecho las declaraciones en él consignadas.
    Art. 478. (506) Los escritos privados reconocidos por el que los hizo o firmó, tienen, respecto de los puntos contenidos en el artículo anterior, la misma fuerza probatoria que la confesión, si el reconocimiento es efectuado por el procesado; o que la declaración de testigos, en los demás casos.



    Art 479. (507) Se mandarán agregar al proceso los papeles y cartas de terceros presentados con el consentimiento de sus autores o dueños.
    Aun sin ese consentimiento, se agregarán los que el tribunal estime conducentes a la comprobación del delito o de sus perpetradores.
    Art. 480. (508) El cotejo de letras o firmas formarán una presunción o indicio de haber sido escrito o firmado un papel o documento por la persona a quien lo atribuyan los peritos que hubieren practicado la diligencia.
  6. De la confesión

    Art. 481. (509) La confesión del procesado podrá comprobar su participación en el delito, cuando reúna las condiciones siguientes:
    1a. Que sea prestada ante el juez de la causa, considerándose tal no sólo a aquel cuya competencia no se hubiere puesto en duda, sino también al que instruya el sumario en los casos de los artículos 6° y 47;
    2a. Que sea prestada libre y conscientemente;
    3a. Que el hecho confesado sea posible y aun verosímil atendidas las circunstancias y condiciones personales del procesado; y
    4a. Que el cuerpo del delito esté legalmente comprobado por otros medios, y la confesión concuerde con las circunstancias y accidentes del aquél.


    Art. 482. (510) Si el procesado confiesa su participación en el hecho punible, pero le atribuye circunstancias que puedan eximirlo de responsabilidad o atenuar la que se le impute, y tales circunstancias no estuvieren comprobadas en el proceso, el tribunal les dará valor o no, según corresponda, atendiendo al modo en que verosímilmente acaecerían los hechos y a los datos que arroje el proceso para apreciar los antecedentes, el carácter y la veracidad del procesado y la exactitud de su exposición.

    Art. 483. (511) Si el procesado retracta lo expuesto en su confesión, no será oído, a menos que compruebe inequívocamente que la prestó por error, por apremio, o por no haberse encontrado en el libre ejercicio de su razón en el momento de practicarse la diligencia.
    Si la prueba se rinde durante el sumario, se substanciará en pieza separada, y sin suspender los procedimientos de la causa principal.


    Art. 484. (512) La confesión que no se prestare ante el juez de la causa, determinado en el número 1° del artículo 481, y en presencia del secretario, no constituirá una prueba completa, sino un indicio o presunción, más o menos grave según las circunstancias en que se hubiere prestado y el mérito que pueda atribuirse a la declaración de aquellos que aseguren haberla presenciado.
    El silencio del imputado no implicará un indicio de participación, culpabilidad o inocencia.
    No se dará valor a la confesión extrajudicial obtenida mediante la intercepción de comunicaciones telefónicas privadas, o con el uso oculto o disimulado de micrófonos, grabadoras de la voz u otros instrumentos semejantes.

    Artículo 484 bis.- Si alguna de las partes lo pidiere, el acusado deberá concurrir a una o más audiencias determinadas, para ser interrogado sobre lo que haya manifestado en el sumario o sobre las declaraciones que hayan formulado los comparecientes en las audiencias del plenario. Esta diligencia podrá pedirse hasta dentro del plazo indicado en el artículo 499.

    Artículo 484 bis A.- Las demás partes podrán ser llamadas a prestar declaración en el plenario, para ser interrogadas al tenor de los puntos relativos a los hechos objeto de la acusación y de la defensa que se indiquen por la parte que lo solicite, en un sobre cerrado, que el tribunal abrirá en la audiencia.
    Las partes llamadas estarán obligadas a concurrir y con este objeto serán citadas en la forma prevista para los testigos y bajo los mismos apercibimientos.
    El juez podrá pronunciarse, de oficio o previa petición de parte y sin ulterior recurso, sobre la pertinencia de las preguntas, al momento de ser formuladas.
    No hay confesión ficta en el proceso penal.

    7. De las presunciones
    Art. 485. (513) Presunción en el juicio criminal es la consecuencia que, de hechos conocidos o manifestados en el proceso, deduce el tribunal ya en cuanto a la perpetración de un delito, ya en cuanto a las circunstancias de él, ya en cuanto a su imputabilidad a determinada persona.
    Art. 486. (514) Las presunciones pueden ser legales o judiciales. Las primeras son las establecidas por la ley, y constituyen por sí mismas una prueba completa, pero susceptible de ser desvanecida mediante la comprobación de ciertos hechos determinados por la misma ley.
    Las demás presunciones se denominan "presunciones judiciales" o "indicios".
    Art. 487. (515) Respecto a la fuerza probatoria de las presunciones legales y al modo de desvanecerlas, se estará a lo dispuesto por la ley en los respectivos casos.
    Art. 488. (516) Para que las presunciones judiciales puedan constituir la prueba completa de un hecho, se requiere:
    1° Que se funden en hechos reales y probados y no en otras presunciones, sean legales o judiciales;
    2° Que sean múltiples y graves;
    3° Que sean precisas, de tal manera que una misma no pueda conducir a conclusiones diversas;
    4° Que sean directas, de modo que conduzcan lógica y naturalmente al hecho que de ellas se deduzca; y
    5° Que las unas concuerden con las otras, de manera que los hechos guarden conexión entre sí, e induzcan todas, sin contraposición alguna, a la misma conclusión de haber existido el de que se trata.
    8. DE LA PRUEBA DE LAS ACCIONES CIVILES

    Artículo 488 bis.- La prueba de las acciones civiles en el juicio criminal se sujetará a las normas civiles en cuanto a la determinación de la parte que debe probar, y a las disposiciones de este Código en cuanto a su procedencia, oportunidad, forma de rendirla y valor probatorio.
    Las partes podrán absolver posiciones en el plenario sólo una vez sobre hechos comprendidos en la acción civil que no digan relación con la existencia del delito y la responsabilidad penal. Regirán para este efecto las reglas contenidas en el artículo 484 bis A.
    El reconocimiento que las partes hicieren en sus escritos, respecto de los hechos que las perjudiquen indicados en el inciso anterior, constituirá confesión.
    Lo previsto en este artículo se aplicará también a las cuestiones civiles a que se refiere el inciso primero del artículo 173 del Código Orgánico de Tribunales.

    Título V
    DEL TERMINO PROBATORIO
    Art. 489. (517) Regirán en las causas criminales las disposiciones contenidas en el Título Del Término Probatorio del Libro II del Código de Procedimiento Civil.
    Art. 490. (518) Las diligencias de prueba deberán ser pedidas, ordenadas y practicadas dentro del término probatorio. Si alguna de las pedidas durante el término dejare de practicarse dentro de él sin culpa de la parte que la pidió ésta podrá, dentro del plazo indicado en el artículo 499, exigir que se la lleve a efecto.
    Art. 491. (519) Durante el término probatorio, las partes podrán examinar los autos en la secretaría e imponerse de la prueba.
    Podrán también, durante el término probatorio y aun después de vencido éste, antes de la dictación de la sentencia, presentar escritos para fundar sus conclusiones en el mérito de la prueba rendida.
    Título VI
    DE LAS TACHAS
    Art. 492. (520) Cada parte puede tachar a los testigos examinados durante el sumario y a aquellos que la parte contraria presentare durante el término probatorio, que tengan alguna de las inhabilidades expresadas en el artículo 460.
    Art. 493. (521) Las tachas se deducirán, respecto de los testigos del sumario en los escritos de acusación y de contestación, y respecto de los demás, dentro de los cinco primeros días del término probatorio.
    No se admitirán las tachas alegadas cuando no se indicare circunstanciadamente la inhabilidad que afecta a los testigos y los medios de prueba con que se pretende acreditarlas.
    Si apareciere que la inhabilidad ha llegado a conocimiento de la parte a quien perjudica la declaración del testigo después de transcurridos los cinco días de que se habla en el inciso 1°, la tacha podrá ser alegada hasta dos días antes de vencerse el término probatorio.
    Art. 494. (522) El decreto recaído sobre las tachas se notificará al contendor dentro de segundo día.
    Art. 495. (523) Las diligencias con que las partes intenten acreditar o contradecir las tachas, se practicarán dentro del término de prueba.
    Art. 496. (524) El juez se pronunciará sobre las tachas en la sentencia definitiva.
    Art. 497 (525) La declaración del testigo estimado inhábil por el juez, podrá tener el valor que indica el artículo 464 de este Código.
    Título VII
    DE LA SENTENCIA

    Art. 498. Vencido el término probatorio, el secretario, de oficio, certificará este hecho.
    Art. 499. Efectuada la certificación exigida en el artículo anterior, el secretario, sin demora, presentará los autos al juez, quien, dentro del plazo fatal de seis días, los examinará para ver si se ha omitido alguna diligencia de importancia.
    Si notare alguna omisión, o si creyere necesario esclarecer algún punto dudoso, mandará practicar las diligencias conducentes, determinándolas con toda precisión, y disponiendo que se proceda con la posible brevedad.
    No faltando diligencia alguna o hechas las ordenadas conforme al inciso anterior, el juez pronunciará sentencia en el plazo legal.

    Art. 500. (528) La sentencia definitiva de primera instancia y la segunda que modifique o revoque la de otro tribunal, contendrán:
    1° La expresión del lugar y día en que se pronuncie;
    2° El nombre, apellidos paterno y materno, profesión u oficio y domicilio de las partes y además, respecto de los procesados, sus apodos, edad, lugar de nacimiento, estado civil y demás circunstancias que los individualicen;
    3° Una exposición breve y sintetizada de los hechos que dieron origen a la formación de la causa, de las acciones, de las acusaciones formuladas contra los procesados, de las defensas y de sus fundamentos;
    4° Las consideraciones en cuya virtud se dan por probados o por no probados los hechos atribuidos a los procesados; o los que éstos alegan en su descargo, ya para negar su participación, ya para eximirse de responsabilidad, ya para atenuar ésta;
    5° Las razones legales o doctrinales que sirven para calificar el delito y sus circunstancias, tanto las agravantes como las atenuantes, y para establecer la responsabilidad o la irresponsabilidad civil de los procesados o de terceras personas citadas al juicio;
    6° La cita de las leyes o de los principios jurídicos en que se funda el fallo;
    7° La resolución que condena o absuelve a cada uno de los procesados por cada uno de los delitos perseguidos; que se pronuncia sobre la responsabilidad de ellos o de los terceros comprendidos en el juicio; y fija el monto de las indemnizaciones cuando se las haya pedido y se dé lugar a ellas; y
    8° La firma entera del juez y del secretario.

    Art. 501. (529) En la sentencia definitiva, el que ha sido emplazado de la acusación debe ser siempre condenado o absuelto. De consiguiente, no puede dejarse en suspenso el pronunciamiento del tribunal, ni aun cuando la absolución haya de dictarse por insuficiencia de la prueba, salvo en los casos en que la ley permite el sobreseimiento respecto del acusado ausnte o demente.

    Art. 502. (530) Si la prueba con que se hubiere acreditado la culpabilidad del procesado consiste únicamente en presunciones, la sentencia las expondrá una a una.
    INCISO SUPRIMIDO


    Art. 503. (531) Las sentencias que condenen a penas temporales expresarán con toda precisión el día desde el cual empezarán éstas a contarse, y fijarán el tiempo de detención o prisión preventiva que deberá servir de abono a aquellos procesados que hubieren salido en libertad durante la instrucción del proceso.
    En las causas acumuladas y en las que habiendo sido objeto de desacumulación deban fallarse en la forma prevista en el artículo 160 del Código Orgánico de Tribunales, la detención o prisión preventiva que haya sufrido un procesado en cualquiera de las causas se tomará en consideración para el cómputo de la pena, aunque resulte absuelto o sobreseído respecto de uno o más delitos que motivaron la privación de libertad.



    Art. 504. (532) Toda sentencia condenatoria expresará la obligación del condenado de pagar las costas de la causa.
    Estas comprenden tanto las procesales como las personales y además los gastos ocasionados por el juicio y que no se incluyen en la costas.
    La sentencia condenatoria podrá disponer también el comiso de los instrumentos o efectos del delito cuando fuera procedente, o decretar su restitución cuando no deban caer en comiso.
    La sentencia condenatoria por el artículo 374 del Código Penal ordenará la destrucción total o parcial, según proceda, de los impresos o de las grabaciones sonoras o audiovisuales de cualquier tipo que se hayan decomisado durante el proceso.

    Art. 505. La sentencia de primera instancia y el cúmplase de la de segunda se notificarán al privado de libertad, en la forma establecida en el artículo 66.
    Al procesado que se encontrare en libertad, se le notificará personalmente la sentencia de primera instancia aun cuando tuviere defensor o mandatario constituido en el proceso. El tribunal deberá adoptar las medidas pertinentes para que la notificación se realice a la mayor brevedad. El plazo para apelar de la sentencia se computará desde la fecha de esta notificación.
    El que practique la notificación de la sentencia de primera instancia deberá entregar al procesado copia íntegra de la sentencia y, si éste fuere analfabeto, deberá leerle íntegramente el fallo. Además, le informará de su derecho a apelar en el acto y, si así lo hiciere, deberá dejar constancia de ello en el acta de notificación. El procesado no podrá declararse conforme con el fallo en este acto, pudiendo siempre hacer reserva de su derecho a apelar.
    El cúmplase de la sentencia de segunda instancia podrá notificarse personalmente al condenado que se encontrare en libertad o, por cédula, a su defensor o mandatario judicial, indistintamente. En el primer caso, junto con notificarse el cúmplase, se dará al condenado copia íntegra del fallo de segunda instancia, debiendo, además, leérsele en el evento de ser analfabeto. Por último, se le informará que la sentencia queda ejecutoriada y que no procede recurso alguno en su contra.

    Art. 506. (534) Si el estudio de los antecedentes produjere en el juez el convencimiento de que el delito de que se trata es una mera falta, dictará su sentencia con arreglo a esa convicción, pero conformándose a las disposiciones de este título.
    Art. 507. (535) Si, de los antecedentes de la causa, aparecieren hechos que den motivo suficiente para hacer cargos al procesado por un crimen o simple delito diverso del que ha sido materia de la acusación y defensa, el juez dispondrá que, una vez fallado por sentencia firme el actual proceso, se substancie por quien corresponda otro juicio acerca de la responsabilidad del procesado con respecto al delito del cual no había sido acusado.


    Art. 508. (536) Ejecutoriada una sentencia absolutoria, procederá el juez a poner en libertad al procesado que aún permanezca preso y que no lo esté por otro motivo, y a devolverle sus libros, papeles y correspondencia, en la forma expresada en el artículo 419. Mandará también cancelar las fianzas y levantar los embargos trabados en sus bienes o las prohibiciones que le hubieren sido impuestas.
    Se devolverán del mismo modo los objetos pertenecientes a terceras personas, así como lo dispone el expresado artículo 419.

    Art. 509. (537) En los casos de reiteración de crímenes o simples delitos de una misma especie, se impondrá la pena correspondiente a las diversas infracciones, estimadas como un solo delito, aumentándola en uno, dos o tres grados.
    Si por la naturaleza de las diversas infracciones éstas no pueden estimarse como un solo delito, el tribunal aplicará la pena señalada a aquella que considerada aisladamente, con las circunstancias del caso, tenga asignada pena mayor, aumentándola en uno, dos o tres grados según sea el número de los delitos.
    Podrán con todo aplicarse las penas en la forma establecida en el artículo 74 del Código Penal, si, de  seguir este procedimiento, haya de corresponder al procesado una pena menor.
    Las reglas anteriores se aplicarán también en los casos de reiteración de una misma falta.
    Para los efectos de este artículo se considerarán delitos de una misma especie aquellos que estén penados en un mismo título del Código Penal o ley que los castiga.


    Artículo 509 bis.- Ejecutoriada que sea la sentencia, el juez revisará personalmente los autos y decretará una a una todas las diligencias y comunicaciones que se requieran para dar total cumplimiento al fallo.
    En consecuencia, deberá remitir las copias que sean necesarias al establecimiento penitenciario, ordenar y controlar el efectivo cumplimiento de las multas y comisos, hacer efectiva la fianza, cuando procediere, y dirigirse a la Contraloría General de la República, al Servicio de Registro Civil e Identificación, al Servicio Electoral, en su caso, y a las demás autoridades que deban intervenir en la ejecución de lo resuelto.
    No se ordenará el archivo de los antecedentes sino después de constatarse que no hay órdenes pendientes u omitidas para el total cumplimiento y ejecución de lo resuelto.

    Título VIII
    DE LA APELACION DE LA SENTENCIA DEFINITIVA
    Art. 510. (538) Toda sentencia definitiva puede ser apelada por cualquiera de las partes, dentro de los cinco días siguientes al de la respectiva notificación.
    La apelación será entablada verbalmente o por escrito; y el recurso se otorgará siempre en ambos efectos.
    Las partes de considerarán emplazadas para concurrir al tribunal superior por el hecho de notificárseles la concesión del recurso de apelación, pudiendo hacer las peticiones que crean del caso, respecto de la sentencia apelada, al deducir el recurso o en la oportunidad a que se refiere el artículo 513.
    Art. 511. (539) El Ministerio Público tendrá el deber de apelar de toda sentencia en que, a su juicio, no se haya apreciado correctamente el delito, o no se haya impuesto al culpable la pena determinada por la ley.
    Art. 512. (540) Concedido el recurso, el juez ordenará elevar los autos al tribunal de alzada a quien corresponda conocer de la apelación, con citación y emplazamiento de las partes.
    El expediente, libros y papeles anexos serán remitidos al secretario de la Corte dentro de las veinticuatro horas siguientes a la última notificación del decreto que otorgó el recurso. El administrador del correo dará el correspondiente recibo al secretario del juzgado; y el de la Corte lo dará a la respectiva administración de correos. Si el juzgado de primera instancia tiene su asiento en la misma ciudad en que reside la Corte, los recibos se darán de secretario a secretario.
    Art. 513. Ingresados los autos, la Corte se pronunciará en cuenta sobre la admisibilidad del recurso. Si nota algún defecto, mandará subsanarlo y si encuentra mérito para considerarlo inadmisible o extemporáneo se estará a lo prescrito en el artículo 213 del Código de Procedimiento Civil.
    En caso contrario, se mantendrán los autos en secretaría por el término fatal de seis días, para que las partes puedan presentar sus observaciones escritas y transcurrido dicho plazo, se oirá la opinión del fiscal, quien deberá dictaminar en el término de seis días; pero si el proceso tiene más de cien fojas, se aplicará lo dispuesto en el inciso tercero del artículo 401.
    El apelado podrá adherirse a la apelación en el plazo fatal indicado en el inciso anterior.

    Art. 514. El fiscal podrá pedir que se confirme, apruebe o revoque la sentencia, o bien, que se la modifique a favor o en contra del procesado.
    Sin perjuicio del dictamen sobre el fondo, podrá también solicitar que se practiquen aquellas diligencias cuya omisión note y que tiendan al esclarecimiento de algún hecho importante.
    De la opinión desfavorable del fiscal se dará traslado a los procesados que hayan comparecido por el término fatal y común de seis días.
    La Corte se hará cargo en su fallo de las observaciones y conclusiones formuladas por el fiscal.



    Art. 515. (550) Antes de ser notificado del decreto de autos, podrán los interesados presentar los documentos de que no hubieren tenido conocimiento o que no hubieren podido proporcionarse hasta entonces, jurando que así es la verdad.
    El tribunal mandará agregar tales documentos al proceso con citación de las demás partes, quienes podrán deducir las objeciones que tengan contra ellos, en el término de tercero día. El escrito de objeciones se agregará también al proceso con conocimiento de las partes.
    Art. 516. (551) Antes de la citación para sentencia, podrán las partes ponerse posiciones sobre hechos diversos de aquellos que hubieren sido materia de otras posiciones en el curso del juicio.
    Dichas posiciones serán absueltas ante el ministro que la Corte designe, o ante el juez a quo, si el tribunal así lo determinare: por el procesado bajo simple promesa de decir verdad; y bajo juramento por los demás interesados.



    Art. 517. (552) Las partes podrán igualmente pedir, hasta el momento de entrar la causa en acuerdo, que ésta se reciba a prueba en segunda instancia.
    1° Cuando se alegare algún hecho nuevo que pueda tener importancia para la resolución del recurso, ignorado hasta el vencimiento del término de prueba en primera instancia; y
    2° Cuando no se hubiere practicado la prueba ofrecida por el solicitante, por causas ajenas a su voluntad; con tal que dicha prueba tienda a demostrar la existencia de un hecho importante para el éxito del juicio.
    Art. 518. (553) Cuando la petición de que se reciba prueba, en conformidad al artículo anterior, no apareciere, a primera vista, bastante justificada, el tribunal dispondrá que se la tenga presente para resolverla después de vista la causa. Apreciados entonces los motivos en que se funda la solicitud, resolverá si debe o no recibirse la causa a prueba.
    La denegación será fundada y se la dictará al fallar el negocio principal.
    Art. 519. (554) El tribunal de apelación, cuando acuerde recibir la causa a prueba, fijará el término, sin extenderlo a más de la mitad del concedido por la ley para la primera instancia; y determinará los hechos a que haya de concretarse la prueba.
    Art. 520. (555) Al solicitar el nuevo término probatorio, la parte nombrará a los testigos de que piensa valerse; y abierto el plazo, cada una de las otras partes presentará, dentro de tercero día, una lista de los suyos. La designación ha de individualizar completamente a los testigos y expresar la residencia de cada uno.
    No podrá examinarse sino a las personas comprendidas en aquella solicitud y en estas listas.
    Art. 521. (556) La prueba será recibida por el ministro del tribunal que sea comisionado o por el juez a quo o por otro juez a quien el tribunal juzgare conveniente cometerla.
    Art. 522. (557) Durante el término, podrá cada parte rendir las pruebas necesarias para acreditar que los testigos a quienes se ésta examinando tienen alguna de las inhabilidades designadas en el artículo 460.
    La lista de testigos con que se trate de comprobar las tachas será presentada, al menos, veinticuatro horas antes del examen, y mandada poner inmediatamente en conocimiento de las otras partes.
    Art. 523. (558) Vencido el término, el secretario pondrá en autos testimonio de este hecho y de la prueba rendida por cada parte; y con la cuenta que diere el relator, el tribunal llamará autos para sentencia.
    Art. 524. (559) Notificadas las partes que hayan comparecido del decreto de autos, la causa será inscrita en el rol de las que estén para tabla, y colocada en ésta tan pronto como le llegue el turno.
    Si el tribunal ejerce otra jurisdicción a más de la criminal, dará preferencia en la tabla a las causas criminales sobre las de cualquier otro orden.
    Art. 525. (560) Si el tribunal notare alguna deficiencia en la instrucción del proceso o si estimare necesarios nuevos datos para el mejor acierto del fallo, dictará un auto en que exprese las diligencias que manda practicar y el funcionario a quien las comete, que puede ser alguno de los designados en el artículo 521.
    Evacuadas las diligencias con citación de las partes, el tribunal fallará la causa, a menos que crea conveniente oír a las partes. En este caso, llamará autos; verán la causa los mismos jueces que ordenaron las diligencias del inciso anterior, y se observarán las disposiciones del artículo precedente.
    Art. 526. (561) La causa será vista en el día designado, si hubiere tiempo y no se presenta algún inconveniente; y en cuanto a la relación, informes orales y acuerdo, se observarán las reglas dadas por el Código de Procedimiento Civil y por el Código Orgánico de Tribunales, en lo que no estén modificadas por el presente.
    Durante los alegatos la Corte, por intermedio de su presidente, podrá invitar a los abogados a que extiendan sus consideraciones a cualquier punto de hecho o de derecho comprendido en el proceso, pero esta invitación no constituirá una obligación para los defensores.
    El tribunal fallará inmediatamente o dentro de seis días; pero este plazo se ampliará hasta veinte días cuando uno o más de los jueces lo pidiere para estudiar mejor el asunto, de lo cual se pondrá testimonio en los autos.
    INCISO SUPRIMIDO

    Art. 527. (562) El tribunal de alzada tomará en consideración y resolverá las cuestiones de hecho y las de derecho que sean pertinentes y se hallen comprendidas en la causa, aunque no haya recaído discusión sobre ellas ni las comprenda la sentencia de primera instancia.
    Si la sentencia de primera instancia omite considerar o resolver las acciones y excepciones civiles, el tribunal de alzada deberá resolverlas de oficio o a petición de parte.
    Artículo 527 bis.- En el cumplimiento de la decisión civil de la sentencia, regirán las disposiciones sobre ejecución de las resoluciones judiciales que establece el Código de Procedimiento Civil. Cuando el cumplimiento corresponda al tribunal que dictó el fallo de primera instancia, se llevará a efecto en cuaderno separado del juicio penal.

    Art. 528. (563) Aun cuando la apelación haya sido deducida por el procesado, podrá el tribunal de alzada modificar la sentencia en forma desfavorable al apelante.
    Puede también ordenar que se instruya nuevo proceso contra el procesado en el caso contemplado en el artículo 507.



    Artículo 528 bis.- Si sólo uno de varios procesados por el mismo delito ha entablado el recurso contra la sentencia, la decisión favorable que se dicte aprovechará a los demás en cuanto los fundamentos en que se base no sean exclusivamente personales del apelante, debiendo el tribunal declararlo así expresamente.
    También favorecerá al procesado en su responsabilidad penal el recurso de un responsable civil cuando en virtud de su interposición se establezca cualquiera situación relativa a la acción penal de que deba  seguirse la absolución del procesado, aunque éste no haya apelado el fallo desfavorable de primera instancia.



    Art. 529. (564) Cuando la sentencia de segunda instancia confirmare la de primera en todas sus partes, condenará en costas al apelante, a menos de ser éste el oficial del Ministerio Público.
    Art. 530. (565) Cuando un procesado fuere condenado por sentencia de primera instancia y absuelto por la de segunda, el tribunal hará comunicar sin demora el fallo absolutorio al juez a quo, a fin de que éste ponga inmediatamente en libertad al procesado; para lo cual podrá utilizar el telégrafo u otro medio de comunicación idóneo con las precauciones que garanticen la autenticidad de la comunicación.
    Lo mismo se observará cuando una sentencia de segunda instancia ponga término a la prisión de un individuo.

    Art. 531. (566) DEROGADO

    Art. 532. (567) Transcurrido el plazo legal para deducir el recurso de casación, sin que las partes lo hayan entablado y no tratándose de los casos de excepción que establecen los incisos siguientes, serán devueltos los autos al juez de primera instancia dentro de veinticuatro horas, para que se dé cumplimiento a la sentencia pronunciada.
    Si se hubiere deducido recurso de casación, y éste hubiere sido desechado por la Corte Suprema, o si este tribunal hubiere dado lugar a la casación en el fondo, la devolución de los autos se hará dentro de las veinticuatro horas siguientes a la última notificación del decreto que mandare cumplir la resolución de la Corte Suprema.
    INCISO SUPRIMIDO

    INCISO DEROGADO
En todos estos casos se observarán para la devolución, en sentido inverso, los mismos trámites indicados en el 2° inciso del artículo 512.

    Título IX
    DE LA CONSULTA

    Art. 533. (568) Las sentencias definitivas de primera instancia que no fueren revisadas por el respectivo tribunal de alzada por la vía de la apelación, lo serán por la vía de la consulta en los casos siguientes:
    1° Cuando la sentencia imponga pena de más de un año de presidio, reclusión, confinamiento, extrañamiento o destierro o alguna otra superior a éstas;
    2° Cuando el fallo aplique diversas penas que, sumadas, excedan de un año de privación o de restricción de la libertad, debiendo, en uno y otro caso, considerarse consultable el fallo respecto de todos los delitos sancionados, y
    3° Cuando el proceso verse sobre delito a que la ley señale pena aflictiva.
    Regirá lo dispuesto en el artículo 528 bis, en cuanto sea aplicable a la consulta.

    Artículo 534.- Los trámites de la consulta serán los mismos indicados en el Título VIII "De la apelación de la sentencia definitiva", con la salvedad de que se verán en cuenta. En los casos en que la Corte funcione dividida en salas, las causas serán distribuidas entre éstas por el Presidente de la Corte, mediante sorteo.
    Si el informe del Fiscal es desfavorable al procesado o cualquiera de las partes pidiere alegatos dentro de los seis días siguientes a la fecha de ingreso del expediente a la secretaría de la Corte, deberán traerse los autos en relación.



NOTA:  10.1
    El Artículo Transitorio de la LEY 18882, publicada el 20.12.1989, dispuso que la modificación al presente artículo regirá aún, respecto de aquellas causas en las cuales se hubiese ordenado traer los autos en relación, a menos que cualquiera de las partes solicite alegatos dentro de los seis días siguientes a la fecha de publicación de esta ley, en cuyo caso se mantendrá el decreto de autos, en relación.
    Título X
    DEL RECURSO DE CASACION

    1. De la casación en general

    Art. 535. (575) La casación en materia penal se rige, salvo lo dispuesto en el Título I del Libro III de este Código, por las prescripciones de los párrafos 1° y 4° del Título XIX, Libro III del Código de Procedimiento Civil, en lo que no sea contrario a lo establecido en el presente título. Regirá también lo dispuesto en el artículo 798 del citado Código.
    No será aplicable lo dispuesto en el inciso segundo del artículo 782 del Código de Procedimiento Civil a los recursos de casación en el fondo que se interpongan en contra de sentencias condenatorias que apliquen penas privativas de libertad.

NOTA:  11.1
    El artículo 1° transitorio de la Ley N° 19.374, publicada en el "Diario Oficial" de 18 de Febrero de 1995, dispuso su entrada en vigencia noventa días después de su publicación en el Diario Oficial, con las excepciones que la misma ley indica.
    Art. 536. (576) Pueden interponer el recurso de casación los que son parte en el juicio, y los que aun sin haber litigado, sean comprendidos en la sentencia como terceros civilmente responsables.
    El actor civil podrá deducirlo en cuanto la sentencia resuelva acerca de sus pretensiones civiles.


    Artículo 536 bis.- El recurso de casación en la forma contra la sentencia de primera instancia debe ser interpuesto dentro del plazo concedido para apelar.
    Si también se deduce el recurso de apelación, se entablarán ambos conjuntamente, a menos que se haya apelado en el acto de la notificación o que la ley establezca un plazo inferior para alzarse, en cuyo caso el escrito de casación podrá presentarse por separado, en el término de cinco días.

    Art. 537. (577) DEROGADO.-


    Art. 538. DEROGADO.-


    Art. 539. (578) La sentencia de término condenatoria en proceso sobre crimen o simple delito no tiene la fuerza de cosa juzgada, mientras dura el plazo para formalizar el recurso de casación.
    Si se interpusiere este recurso, mientras penda su conocimiento, aquélla queda en suspenso.
    Pero si la sentencia de término absuelve al procesado, éste será desde luego puesto en libertad sin la espera de los incisos precedentes.


    Art. 540. (579) Para la elevación de los autos al tribunal superior y su devolución al inferior, se observarán las prescripciones establecidas en el artículo 512.
    Inciso segundo.- Derogado

    2. Del recurso de casación en la forma

    Art. 541. (580) El recurso de casación en la forma sólo podrá fundarse en alguna de las causales siguientes:
    1a. Falta de emplazamiento de alguna de las partes;
    2a. No haber sido recibida la causa a prueba, o no haberse permitido a alguna de las partes rendir la suya o evacuar diligencias probatorias que tengan importancia para la resolución del negocio. Para alegar esta causal contra una sentencia de segunda instancia será menester que se haya pedido expresamente, en dicha instancia, que se reciba la causa a prueba y que este trámite sea procedente;
    3a. No haberse agregado los instrumentos presentados por las partes;
    4a. No haberse hecho la notificación de las partes para alguna diligencia de prueba;
    5a. No haberse fijado la causa en la tabla para su vista en los tribunales colegiados, en la forma establecida en el artículo 163 del Código de Procedimiento Civil;
    6a. Haber sido pronunciada la sentencia por un tribunal manifiestamente incompetente, o no integrado con los funcionarios designados por la ley;
    7a. Haber sido pronunciada por un juez o con la concurrencia de un juez legalmente implicado, o cuya recusación estuviere pendiente o hubiere sido declarada por tribunal competente;
    8a. Haber sido acordada en un tribunal colegiado por menor número de votos o pronunciada por menor número de jueces que el requerido por la ley; o con la concurrencia de jueces que no hayan asistido a la vista de la causa o faltando alguno de los que hayan asistido a ella;
    9a. No haber sido extendida en la forma dispuesta por la ley;
    10. Haber sido dada ultra petita, esto es, extendiéndola a puntos inconexos con los que hubieren sido materia de la acusación y de la defensa;
    11. Haber sido dictada en oposición a otra sentencia criminal pasada en autoridad de cosa juzgada; y
    12. Haberse omitido, durante el juicio, la práctica de algún trámite o diligencia dispuesto expresamente por la ley bajo pena de nulidad.
    Cuando el recurso de casación en la forma se dirija contra la decisión civil, podrá fundarse en las causales anteriores, en cuanto le sean aplicables, y además en alguna de las causales 4a., 6a. y 7a. del artículo 768 del Código de Procedimiento Civil.

    Art. 542. (581) Cuando la causa alegada necesitare de prueba, el tribunal abrirá para rendirla un término prudencial, que no exceda de diez días.
    Art. 543. (582) La vista de la causa se hará en la misma forma que la del recurso de apelación; y el fallo se expedirá en el término fijado para dicho recurso.
    Art. 544. (583) La sentencia que se pronuncie sobre el recurso de casación en la forma expondrá brevemente las causales de nulidad deducidas y los fundamentos alegados; las razones en cuya virtud el tribunal acepta una o rechaza cada una de las causales deducidas; y la decisión que declare la validez o la nulidad de la sentencia atacada.
    Aceptando una de las causales, el tribunal no necesita pronunciarse sobre las otras.
    Cuando se acoja un recurso de casación en la forma por algunas de las causales 9a., 10a. y 11a. del artículo 541, el tribunal dictará, acto continuo y sin nueva vista, pero separadamente, la sentencia que crea conforme a la ley y al mérito del proceso, pudiendo para estos efectos reproducir los fundamentos de la resolución casada que en su concepto sean válidos para fundar la decisión.
    Las mismas reglas se aplicarán si la sentencia es casada de oficio.
    En los demás casos, se procederá como lo ordena el artículo 786 del Código de Procedimiento Civil.

    Art. 545. (584) Cuando el tribunal estimare que la falta de observancia de la ley de procedimiento que ha dado causa a la nulidad, proviene de mera desidia del juez o jueces que dictaron la sentencia anulada, impondrá a éstos el pago de las costas causadas, sin perjuicio de alguna otra medida correccional indicada por la ley.
    Si hay antecedentes para estimar que la contravención a la ley fue cometida a sabiendas o por negligencia e ignorancia inexcusables, se ordenará someter a juicio al juez o jueces a quienes se presuma culpables.
  3. Del recurso de casación en el fondo

    Art. 546. (585) La aplicación errónea de la ley penal que autoriza el recurso de casación en el fondo, sólo podrá consistir:
    1° En que la sentencia, aunque califique el delito con arreglo a la ley, imponga al delincuente una pena más o menos grave que la designada en ella, cometiendo error de derecho, ya sea al determinar la participación que ha cabido al condenado en el delito, ya al calificar los hechos que constituyen circunstancias agravantes, atenuantes o eximentes de su responsabilidad, ya, por fin, al fijar la naturaleza y el grado de la pena;
    2° En que la sentencia, haciendo una calificación equivocada del delito, aplique la pena en conformidad a esa calificación;
    3° En que la sentencia califique como delito un hecho que la ley penal no considera como tal;
    4° En que la sentencia o el auto interlocutorio, calificando como lícito un hecho que la ley pena como delito, absuelva al acusado o no admita la querella;
    5° En que, aceptados, como verdaderos los hechos que se declaran probados, se haya incurrido en error de derecho al admitir las excepciones indicadas en los números 2°, 4°, 5°, 6°, 7° y 8° del artículo 433; o al aceptar o rechazar en la sentencia definitiva, las que se hayan alegado en conformidad al inciso 2° del artículo 434;
    6° En haberse decretado el sobreseimiento incurriendo en error de derecho al calificar las circunstancias previstas en los números 2°, 4°, 5°, 6° y 7° del artículo 408; y
    7° En haberse violado las leyes reguladoras de la prueba y siempre que esta infracción influya substancialmente en lo dispositivo de la sentencia.
    En cuanto al recurso de casación en el fondo se dirija contra la decisión civil de la sentencia, regirá lo dispuesto en el artículo 767 del Código de Procedimiento Civil.

    Art. 547. (586) En la sentencia, que deberá dictarse dentro de los veinte días siguientes, se expondrá: los fundamentos que sirvan de base a la resolución del tribunal; la decisión de las diversas cuestiones controvertidas; y la declaración explícita de si es nula o no la sentencia reclamada.
    Art. 548. (587) En los casos en que la Corte Suprema acoja el recurso deducido en interés del condenado, podrá aplicar a éste, como consecuencia de la causal acogida y dentro de los límites que la ley autoriza, una pena más severa que la impuesta por la sentencia invalidada.
    Si sólo uno de entre varios procesados ha entablado el recurso, la nueva sentencia aprovechará a los demás en lo que les sea favorable, siempre que se encuentren en la misma situación que el recurrente y les sean aplicables los motivos alegados para declarar la casación de la sentencia.


    Art. 549. (588) Notificada a las partes la sentencia de este tribunal, el proceso será devuelto a la Corte de Apelaciones dentro de segundo día, con las formalidades a que se refiere el artículo 540.
    La sentencia de la Corte Suprema y la de la Corte de Apelaciones serán publicadas en la "Gaceta de los Tribunales".
NOTA:  11
    Actualmente es la "Revista de derecho, Jurisprudencia y Ciencias Sociales y Gaceta de los Tribunales" la cual se reputa como "Gaceta de los Tribunales" para todos los efectos legales y reglamentarios, según al Art. 2° del Decreto N° 3.914, de 7 de agosto de 1950, publicado en el Diario Oficial de 21 de noviembre de 1950, que refundió ambas revistas a partir del 1° de enero de 1951.
    Libro Tercero
    DE LOS PROCEDIMIENTOS ESPECIALES
    Título I
    DEL PROCEDIMIENTO SOBRE FALTAS
    Art. 550. (589) Todo juicio sobre faltas se tramitará conforme al presente título y en los particulares a que él no provea, conforme a las prescripciones compatibles del Libro II.
    Art. 551. (590) El juicio sobre faltas será verbal y breve; pero si se sigue ante el juez de letras o ante un ministro de la Corte de Apelaciones, en el caso del número 2° del artículo 77 y en el del artículo 506, se le tramitará en la forma prescrita por el Libro II.
    Art. 552. (591) Pueden perseguirse de oficio las faltas no expresadas en el número 11 del artículo 18.
    Art. 553. (592) En la primera instancia del juicio sobre faltas seguido de oficio, hará de acusador público el empleado de policía que dio parte del hecho al tribunal o le presentó al inculpado; o la persona a quien el tribunal designare a falta de ellos.
    Con todo, no será necesaria la asistencia a declarar de los mencionados funcionarios policiales, ni de los que figuren como testigos en la denuncia, salvo que el inculpado impugne la declaración del funcionario o del testigo y que el juez por resolución fundada ordene su comparecencia.
    La denuncia contenida en un parte policial deberá expresar si se citó al inculpado para que concurriere al tribunal en día y hora determinado, bajo apercibimiento de proceder en su rebeldía. Tal citación deberá hacerse por escrito, entregándole el respectivo documento si estuviese presente, o mediante nota que se dejará en lugar visible de su domicilio, si estuviese ausente. Una copia de la citación se acompañará a la denuncia, con indicación de la forma en que se puso en conocimiento del inculpado.
    Si la falta consistiere en el hurto de especies cuyo valor no exceda de una unidad tributaria mensual, se acompañará al respectivo parte policial una declaración jurada del afectado, si fuese habido, sobre la preexistencia de las cosas sustraídas y una apreciación de su valor.

    Art. 554. (593) Hecha la denuncia o presentada la querella, el tribunal la mandará poner en conocimiento del querellado; fijará día y hora para el juicio, dentro de quinto día; y ordenará que el acusador y el acusado comparezcan con sus testigos y documentos, bajo apercibimiento de proceder en rebeldía de los inasistentes.
    Al mismo tiempo, requerirá informe acerca de las anotaciones del inculpado en el Registro General de Condenas.
    Tratándose de la denuncia a que se refiere el artículo precedente, y cumplidos los trámites establecidos en dicha disposición, el juez podrá dictar resolución de inmediato, si estima que no hay necesidad de practicar diligencias probatorias.
    En caso de que el denunciado no haya sido citado de conformidad al artículo anterior, la notificación de la denuncia se hará mediante carta certificada enviada a su domicilio.

    Art. 555. (594) Al deducir la acusación puede el actor pedir que el tribunal mande citar a alguno o más de los testigos de que piensa valerse; y el acusado puede hacer una petición semejante dentro de los dos días siguientes a la notificación de la acusación.
    El tribunal dará la orden; y apercibirá a los testigos hasta obtener que comparezcan, sin perjuicio de la pena determinada en el número 1° del artículo 496 del Código Penal.
    Esta orden podrá ser notificada, no sólo por un ministro de fe, sino por un empleado de policía o por cualquiera persona a quien el tribunal la cometa.
    El tribunal podrá, por esta solicitud, postergar la vista de la causa hasta por cinco días.
    Art. 556. (595) Aunque no comparezcan todos los testigos citados, se hará la vista de la causa, oyendo a las partes y a los testigos presentes.
    Si una de las partes estimare innecesario presentar otros testigos, a más de los que han comparecido, podrá pedir al tribunal que pronuncie sentencia sin esperar a los inasistentes.
    En el caso contrario, el tribunal señalará nuevo día para continuar la vista con las partes y los testigos que no habían comparecido; decretará orden de detención contra ellos y los declarará incursos en el apercibimiento del 2° inciso del artículo 555.
    INCISO DEROGADO

    Artículo 557.- Si alguno de los testigos residentes en el territorio jurisdiccional del tribunal que conoce del proceso, estuviere imposibilitado para comparecer, el tribunal irá a tomarle declaración o cometerá esta diligencia a un ministro de fe.

    Art. 558. (597) Si el inculpado o los testigos residieren fuera del distrito jurisdiccional en que se dice que se cometió la falta, el juez dirigirá exhorto a la autoridad judicial respectiva para que, antes del día de la vista de la causa, les tome declaración al tenor de los hechos que expresará.
    Artículo 559.- Si el inculpado hubiere sido detenido, el juez pondrá en su conocimiento la denuncia respectiva y lo interrogará de acuerdo a su contenido. En caso que el inculpado reconociera ante el tribunal su participación en los hechos constitutivos de la falta que se le atribuye y se allanare a la sanción que el mismo tribunal le advirtiere que contempla la ley para estos casos, se dictará sentencia definitiva de inmediato, la que no será susceptible de recurso alguno. El juez, en este evento, no aplicará la sanción en su grado máximo, salvo que el infractor sea reincidente o haya incurrido en faltas reiteradas. La sentencia se notificará al denunciante o querellante particular, si lo hubiere.
    Si el detenido negase la existencia de la falta o su participación punible en ésta, se procederá a la vista de la causa en la audiencia inmediata, a menos que sea necesario postergarla para reunir las pruebas. En tal caso, el inculpado será puesto en libertad, cuando proceda esta medida con arreglo a la ley, pero con la obligación de comparecer al juicio.

    Art. 560. (599) Puede, no obstante, el inculpado excusarse de comparecer personalmente nombrando un apoderado que lo represente; o defendiéndose en escrito, que será leído en la audiencia; salvo el caso de que su presencia sea indispensable, a juicio del juez, para la acertada resolución del negocio.
    Art. 561. (600) La vista de la causa consistirá en lo siguiente:
    Estando presente las partes y los testigos, o en rebeldía de aquéllas, el juez hará dar lectura a la  acusación y a los antecedentes; el inculpado expondrá su defensa; el juez interrogará a los testigos y las partes podrán dirigirles preguntas calificadas por él; si se opusieren tachas a algunos, el juez interrogará sobre ellas a los mismos, a las partes y a los otros testigos; y dictará un acta sucinta, la cual firmará con las partes, los testigos y el secretario.


    Art. 562. (601) El juez pronunciará sentencia acto continuo o al día siguiente.
    El juez apreciará la prueba y los antecedentes de la causa de acuerdo con las reglas de la sana crítica.
    La sentencia expresará la fecha, la individualización del inculpado y del denunciante y querellante, si los hubiere; los hechos constitutivos de la falta; somera y brevemente los fundamentos de hecho y de derecho en que se funda, y si absuelve o condena al inculpado, señalando, en este último caso, la pena a que se le condena.

    Artículo 562 bis.- Las resoluciones se notificarán por carta certificada que deberá contener copia íntegra de ellas, salvo la que ordene prisión, que será notificada en persona al condenado.
    Se notificarán también por carta certificada las sentencias dictadas en rebeldía del denunciado a quien no se haya entregado personalmente la citación, en el caso previsto en el artículo 553.
    Se entenderá practicada la notificación por carta certificada al tercer día contado desde la fecha de su recepción por la oficina de Correos respectiva, lo que deberá constar en un libro que, para tal efecto, deberá llevar el secretario. Lo anterior es sin perjuicio de lo dispuesto en el párrafo cuarto del Título III del Libro Primero.

    Art. 563. (602) Transcurridos cinco días desde la notificación de la sentencia sin que las partes hayan deducido recurso de apelación, será aquélla ejecutada por el mismo juez que la pronunció.
    Las multas deberán ser enteradas por el infractor dentro de los cinco días hábiles siguientes a la fecha en que quede ejecutoriada la resolución respectiva o en las fechas que el juez determine, en uso de la facultad que le confiere el inciso segundo del artículo 70 del Código Penal siempre que este último no sea inferior al plazo precedente. En caso de retardo en el pago, el tribunal podrá decretar por vía de sustitución y apremio la reclusión nocturna del infractor a razón de una noche por cada quinto de unidad tributaria mensual, con un máximo de quince noches, sin perjuicio de lo establecido en el artículo siguiente. La reclusión nocturna consistirá en el encierro en establecimientos especiales, separados de los que alberguen a personas privadas de libertad, y se regirá, en lo que sea aplicable, por las disposiciones  de la ley Nº 18.216 y su reglamento.

    Art. 564 (603) Si resultare mérito para condenar por faltas a un inculpado contra quien nunca se hubiere pronunciado condenación, el juez le impondrá la pena que corresponda; pero, si aparecieren antecedentes favorables, podrá dejarla en suspenso hasta por un año, declarándolo en la sentencia misma, y apercibiendo al inculpado para que se enmiende.
    Si dentro de ese plazo, éste reincidiere, el fallo que se dicte en el segundo proceso lo condenará a cumplir la pena suspendida y la que corresponda a la nueva falta, simple delito o crimen de que se le juzgue culpable.
    El juez no podrá hacer uso de la facultad que se le confiere en el inciso primero cuando la falta sea alguna de las que contempla el N° 19 del artículo 494, o el N° 21 del artículo 495.En el caso de la falta contemplada en el artículo 494 bis del Código Penal, no podrá suspenderse, al mismo tiempo, la pena de prisión y la de multa.
    Cualquiera sea la falta, si de los antecedentes personales del infractor, su conducta anterior y posterior a ella y la naturaleza, móviles y modalidades determinantes del hecho punible, puede presumirse que no volverá a delinquir, el juez, una vez ejecutoriada la sentencia, podrá conmutar la pena de multa, de acuerdo con el infractor, por la realización de trabajos determinados en beneficio de la comunidad.
    La resolución que otorgue la conmutación deberá señalar expresamente el tipo de trabajo, el lugar donde deba realizarse, su duración y la persona o institución encargada de controlar su cumplimiento.
    El tiempo que durarán estos trabajos quedará determinado reduciendo el monto de la multa a días, a razón de un día por un quinto de unidad tributaria mensual, los que podrán fraccionarse en horas para no afectar la jornada laboral o escolar que tenga el infractor, entendiéndose que el día comprende ocho horas laborales. Los trabajos se desarrollarán durante un máximo de ocho horas a la semana, y podrán incluir días sábado y feriados.
    Si no se realizaren en forma cabal y oportuna los trabajos determinados por el tribunal quedará sin efecto la conmutación por el solo ministerio de la ley, y deberá cumplirse íntegramente la sanción primitivamente aplicada, a menos que el juez, por resolución fundada, adopte otra decisión.



    Art. 565. (604) La apelación sólo procederá contra la sentencia definitiva y el recurso será otorgado en ambos efectos.
    Por el hecho de la notificación de la concesión del recurso, se entenderán emplazadas las partes para comparecer ante el tribunal de alzada en el término legal; que será de tres días cuando el tribunal sea constituido por el juez de letras del territorio jurisdiccional, y el de emplazamiento en los demás casos.

    Art. 566. (605) Recibido en el tribunal de alzada el proceso fallado por un juez inferior y transcurrido el término del emplazamiento, el juez señalará día para la vista de la causa; la cual se hará hayan o no comparecido las partes.
    Si la sentencia de primera instancia hubiere sido pronunciada por un juez letrado, la Corte de Apelaciones tramitará el recurso como si se tratase de un artículo; y lo resolverá, aunque las partes no comparezcan.
    Art. 567. (606) El tribunal de alzada podrá admitir a las partes las pruebas que no hubieren producido en primera instancia; pero la testimonial sólo cuando no se la hubiere podido rendir en dicha instancia y acerca de hechos que no figuren en la prueba rendida y que sean necesarios en concepto del tribunal para la acertada resolución del juicio.
    Para el efecto podrá abrir un término que no pase de seis días. La prueba se recibirá conforme a las reglas establecidas en este título, y la Corte comisionará para recibirla a uno de sus ministros o a un juez letrado.
    Art. 568. (607) No procede el recurso de casación en la forma ni en el fondo en estos juicios; sin embargo, las Cortes de Apelaciones podrán anular de oficio las sentencias por las causales 1a, 6a, 7a, 10a y 11a del artículo 541.
    Art. 569. (612) Si el tribunal que conoce en un juicio sobre faltas, estima que el hecho que ha motivado el proceso constituye un simple delito o un crimen, dará a la causa la tramitación prescrita en el Libro II de este Código; y, si no fuere competente para seguir conociendo, remitirá los atecedentes al tribunal a quien corresponda.
    En aquellos casos en que se ponga a disposición del juez de menores al inculpado de haber cometido alguna de las faltas señaladas en el artículo 494, N° 19, del Código Penal, dicho juez podrá imponer al menor de dieciséis años o al que sea declarado sin discernimiento, alguna de las medidas establecidas en la Ley de Menores, N° 16.618, o la de participar en actividades determinadas en beneficio de la comunidad, si resultare conducente a su rehabilitación. Estas actividades deberán fijarse de común acuerdo con el representante legal del menor o con el defensor público, en su caso; se regirán en cuanto a su forma por lo dispuesto en el artículo 564, y no podrán extenderse por más de dos meses.

    Art. 570. (613) En todo juzgado se llevará un libro en que se anoten las sentencias que se dicten en los juicios sobre faltas, debiendo el juez dar cumplimiento a lo dispuesto en el artículo 509 bis, según sea procedente en derecho.                                    Art. 3
    Los procesos se ventilarán en cuadernos separados, que se archivarán anualmente en la secretaría del juzgado de letras.
    Pero si el juicio no diere lugar a más tramitación que la citación de las partes, pueden estamparse en el mismo libro el acta original del comparendo y la sentencia del juez.

    Título II
    DEL PROCEDIMIENTO EN LOS JUICIOS EN QUE SE EJERCITA
LA ACCION PRIVADA QUE NACE DE CRIMEN O SIMPLE DELITO
    Art. 571. (614) La acción penal privada que nace de un crimen o de un simple delito, se ejercitará según las reglas dictadas para el ejercicio de la acción pública, en cuanto fueren compatibles con las que se establecen en el presente título.
    Art. 572. (615) El juicio empezará por querella.

    Art. 573. (616) No se dará curso a las querellas por los delitos de adulterio, de calumnia o de injuria después de cinco años contados desde que se cometieron.
    Tampoco se dará curso a la formulada por el delito de matrimonio celebrado por menores sin el consentimiento de sus padres o de las personas que hagan sus veces para este efecto, si consta o aparece que el padre o dichas personas han tenido conocimiento del matrimonio dos meses, a lo menos, antes de querellarse.
    Art. 574. (617) Si se trata de delitos de calumnia o injuria, el juez proveerá la querella citando al querellante y querellado a un comparendo para un día determinado, dentro del quinto.
    El comparendo sólo tendrá por objeto procurar un avenimiento que ponga término al juicio.
    Si el comparendo no se verifica por inasistencia del querellado o si el avenimiento no se produce, la causa seguirá su curso, de acuerdo con el artículo 577.
    Las partes podrán concurrir al comparendo por medio de mandatarios debidamente facultados para llegar a un avenimiento.
    Art. 575. (618) Si no asiste el querellante, o su mandatario con la facultad indicada en el artículo anterior, se le tendrá por desistido de su acción.
    No obstante, la parte inasistente, podrá dentro de los tres días siguientes al fijado para el comparendo, justificar que se encontró en la imposibilidad de concurrir, en cuyo caso se decretará por una sola vez nueva citación a comparendo.
    Art. 576. (620) Cuando la querella verse sobre calumnia o injuria proferida por escrito, se presentará el documento que la contuviere.
    Si hubiere sido inferida en juicio, acompañará el querellante un testimonio del escrito o documento en que se hubiere vertido, un certificado en que consten la terminación del juicio y la resolución del tribunal que hubiere declarado que la calumnia o la injuria dan mérito para proceder criminalmente.
    Art. 577. (621) En el caso del inciso 3° del artículo 574, o si se trata de un delito diverso de los de calumnia o injuria, el juez mandará recibir la información ofrecida por el querellante para acreditar los hechos que constituyen el delito y sus circunstancias.
    Art. 578. (622) Toda información será recibida por el juez al tenor de la querella en los días inmediatos;
y, mientras se la rinde, se mandarán practicar las diligencias periciales o cualesquiera otras que sean necesarias para la comprobación del delito y la determinación del delincuente.
    Art. 579. (247) En los juicios en que se ejercite la acción privada, los peritos serán nombrados por las partes de común acuerdo, o por el juez en su desacuerdo.
    Art. 580. Las actuaciones del sumario serán públicas, salvo que por motivos fundados, el juez ordene lo contrario.
    Art. 581. (624) Si el juez no encuentra mérito para sobreseer, ejecutoriada que sea la resolución que declara cerrado el sumario, y siempre que el inculpado haya sido objeto de un auto de procesamiento, ordenará que el querellante formule acusación dentro del término fatal de seis días. En el mismo plazo podrá interponer la acción civil, la que se tramitará conjuntamente con la acción penal.
    El sobreseimiento deberá ser definitivo; y el temporal sólo procederá en los casos de los números 3° y 4° del artículo 409.

    Art. 582. (625) Formulada la acusación y la acción civil en su caso, el querellado tendrá el plazo fatal de seis días para contestarla.


    Art. 583. (626) No es necesario oír al Ministerio Público en los juicios sobre calumnia o injuria inferidas a particulares. En los demás juicios en que se ejercite la acción privada, el Ministerio Público será oído antes de pronunciarse la sentencia definitiva.
    Art. 584. (629) La sentencia condenará en costas a la parte que fuere vencida.
    Art. 585 (630) Será consultada la sentencia que se pronunciare en alguno de los casos del artículo 533.
    Art. 586. (631) Si el querellado, en los delitos de injuria y calumnia, después de notificada la querella, desobedece o elude la citación o la orden de detención o prisión, el juicio, por esta sola circunstancia, se seguirá en su rebeldía hasta su conclusión definitiva.
    No será necesaria la declaración del inculpado para someterlo a proceso.
    En estos casos el procesado será defendido y representado por el abogado y el procurador de turno.
    Las notificaciones se practicarán al procurador de turno en la forma ordinaria, incluso las que deban hacerse personalmente al procesado.
    La disposición anterior se aplicará cada vez que el querellado incurra en las rebeldías a que se refiere el inciso 1°.
    Se aplicará también en este caso lo dispuesto en el artículo 604 aun cuando el procesado haya sido condenado a pena corporal.


    Art. 587. (632) Si el querellante o el querellado no practican las diligencias necesarias para dar curso progresivo al procedimiento durante treinta días, el tribunal que esté conociendo de la causa en primera o en segunda instancia, de oficio o a petición de parte, formulada en cualquier estado del juicio, declarará abandonada la acción.
    Esta declaración producirá los efectos del sobreseimiento definitivo.
    Lo mismo se observará si, habiendo muerto o caído en incapacidad el querellante, no ocurren sus herederos o sus representantes legales a sostener la acción, dentro del término de sesenta días.
    Este sobreseimiento no obsta para que el ofendido persiga por la vía civil las indemnizaciones que se le deben.
    Art. 588. Cuando en estos juicios proceda la acumulación de autos en conformidad a las reglas generales, el más nuevo se acumulará al más antiguo.
  Título III
  DEL PROCEDIMIENTO POR CRIMEN O SIMPLE DELITO CONTRA
PERSONAS AUSENTES

    Art. 589. (633) Será considerado como ausente el inculpado o procesado cuyo paradero fuere desconocido, o que residiere en el extranjero sin que sea posible u oportuno obtener su extradición para que comparezca ante el tribunal que debe juzgarlo.



    Art. 590. (634) Para que tengan valor legal en contra de un procesado ausente las diligencias del sumario, y las del plenario cuando se trate de delitos que no merezcan pena corporal, es menester que previamente sea declarado rebelde.



    Art. 591. (635) El inculpado o procesado será declarado rebelde:
    1° Cuando, citado al juicio por haber mérito para proceder en su contra por alguno de los simples delitos expresados en el artículo 247 y las faltas a que se refiere el artículo 494 bis del Código Penal, no comparece, y mandado aprehender, no se le encuentra en su casa ni en otra parte, y se ignora su paradero;
    2° Cuando, decretada su detención o prisión preventiva, no pudiere encontrársele en su casa ni en otra parte, y se ignora su paradero;
    3° Cuando, puesto en libertad bajo fianza, no compareciere a los actos del juicio en que se requiera su presencia, o no obedeciere al llamamiento del juez;
y, mandado aprehender, no fuere encontrado en su casa ni en otra parte, y se ignore su paradero;
    4° Cuando se fugue del establecimiento en que se hallare detenido o preso, y hubieren resultado infructuosas las diligencias practicadas para su aprehensión; y
    5° Cuando se supiere que el procesado se encuentra en país extranjero y no sea posible u oportuno obtener su extradición.


    Art. 592 (636) Antes de declarar la rebeldía del inculpado o procesado, el juez expedirá las órdenes correspondientes para citarlo o aprehenderlo.
    Las órdenes de citación o aprehensión se despacharán tanto a Carabineros como a la Policía de Investigaciones y estas instituciones deberán transmitirlas a todas sus reparticiones y unidades.



    Art. 593. (637) Las órdenes contendrán, en cuanto sea posible, los siguientes pormenores:

    1° El nombre, apellido paterno y materno, cargo, profesión u oficio del inculpado, el apodo que tenga, su residencia y las señas en virtud de las cuales pueda ser identificado;
    2° El delito por el cual se le persigue;
    3° La circunstancia señalada en el artículo 591 que haya dado motivo para expedir la orden; y
    4° El término dentro del cual deba comparecer el inculpado, bajo apercibimiento de ser declarado reo rebelde y pararle los perjuicios correspondientes. Este término será de treinta días, contados desde aquel en que se expida la primera orden de citación o aprehensión.


NOTA:
    El artículo 9° de la Ley 19047, publicada el 14.02.1991, modificado por la Ley 19158, otorga facultad para mantener la palabra reo por estar empleada en sentido genérico.
    Art. 594. (639) Si el ausente no compareciere durante el plazo señalado, el secretario certificará el hecho y el tribunal expedirá el auto en que lo declarará rebelde.
    En virtud de este auto las resoluciones que se dicten en el proceso se tendrán por notificadas personalmente al rebelde en la misma fecha en que se pronuncian.

    Art. 595. (640) Las investigaciones del sumario no se suspenderán por la ausencia del procesado, sino que seguirán adelante hasta su conclusión, sin perjuicio de practicarse las diligencias expresadas en los artículos precedentes. Una vez terminado el sumario, el juez dictará sobreseimiento definitivo o temporal, de acuerdo al mérito que arrojen los antecedentes y con arreglo a lo dispuesto en los artículos 408 y 409.
    Si el sobreseimiento se dicta en virtud de la causal 5a. del artículo 409 y el delito de que se trata merece pena corporal, se entenderá reservada la facultad de formular la acusación en forma cuando el rebelde sea habido.
    Si el delito que se imputa al ausente no merece pena corporal y hubiere mérito para ello, la causa seguirá adelante en conformidad al artículo 603.



    Art. 596. (641) Si el procesado se fugare o no compareciere durante el plenario, se suspenderá el procedimiento durante el juicio principal mientras se practican las diligencias necesarias para declararlo rebelde.
    Hecha esta declaración, el juez mandará sobreseer temporalmente y hasta que sea habido el procesado, a menos que el proceso verse sobre delito que no merezca pena corporal, caso en que se seguirá la causa en conformidad al artículo 603.

    Art. 597. (642) El auto de sobreseimiento que se dictare en conformidad a los dos artículos precedentes, será consultado en los mismos casos en que debe serlo toda sentencia definitiva.
    Art. 598. (643) Cuando el procesado sea declarado rebelde durante el plenario, se observarán las reglas siguientes:
    1° Si la acción civil no ha sido ejercitada por la parte ofendida o si deducida ésta el procesado no ha sido emplazado, se entenderá reservada y se mantendrán para ese efecto los embargos hechos y las cauciones prestadas;
    2° Si se ha deducido la acción civil y el procesado ha sido emplazado o tiene mandatario constituido en el proceso, se continuará su sustanciación no obstante el sobreseimiento, conforme a las normas de este Código, hasta el cumplimiento de la sentencia civil que se dicte, salvo el caso a que se refiere el número siguiente, y
    3° El juez podrá suspender el pronunciamiento de la sentencia civil cuando la existencia del delito haya de ser su fundamento preciso o tenga en ella influencia notoria.


    Art. 599. (644) Si el procesado se fugare después de notificada la certificación a que se refiere el artículo 498, el juez procederá a declarar la rebeldía en la forma designada en el presente título; y la causa se adelantará de oficio hasta su conclusión definitiva,  debiendo defender y representar al prófugo el abogado y procurador de turno a quien se harán las notificaciones en la forma ordinaria.

   
    Art. 600 (645) Cuando el procesado rebelde se presentare o fuere aprehendido, la causa seguirá su curso desde el punto en que se encontraba al dictarse el auto de sobreseimiento temporal. Se aplicará en este caso lo dispuesto en el inciso 2° del artículo 604.
    Si hubiere recaído sentencia de término, el juez ordenará su cumplimiento como si el procesado se hubiere encontrado presente durante el juicio, salvo lo dispuesto en el artículo 604 expresado.



    Art. 601. (646) Si, en un mismo proceso, hubiere uno o más procesados rebeldes y uno o más procesados presentes, se procederá respecto de los primeros en conformidad a las disposiciones de los artículos precedentes; pero la causa seguirá adelante por todos sus trámites con relación a los segundos, hasta su conclusión.
    Las diligencias para declarar la rebeldía de los procesados ausentes no retardarán en ningún caso la tramitación de la causa respecto de los presentes.


    Art. 602. (647) Si el procesado ausente comparece o es aprehendido antes de que se falle la causa de los presentes, podrá ésta ser suspendida hasta que la del ausente se ponga en el mismo estado. En ningún caso se pronunciará la sentencia hasta que el juicio pueda ser fallado a la vez con respecto a los que estaban presentes y a los ausentes que hubieren sido habidos.


    Art. 603. (648) Cuando el delito que hubiere motivado el proceso contra un ausente no tenga asignada una pena corporal, la causa seguirá su curso una vez que el procesado sea declarado rebelde; y éste será representado y defendido por el procurador y abogado de turno. En este caso las notificaciones se harán al primero de estos funcionarios en la forma ordinaria.

     


    Art. 604. (649) Cuando el condenado como ausente a una pena no corporal, se presentare o fuere habido, podrá, dentro de los cinco días siguientes al de la notificación que se le haga de la sentencia definitiva, apelar de ella si no hubiere sido revisada por tribunal superior, o pedir que se la deje sin efecto, reponiéndose el proceso al estado de prueba.
    En el último caso, no tendrá derecho a exigir que se ratifiquen los testigos que hubieren declarado en el sumario o en el plenario y quedará válida la prueba rendida en el juicio anterior.

    Art. 605. (650) Siempre que el procesado fuere habido, pagará las costas causadas con su rebeldía, a menos que compruebe haber tenido imposibilidad de saber que se le estaba procesando.

    Art. 606. (651) Para el efecto de lo dispuesto en los artículos precedentes, se entenderá que las penas no corporales son: la de inhabilitación para cargos y oficios públicos, derechos políticos y profesiones titulares; la de suspensión de cargo u oficio público o profesión titular; la de pérdida o comiso de los efectos o instrumentos del delito; la de inhabilidad perpetua para conducir vehículos a tracción mecánica o animal; la de suspensión par conducir vehículos a tracción mecánica o animal; y las meramente pecuniarias.
    Las demás se estimarán corporales.

    Art. 607. (642) Siempre que la causa se archive por estar en rebeldía todos los procesados, se mandarán devolver a los dueños que no resulten civil o criminalmente responsables del delito, los efectos e instrumentos del mismo, y las demás piezas de convicción que hubieren sido recogidas durante el juicio; pero, antes de hacerse la devolución, el secretario pondrá en el proceso una descripción minuciosa de los objetos que se devuelvan.
    Art. 608. (653) DEROGADO.-


    Art. 609. (654) Inmediatamente que se descubra la evasión de un preso o detenido, el juez de letras instruirá un sumario para la investigación del hecho; y procederá contra los que resulten culpables de descuido y connivencia.
    En el proceso del prófugo se pondrá testimonio de la fuga, con expresión del día en que ésta acaeció, y se dictarán las órdenes necesarias para su captura.

    Art. 610. (655) Cuando el prófugo fuere aprehendido, se procederá a identificar su persona; y, comprobada la identidad, o si ésta no ofreciere duda, se continuará la causa o se le hará cumplir la sentencia firme que hubiere recaído en ella.
    Si el prófugo era ya reo rematado cuando se verificó la evasión, se instruirá el proceso respectivo para la aplicación de las penas señaladas a los que quebrantan sentencias por el Título IV del Libro I del Código Penal.


NOTA:
      El artículo 9° de la Ley 19047, publicada el 14.02.1991, modificado por la Ley 19158, otorga facultad para mantener la palabra reo por estar empleada en sentido genérico.
    Título IV
    DEL PROCEDIMIENTO RELATIVO A PERSONAS QUE TIENEN
FUERO CONSTITUCIONAL
    1. Diputados y Senadores
    Art. 611. (656) Ningún tribunal, aunque halle mérito para imputar un delito a una persona con el fuero del artículo 58 de la Constitución, procederá contra ella, sino cuando la Corte de Apelaciones respectiva reunida en tribunal pleno, declare que ha lugar a formarle causa.

    Art. 612. (657) Tan pronto como de los antecedentes del proceso o de la información rendida, a petición de parte, aparezcan contra una persona con el fuero del artículo 58 de la Constitución datos que podrían bastar para decretar la detención de un inculpado, el juez de primera instancia elevará los autos al tribunal de alzada correspondiente, a fin de que si halla mérito, haga la declaración de que ha lugar a formación de causa.
    Si viendo el proceso por cualquier otro motivo, el tribunal de alzada halla mérito, hará igual declaración.

    Art. 613. (658) La resolución en que se declare haber lugar a formación de causa, es apelable para ante la Corte Suprema; y una vez que se halle firme será comunicada por la Corte de Apelaciones respectiva a la rama del Congreso a que pertenece el inculpado.

    Art. 614. (659) Si una persona que tiene el fuero del artículo 58 de la Constitución es detenida por habérsele sorprendido en delito flagrante, el juez a quien corresponda el conocimiento del negocio la pondrá inmediatamente a disposición de la Corte de Apelaciones respectiva, acompañando originales o copia de las diligencias que practique en conformidad a lo dispuesto en el artículo 264 de este Código.
    Sin perjuicio, remitirá más adelante, en la misma forma, las diligencias que practique con posterioridad y que sean conducentes.

    Art. 615. (660) Lo prescrito en los artículos precedentes de este título se extiende a la persona que haya sido elegida Diputado o Senador, desde el día de su elección, y en los demás casos, desde que se adquiera la respectiva calidad, de conformidad a la Constitución Política de la República.
    Si el juez estuviere conociendo ya, suspenderá todo procedimiento que a ella se refiera, mientras la Corte respectiva no declare que ha lugar a formarle causa.

    Art. 616. (661) Mientras no se declare haber lugar la formación de causa, el tribunal que conozca del proceso se abstendrá de practicar actuaciones que se refieran a la persona con el fuero del artículo 58 de la Constitución a quien se impute el delito, a menos de recibir expreso encargo de la respectiva Corte de Apelaciones.
    Art. 617. (662) Si la Corte declara no haber lugar a la formación de causa, el tribunal ante quien penda el proceso sobreseerá definitivamente a la persona favorecida con aquella declaración y hará archivar los antecedentes, si no hay otros inculpados o procesados en el mismo proceso.


    Art. 618. (663) Cuando en un mismo proceso aparezcan complicados individuos que no tuvieren el fuero del artículo 58 de la Constitución con otros que lo posean, el juicio seguirá adelante con relación a los primeros y se observarán respecto a los segundos las reglas establecidas en el presente título.
    2. Intendentes y Gobernadores
    Art. 619. (664) Ningún tribunal procederá criminalmente contra un Intendente o Gobernador, sin que el Senado haya declarado que ha lugar la formación de causa.
    Art. 620. (665) A fin de poder pedir el desafuero de un Intendente o de un Gobernador, se rendirá ante la Corte de Apelaciones respectiva, una información de los hechos en que pueda fundarse la declaración del Senado.
    El tribunal tomará conocimiento del escrito en que se ofrezca la información, designará uno de sus miembros para que la reciba, dentro de diez días; y rendida o transcurrido este plazo, la remitirá al Senado.
    Art. 621. (666) El Senado se pronunciará sobre la petición de desafuero dentro de treinta días, contados desde que se haya dado cuenta de ella en sesión de la Corporación.
    Inciso segundo.- Derogado.

    Si el Senado no se pronuncia dentro de los treinta días, se entenderá que ha lugar la formación de causa.

    Art. 622. (667) Lo dispuesto en los artículos 612 a 618 inclusive, se extiende a los casos en que estuviere complicado en una causa criminal un Intendente o un Gobernador, substituyendo las Cortes a que aluden esos artículos por el Senado.
    Título V
    DE LA QUERELLA DE CAPITULOS
    Art. 623. (668) La querella de capítulos tiene por objeto hacer efectiva la responsabilidad criminal de los jueces y oficiales de Ministerio Público por actos ejecutados en el ejercicio de sus funciones que importen una infracción penada por la ley.
    Puede ser deducida por el Ministerio Público o por un individuo particular.
    Art. 624. (670) En el escrito de querella se especificarán con toda precisión los capítulos de acusación, y se indicarán los hechos que constituyan la infracción de la ley penal cometida por el funcionario capitulado. Este escrito deberá ser firmado por abogado, si la querella no fuere entablada por el Ministerio Público.
    Art. 625. (671) La querella se presentará aparejada con todos los documentos necesarios; pero bastará que el querellante, cuando no hubiere podido obtener algunos de ellos, indique la oficina en que se encuentren y que pida que se manden agregar a los autos con la brevedad posible.
    Si para acreditar los hechos fuere preciso rendir una información sumaria, acompañará también el querellante la lista de los testigos de que piensa valerse.
    Art. 626. (672) Si la acción es ejercitada por el directamente perjudicado o por el Ministerio Público, no estará el querellante obligado a rendir fianza.
    Pero sí lo estará cualquiera otra persona para responder a las resultas del juicio e indemnizar al querellante en el caso de que sea absuelto.
    El monto de la fianza será fijado por el tribunal, tomando en cuenta la gravedad de los hechos imputados y la condición del querellante.
    Art. 627. (673) Cuando la querella fuere interpuesta por un particular, el tribunal ordenará que el Ministerio Público dictamine en el término de tercero día acerca de la procedencia de los diversos capítulos de acusación; y con lo que éste expusiere, resolverá dentro de los tres días siguientes, cuáles capítulos son aceptados y cuáles deben repelerse por no ser legales o conducentes.
    Cuando la querella fuere deducida por el Ministerio Público, el tribunal dictará, sin más trámite, dicha resolución dentro del término expresado.
    Art. 628. (674) Admitido algún capítulo de acusación, el tribunal hará agregar los documentos pedidos y recibirá la información ofrecida.
    En caso necesario, se trasladará al lugar en que el funcionario capitulado ejerce sus funciones; y, haciendo salir a éste de su territorio jurisdiccional, si así conviniere al éxito de la investigación, practicará las diligencias que no sea fácil llevar a ejecución en el lugar en que debe seguirse el juicio.
    Terminadas las diligencias, el capitulado reasumirá sus funciones.
    Art. 629. (675) Una vez levantada la información, se comunicarán los autos al querellante para que, en el término de seis días, exponga lo conveniente a su derecho. Se oirá, en seguida, al querellado dentro de igual término; y a continuación se pasarán los autos al Ministerio Público para que dictamine dentro de los seis días siguientes.
    Si no hubiere ofrecido información por el querellante, la audiencia de las partes y del Ministerio Público se verificará cuando se presenten o agreguen los documentos del caso.
    Si la querella ha sido deducida por el Ministerio Público, éste será oído antes que el querellado.
    Art. 630. (676) Dentro de los seis días siguientes a aquel en que se hubiere practicado el último de los trámites prescritos en el artículo anterior, el tribunal resolverá lo que estime de justicia, declarando en un auto fundado si es o no admisible la acusación.
    Este auto, en caso de no ser apelado, será elevado en consulta ante el tribunal de alzada correspondiente.
    Art. 631. (677) Cuando por sentencia firme se hubiere declarado admisible la acusación, el funcionario capitulado quedará de hecho suspendido del ejercicio de sus funciones; la causa se seguirá contra él en la forma ordinaria y se procederá en el acto a la iniciación del sumario y demás actuaciones a que hubiere lugar, en conformidad a las reglas establecidas en el Libro II de este Código.
    Art. 632. (678) Si la acusación fuere declarada inadmisible, el tribunal impondrá al querellante particular el pago de las costas y la indemnización de los perjuicios causados al querellado, los que serán tasados con audiencia de las partes.
    No se cancelará la fianza rendida mientras no se satisfagan las costas y los perjuicios indicados.
    Art. 633. (680) DEROGADO.-


    Art. 634.(681) DEROGADO.-


    Título VI
    DE LA EXTRADICION


    DE LA EXTRADICION


    Art. 635. (683) Cuando en la instrucción de un proceso resulte comprometido un individuo que se encuentre en país extranjero como inculpado de un delito que tenga señalada en la ley una pena privativa de libertad que en cualquiera de sus grados exceda de un año, el juez de la causa elevará los antecedentes o cumpulsas a la Corte Suprema de Justicia a fin de que este tribunal declare si debe pedirse la extrandición del procesado al Gobierno del país en el que actualmente se encuentre.
    En este caso el juez podrá procesar al inculpado ausente, sin necesidad de oírlo y sólo desde que estén acreditados los requisitos del artículo 274. El procurador de turno deberá ser notificado del auto de procesamiento. El mismo procedimiento se empleará en los casos enumerados en el artículo 6° del Código Orgánico de Tribunales.


NOTA:  12
    Véanse los artículos 344 al 381 del Código de Derecho Internacional Privado.
    Véanse:
NOTA:  13
    Véanse Tratado de Extradición suscrito con Bélgica de 29 de Mayo de 1899, promulgado el 13 de Marzo de 1904 y publicado en el Diario Oficial de 5 de Abril de 1904, ampliado por convención suscrita el 25 de Febrero de 1935, promulgada por Decreto N° 795, de 11 de Julio de 1935, y ampliado por Cambio de Notas de 28 de Abril y 5 de Mayo de 1958. (Memoria del Ministerio de Relaciones Exteriores año 1958, Tomo II, pág 486).
    Tratado de Extradición suscrito con Bolivia el 15 de Diciembre de 1910, aclarado por Cambio de Notas entre el Ministerio y la Legación de Bolivia de 27 de Abril de 1931, promulgado por Decreto N° 500, de 8 de Mayo de 1931, y publicado en el Diario Oficial de 26 de Mayo de 1931.
    Tratado sobre Extradición suscrito con Brasil en Río de Janeiro el 8 de Noviembre de 1935, promulgado por Decreto N° 1.180, de 18 de Agosto de 1937, y publicado en el Diario Oficial de 30 de Agosto de 1937.
    Tratado sobre Extradición suscrito con Colombia el 16 de Noviembre de 1914, promulgado por Decreto N° 1.472, de 18 de Diciembre de 1928, y publicado en el Diario Oficial de 7 de Enero de 1929.
    Tratado de Extradición suscrito con Ecuador el 10 de Septiembre de 1897, promulgado por Decreto de 27 de Septiembre de 1899 y publicado en el Diario Oficial de 9 de Octubre de 1899.
    Tratado de Extradición suscrito con España el 30 de Diciembre de 1895 y su "Protocolo Complementario", suscrito el 1° de Agosto de 1896, promulgado el 3 de Abril de 1897 y publicado en el Diario Oficial de 3 de Abril de 1897.
    Tratado de Extradición suscrito con Estado Unidos el 17 de Abril de 1900 y su "Protocolo Complementario", suscrito el 15 de Junio de 1901, promulgado por Decreto de 6 de Agosto de 1902, y publicado en el Diario Oficial de 11 de Agosto de 1902.
    Tratado de Extradición suscrito con Inglaterra, promulgado por Decreto de 14 de Abril de 1898, y en el publicado en el Diario Oficial de 22 de Abril de 1898. Por Cambio de Notas de 29 de Diciembre de 1927 se hizo extensivo a los territorios bajo mandato británico. Por Cambio de Notas de 12 de Abril y 7 de Agosto de 1928 relativo a la autoridad de Samo Occidental. Por Cambio de Notas de 28 de Junio y 13 de Julio de 1934 se hizo extensivo a varios Estados Malayos Federados y no Federados. Por Cambio de Notas de 12 y 29 de Marzo de 1937 se hizo extensivo a los Protectorados de Zanzíbar y de Islas Salomón. El Estado de Malawi continuá aplicando este Tratado en virtud de Notas de 6 de Enero y de 8 de Junio de 1967, y el Estado de Swazilandia aceptó las responsabilidades derivadas de este Tratado por Cambio de Notas de 1970. Por Cambio de Notas de 7 de Marzo y 29 de Mayo de 1978 se hizo extensivo al Estado de Las Bahamas y por Nota de 11 de Septiembre de 1979 se hizo extensivo a la República de Kirbati.
    Tratado de Extradición suscrito con Paraguay el 22 de Mayo de 1897, promulgado por Decreto de 2 de Octubre de 1928, y publicado en el Diario Oficial de 13 de Noviembre de 1928.
    Tratado de Extradición suscrito con Perú el 5 de Noviembre de 1932, promulgado por Decreto N° 1.152, de 11 de Agosto de 1936, y publicado en el Diario Oficial de 27 de Agosto de 1936.
    Tratado de Extradición suscrito con Uruguay el 10 de mayo de 1897, promulgado por Decreto de 23 de Noviembre de 1909, y publicado en el Diario Oficial de 30 de Noviembre de 1909.
    Tratado de Extradición suscrito con Venezuela el 2 de Junio de 1962, promulgado por Decreto N° 355, de 10 de Mayo de 1965, y publicado en el Diario Oficial de 1° de Junio de 1965.
    Convención sobre Extradición, aprobada en la Séptima Conferencia Panamericana de Montevideo el 26 de Diciembre de 1933, promulgada por Decreto N° 942, de 6 de Agosto de 1935, y publicada en el Diario Oficial de 19 de Agosto de 1935 (Chile con reservas).
    Art. 636. (684) Para que el juez de primera instancia eleve los autos a la Corte Suprema, será necesario que se haya dictado previamente auto firme de prisión o recaído sentencia firme contra el acusado cuya extradición se pretende.
    Deberá también constar en el proceso el país y lugar en que el procesado se encuentre en la actualidad.


    Art. 637. (685) Recibido el proceso por la Corte Suprema, lo pasará en vista al fiscal para que dictamine si es o no procedente la petición de extradición en conformidad a los tratados celebrados con la nación en que el procesado se encontrare refugiado, o en defecto de tratado, con arreglo a los principios del Derecho Internacional.
    Durante la tramitación de la extradición, la Corte Suprema podrá solicitar al Ministerio de Relaciones Exteriores que se pida al Gobierno del país en que se encuentra el procesado, que ordene la detención provisional de éste.


    Art. 638. (686) Oído el Ministerio Público, la Corte verá la causa sin más trámite que ponerla en tabla y en lugar preferente, y resolverá en un auto fundado si debe o no procederse a solicitar la extradición del procesado.


    Art. 639.(687) En caso afirmativo, la Corte Suprema se dirigirá al Ministerio de Relaciones Exteriores, acompañando copia del auto de que se trata en el artículo anterior; y pidiendo que se practiquen las gestiones diplomáticas que sean necesarias para obtener la extradición.
    Acompañará, además, copia autorizada de los antecedentes que hubieren dado mérito para dictar el auto de prisión en contra del procesado, o de la sentencia firme que hubiere recaído en el proceso, si se trata de un procesado rematado.
    Cumplidos estos trámites la Corte Suprema devolverá el expediente al juzgado de origen.



    Art. 640. (688) El Ministerio de Relaciones Exteriores, después de legalizar los documentos acompañados, hará practicar las gestiones necesarias para dar cumplimiento a la resolución de la Corte Suprema; y si obtuviere la extradición, lo hará conducir del país en que se encontrare hasta ponerlo a disposición de aquel tribunal.

    Art. 641. (689) En el caso a que se refiere el artículo precedente, la Corte Suprema ordenará que el inculpado sea puesto a disposición del juez de la causa, a fin de que el jucio siga su tramitación; o de que cumpla su condena, si se hubiere pronunciado sentencia firme.

    Art. 642. (690) Si la Corte Suprema declarare no ser procedente la extradición, o si ésta no fuere acordada por las autoridades de la nación en que el procesado se encuentra refugiado, se devolverá el proceso al juez de la causa para que proceda como lo determina la ley  respecto de los ausentes.


    Art. 643. (691) Si el proceso comprendiere a un procesado que se encuentre en el extranjero y a otros procesados presentes, se observarán las disposiciones anteriores en cuanto al primero, y sin perjuicio de su cumplimiento, seguirá la causa sin interrupción en contra de los procesados presentes. El proceso, en tal caso, será elevado en copia a la Corte Suprema.
    Si el procesado fuere entregado, se observará lo dispuesto en el artículo 602 en cuanto fuere aplicable.



  2. De la extradición pasiva

    Art. 644. (692) Cuando el Gobierno de un país extranjero pida al de Chile la extradición de individuos que se encuentren aquí y que allá estén procesados o condenados a pena, el Ministerio de Relaciones Exteriores transmitirá la petición y sus antecedente a la Corte Suprema.
    Si el Ministerio, a virtud de tratados con la nación requeriente, hubiere hecho arrestar al procesado, lo mandará poner a disposición del Presidente de la misma Corte.



    Art. 645. (693) Recibidos los antecedentes, corresponderá al Presidente de la Corte Suprema conocer en primera instancia de la solicitud de extradición.
    Art. 646. (694) Si los antecedentes dan mérito, se decretará el arresto del procesado. En caso contrario, se recibirá la información que ofrezca el encargado de solicitar la extradición.
    Para decretar el arresto se procederá conforme a lo establecido en el párrafo 2° del Título IV, primera parte del Libro II.


    Art. 647. (695) La investigación se contraerá especialmente a los puntos siguientes:
    1° A comprobar la identidad del procesado;
    2° A establecer si el delito que se le imputa es de aquellos que autorizan la extradición según los tratados vigentes o, a falta de éstos, en conformidad a los principios del Derecho Internacional; y
    3° A acreditar si el sindicado como procesado ha cometido o no el delito que se le atribuye.


    Art. 648. (696) Sin necesidad de información previa acerca de los puntos 2° y 3° determinados en el artículo precedente, se decretará el arresto del procesado una vez  establecida su identidad, siempre que se presentare la sentencia que lo haya condenado o el decreto de prisión expedido en su contra por el tribunal que conozca de la causa, y con tal que el delito imputado sea de aquellos que autoricen la extradición y que el auto de prisión se funde en motivos que hagan presumir la culpabilidad del procesado.


    Art. 649. (697) Aprehendido el procesado, se procederá a tomarle declaración acerca de su identidad y de su participación en el delito que se le imputa. Si en comprobación de sus aseveraciones adujere el testimonio de personas que se encuentren en Chile, el Presidente que instruye el sumario evacuará las citas que creyere conducentes y podrá comisionar al respectivo juez letrado para tomar declaración a los testigos que residieren fuera de la provincia de Santiago.VER NOTA 3.1


    Art. 650. (698) Durante el juicio, no se dará lugar a la libertad provisional.
    Art. 651. (699) Terminada la investigación, se comunicarán los antecedentes al Ministerio Público, quien, en vista de ellos y con arreglo a los tratados o principios del derecho Internacional, pedirá que se otorgue o se deniegue la extradición solicitada.
    Art. 652. (700) De la vista fiscal se dará traslado al procesado por un término prudencial y prorrogable, que en ningún caso podrá exceder de veinte días; y con su contestación, o en su rebeldía, se citará para oír sentencia.
    Si el gobierno requeriente hubiere encargado a alguna persona las gestiones para la extradición, esta persona será oída en primer lugar, en seguida el procesado y el último lugar el Ministerio Público.

    Art. 653. (701) Deberá dictarse sentencia dentro de quinto día, la que se llevará en consulta a la Corte si no es apelada.
    Art. 654. (702) En segunda instancia se mandarán traer los autos en relación con citación del procesado, del fiscal y del encargado por el Gobierno requeriente, si hubiere alguno; y la causa se verá en la forma ordinaria, oyendo el informe oral que quiera emitir cualquiera de dichas personas. Este procedimiento se observará, sea que la revisión se haga por la vía de apelación, sea que se haga por la vía de consulta.

    Art. 655. (703) Cuando la sentencia de la Corte Suprema dé lugar a la extradición, se ordenará por el juez a quo poner el procesado a disposición del Ministerio de Relaciones Exteriores, a fin de que sea entregado al agente diplomático que haya solicitado la extradición.
    Pero si la sentencia deniega la extradición, el mismo juez procederá a poner en libertad al procesado, y la Corte comunicará al Ministerio de Relaciones Exteriores el resultado del juicio, incluyendo copia autorizada de la sentencia que en él hubiere recaído.


    Art. 656. (704) Se mandará sobreser definitivamente en cualquier estado de la causa en que se comunique al tribunal que el Gobierno requeriente desiste de su reclamación.
    Título VII
    DE LA REVISION DE LAS SENTENCIAS FIRMES
    Art. 657. (705) La Corte Suprema podrá rever extraordinariamente las sentencias firmes en que se haya condenado a alguien por un crimen o simple delito, para anularlas, en los casos siguientes:
    1° Cuando, en virtud de sentencias contradictorias, estén sufriendo condena dos o más personas por un mismo delito que no haya podido ser cometido más que por una sola;
    2° Cuando esté sufriendo condena alguno como autor, cómplice o encubridor del homicidio de una persona cuya existencia se compruebe después de la condena;
    3° Cuando alguno esté sufriendo condena en virtud de sentencia que se funde en un documento o en el testimonio de una o más personas, siempre que dicho documento o dicho testimonio haya sido declarado falso por sentencia firme en causa criminal; y
    4° Cuando, con posterioridad a la sentencia condenatoria, ocurriere o se descubriere algún hecho o apareciere algún documento desconocido durante el proceso, que sean de tal naturaleza que basten para establecer la inocencia del condenado.
    Art. 658. (706) El recurso de revisión podrá ser interpuesto, en cualquier tiempo, por el Ministerio Público o por el condenado, su cónyuge, ascendientes, descendientes o hermanos legítimos o naturales. Podrán asimismo interponerlo el condenado que ha cumplido su condena, o los parientes a quienes se acaba de expresar cuando el condenado hubiere muerto y se tratase de rehabilitar su memoria.
    Art. 659. (707) El recurso expresará con precisión su fundamento legal, será firmado por un procurador y un abogado, cuando no sea deducido por el Ministerio Público, y se acompañarán a él los documentos que comprueben los hechos en que se funda.
    Si la causal alegada fuere la del número 2° del artículo 657, el recurso declarará además los medios con que se intenta probar que la persona víctima del pretendido homicidio ha vivido después de la fecha en  que la sentencia la supone fallecida; y si fuere la del número 4° indicará el hecho o el documento desconocido durante el proceso, expresará los medios con que se pretenda acreditar el hecho y se acompañará, en su caso, el documento o, si no fuere posible, se manifestará al menos su naturaleza y el lugar y archivo en que se encuentra.
    El recurso que no se conformare a estas prescripciones será desechado de plano.
    Apareciendo interpuesto el recurso en forma legal, se dará traslado de él al fiscal, o al procesado si el recurrente hubiere sido el Ministerio Público; y en seguida se mandará traer la causa en relación; y, vista en la forma ordinaria, se fallará sin más trámites.


    Art. 660. (708) Si se trata del segundo o cuarto de los casos mencionados en el artículo 657 y se hubiere ofrecido rendir prueba de testigos, el tribunal señalará al efecto un término prudencial y comisionará para recibirla a uno de sus miembros, o al juez de letras del territorio jurisdiccional en que se encuentren los testigos si la comparecencia de éstos ante el tribunal ofreciere graves inconvenientes. Tan pronto como expire el término, serán oídos el procesado y el fiscal, y se mandarán traer los autos en relación sin más trámites, a menos que el tribunal decrete nuevas diligencias para mejor proveer.


    Art. 661. (709) La interposición del recurso de revisión no suspenderá el cumplimiento de la sentencia que se intenta anular, a menos que, por tratarse de una pena irreparable, el tribunal ordene la suspensión hasta que el recurso sea fallado.
    Art. 662. (710) Si el recurso se fundare en el primer motivo de los señalados en el artículo 657, la Corte Suprema, declarando la contradicción entre las sentencias si en efecto existe, anulará una y otra y mandará instruir de nuevo el proceso por el juez que corresponda.
    Art. 663. (711) Si la Corte estimare probado que la persona que se consideraba víctima de homicidio existió después de la fecha en que la supone fallecida la sentencia atacada, anulará ésta.
    Si encontrare mérito, mandará seguir causa por el juez correspondiente.
    Si no hallare mérito para nuevo procedimiento, mandará poner en libertad al condenado rematado.




    Art. 664. (712) La Corte, en fuerza de la sentencia ejecutoriada que declara la falsedad del documento o de la declaración o declaraciones en que se fundó la sentencia condenatoria, anulará ésta, y mandará que el juez competente instruya nuevo proceso en la forma ordinaria.
    En el nuevo proceso no se oirá a los testigos cuyo perjurio declaró la sentencia ejecutoriada.
    Art. 665. (713) Ninguno de los jueces que hubieren intervenido en el pronunciamiento de la sentencia que se declare nula en virtud de las disposiciones del presente título, podrá tomar parte en el nuevo juicio que la Corte Suprema mandare instruir con arreglo a los tres artículos que preceden.
    Art. 666. (714) En el nuevo proceso, los jueces deberán aplicar la ley aunque la pena sea mayor a la impuesta por la sentencia anulada.
    En este caso, siendo posible, se descontará de la nueva pena la que el condenado llevaba sufrida a consecuencia de la condena anterior.



    Art. 667. (715) Si la sentencia de la Corte Suprema o la que pronunciare el tribunal llamado a conocer de la nueva causa, declara haber sido probada satisfactoriamente la completa inocencia del acusado, podrá éste exigir que dicha sentencia se publique en el Diario Oficial, y que se le devuelvan por quien las hubiere percibido, las sumas que haya pagado en razón de costas e indemnización de perjuicios en cumplimiento de la sentencia anulada.
    El mismo derecho corresponderá a los herederos del condenado que hubiere fallecido.
    Título VIII
    DEL PROCEDIMIENTO EN CASO DE PERDIDA DE PROCESOS
CRIMINALES
    Art. 668. (716) Si desaparece un expediente de juicio criminal, el juez competente procederá inmediatamente a las investigaciones para encontrarlo; y si es del caso, instruirá un sumario para castigar al culpable.
    Art. 669. (717) Si el expediente no parece dentro de los diez días siguientes, el juez de la causa comenzará de nuevo la instrucción del proceso aprovechando aquellas piezas de que exista copia fidedigna y procediendo en lo demás en la forma ordinaria.
    El tribunal podrá de plano tener como auténticas las copias simples de cualquiera pieza del proceso, timbradas por el secretario.
    En las comunas o agrupación de comunas en donde existan varios jueces con jurisdicción en lo criminal, el juez que tramitó la causa conocerá del proceso por desaparición del expediente.


    Art. 670. (718) Siempre que se pierda una parte de un proceso criminal, el juez procederá respecto de las piezas desaparecidas en la forma que se indica en los dos artículos precedentes, y suspenderá, si es preciso, el curso del negocio principal.
    Art. 671. (719) Si en el proceso ha recaído sentencia firme que se conserve original o en copia auténtica, se la cumplirá, sin perjuicio de practicarse las indagaciones e instruirse el sumario a que se refiere el artículo 668.
  LIBRO CUARTO
  DEL CUMPLIMIENTO Y EJECUCION
  Título I 
  DEL DESTINO DE LAS ESPECIES
  Párrafo I
  De las especies decomisadas
    Artículo 672.- El comiso de los instrumentos y efectos del delito se declarará en la sentencia, según lo previsto en el artículo 504. Si no se hubiere resuelto en ella, se podrá decretar en cualquier tiempo, mientras existan las especies en poder del tribunal. Los incidentes o recursos a que diere lugar dicha decisión se tramitarán en cuaderno separado y no afectarán al fallo ni entorpecerán su cumplimiento.

    Artículo 673.- Las armas de fuego, municiones, explosivos y demás elementos a que se refiere la Ley sobre Control de Armas que sean decomisados, se remitirán a la autoridad que señala esa misma ley.
    Las demás especies decomisadas se pondrán a disposición del Fisco, para los efectos establecidos en el artículo 60 del Código Penal. Esta autoridad podrá ordenar la destrucción de las que no tuvieren valor y no fueren utilizables.
    Los dineros y otros valores decomisados en favor del Fisco se destinarán a beneficio de la Junta de Servicios Judiciales.
    En los casos de los artículos 366 quinquies, 374 bis, inciso primero y 374 ter del Código Penal, el tribunal destinará los instrumentos tecnológicos decomisados, tales como computadores, reproductores de imágenes o sonidos y otros similares, al Servicio Nacional de Menores o a los departamentos especializados en la materia de los organismos policiales que correspondan. Las producciones incautadas como pruebas de dichos delitos podrán destinarse al registro reservado a que se refiere el inciso segundo del artículo 369 ter del Código Penal. En este caso, la vulneración de la reserva se sancionará de conformidad con lo dispuesto en el párrafo 8 del Título V, del Libro II del Código Penal.

    Artículo 674.- Tratándose de especies corruptibles o perecibles el juez las pondrá a disposición de un martillero para que proceda a su venta directa o subasta.
    Si se decretare el comiso, se hará efectivo sobre el producto de la enajenación.
    Si en definitiva no fuere procedente el comiso, se entregará el producto de la enajenación a quien corresponda.

  Párrafo 2
  De las especies retenidas y no decomisadas


    Artículo 675.- La especies no decomisadas retenidas que se encuentren a disposición del tribunal y que no hayan sido reclamadas, se subastarán de acuerdo con la ley N° 12.265, una vez transcurridos seis meses a lo menos desde la fecha en que recayó resolución firme poniendo término al proceso. Si el sobreseimiento fuere temporal, este plazo será de un año.
    Tratándose de especies corruptibles o perecibles se aplicará lo dispuesto en el artículo 674.

    Artículo 676.- En la subasta de especies de venta controlada se estará a lo establecido en los reglamentos Art. decimo vigentes.


    Artículo 677.- Los dineros puestos a disposición de los tribunales que no caigan en comiso ni hayan sido reclamados dentro de los plazos señalados en el artículo 675, se girarán a la orden de la Junta de Servicios Judiciales para sus fines.

    Artículo 678.- En el mes de junio de cada año, los secretarios de juzgados presentarán a la respectiva Corte de Apelaciones un informe detallado sobre el destino dado a las especies que hayan sido puestas a disposición del tribunal.

    Artículo 679.- Las disposiciones de este Título se aplicarán en defecto de normas especiales relativas a las especies decomisadas o a las otras materias contenidas en él.

  TITULO II
  DE LAS COSTAS


    Artículo 680.- Cuando el procesado sea absuelto o sobreseído definitivamente, el querellante será condenado en costas, a menos que haya tenido motivo plausible para interponer la acción penal.
    Artículo 681.- Cuando sean varios los condenados al pago de costas, el tribunal fijará la parte o proporción que corresponda a cada uno.


        TITULO III
  DE LAS MEDIDAS APLICABLES A LOS ENAJENADOS
        MENTALES
    1.- Del enajenado mental que delinque

    Artículo 682.- Cuando el acusado absuelto o sobreseído definitivamente por estar exento de responsabilidad criminal en virtud de la causal del número 1° del artículo 10 del Código Penal, sea un enajenado mental cuya libertad constituya un peligro, en los términos señalados en el artículo 688, el tribunal dispondrá en la sentencia que se le aplique, como medida de seguridad y protección, la de internación en un establecimiento destinado a enfermos mentales.
    En caso contrario, ordenará que sea entregado bajo fianza de custodia y tratamiento en la forma señalada en el artículo 692.
    Y si la enfermedad ha desaparecido o no requiere tratamiento especial, será puesto en libertad sin condiciones.

    Artículo 683.- No obstante, si la absolución o el sobreseimiento favorecen a un procesado que al tiempo de cometer el delito era enajenado mental, pero se funda en un motivo diverso de la exención de responsabilidad criminal establecida en el número 1° del artículo 10 del Código Penal, se le pondrá a disposición de la autoridad sanitaria si su libertad constituye riesgo, y si no lo constituye se le dejará libre.



  2.- Del procesado que cae en enajenación


    Artículo 684.- Si después de cometido el delito cayere el imputado en enajenación mental, se continuará la instrucción del sumario hasta su terminación; y si no procediere sobreseimiento en la causa o en su favor, el juez decidirá si continúa o no el procedimiento, teniendo en consideración, para resolver, la naturaleza del delito y la de la enfermedad. Para este efecto, el tribunal podrá pedir informe al médico legista.
    El mismo procedimiento se aplicará cuando la enajenación mental sobrevenga en cualquier momento antes de dictarse la sentencia de término.

    Artículo 685.- Cuando, en los casos del artículo anterior, se ordenare la continuación del procedimiento, se estará a lo previsto en los artículos 682 y 683 si resultare absuelto, o a lo establecido en el artículo 687 si fuere condenado a penas privativas o restrictivas de libertad.

    Artículo 686.- Si se resuelve que no se continúe el procedimiento contra un enfermo mental incurable, se dictará en su favor sobreseimiento definitivo, poniéndolo a disposición de la autoridad sanitaria si su libertad constituye un peligro, y en caso contrario se ordenará su libertad.
    Se dictará sobreseimiento temporal, si la enfermedad es curable, para continuar el proceso una vez que el procesado recupere la razón. Al procesado cuya libertad constituya un peligro y a aquél a quien podría corresponder una pena probable mínima no inferior a cinco años y un día de privación o restricción de libertad, se les recluirá entre tanto en un establecimiento para enfermos mentales, en los demás casos se entregará el procesado bajo fianza de custodia y tratamiento.


    Artículo 687.- Si después de la sentencia condenatoria cayere el condenado en enajenación mental, dictará el juez una resolución fundada declarando que no se deberá cumplir la sanción restrictiva o privativa de libertad. El condenado cuya libertad constituya peligro será puesto a disposición de la autoridad sanitaria. Aquél cuya libertad no constituya riesgo será entregado bajo fianza de custodia y tratamiento, siempre que la pena o penas aplicadas constituyan en conjunto una privación o restricción de libertad por más de cinco años, si es inferior la condena, se le pondrá en libertad.
    Siendo curable la enfermedad, se suspenderá el cumplimiento de la sentencia en una resolución fundada, hasta que el enajenado recupere la razón. El condenado cuya libertad constituya riesgo, y el que, sin estar en tal caso, haya sido condenado a penas superiores a cinco años de restricción o privación de libertad, será internado en un establecimiento para enfermos mentales; en las demás situaciones será entregado bajo fianza de custodia o tratamiento de acuerdo a lo dispuesto en el artículo 692.
    En cualquier tiempo que el enfermo mental recupere la razón se hará efectiva la sentencia si no hubiere prescrito la pena. Si ella le impusiere privación o restricción de la libertad, se imputará a su cumplimiento el tiempo que haya durado la enajenación mental.

  3.- Reglas comunes


    Artículo 688.- Para los fines previstos en este Título, se entenderá por enajenado mental cuya libertad constituye peligro, aquel que como consecuencia de su enfermedad pueda atentar contra sí mismo o contra otras personas, según prognosis médico legal.

    Artículo 689.- Todo informe psiquiátrico decretado en la causa, además de contener las conclusiones referentes a la salud mental del reo, deberá indicar concretamente si éste debe o no ser considerado un enajenado mental, si la enfermedad es o no curable, si su libertad representa un peligro según lo dicho en el artículo precedente y, en general, las modalidades del tratamiento a que deba ser sometido.




NOTA
      El artículo 9° de la Ley 19047, publicada el 14.02.1991, modificado por la Ley 19158, otorga facultad para mantener la palabra reo por estar empleada en sentido genérico.
    Artículo 690.- Para adoptar las medidas a que se refiere este título, se requerirá de un informe del establecimiento donde el reo  hubiere permanecido internado o privado de libertad durante el proceso, sobre la anormalidad o normalidad de su comportamiento, informe que se evacuará oyendo al médico del plantel, si lo hubiere. En todo caso, se exigirá el dictamen de un perito, por lo menos, sea que haya informado durante la tramitación de la causa o que lo haga especialmente para la determinación de la medida aplicable.



NOTA
      El artículo 9° de la Ley 19047, publicada el 14.02.1991, modificado por la Ley 19158, otorga facultad para mantener la palabra reo por estar empleada en sentido genérico.
    Artículo 691.- La medida de seguridad y protección de internación de un enajenado mental deberá cumplirse en un establecimiento destinado a enfermos mentales, y se llevará a efecto en la forma y condiciones que establezca el juez.

    Artículo 692.- Cuando se decrete como medida de seguridad y protección la custodia y tratamiento de un enfermo mental, se dispondrá su entrega a su familia, a su guardador, o a alguna institución pública o particular de beneficencia, socorro o caridad. El juez fijará las condiciones de la custodia y controlará que se realice el tratamiento médico a que deba ser sometido, pudiendo exigir informaciones periódicas. Podrá también exigir fianza de que serán cumplidas las condiciones impuestas.

    Artículo 693.- La internación como medida de seguridad sólo podrá durar mientras subsistan las condiciones que la hicieron necesaria, y no podrá extenderse más allá de la sanción restrictiva o privativa de libertad prescrita en la sentencia, o del tiempo que corresponda a la pena mínima probable, el que será señalado por el juez en el fallo.
    Se entiende por pena mínima probable, para estos efectos, el tiempo mínimo de privación o restricción de libertad que la ley prescriba para el delito o delitos por los cuales se ha procesado o acusado al procesado.
    Sin embargo, cumplido el plazo de internación, el procesado pasará a disposición de la autoridad sanitaria, si su libertad constituye riesgo.



    Artículo 694.- La entrega del enfermo mental a disposición definitiva de la autoridad sanitaria, termina con todo control o responsabilidad de las autoridades judiciales o penitenciarias sobre su persona.
    La autoridad sanitaria será el Servicio de Salud correspondiente o la que determinen las leyes sobre la materia.
    Dicha autoridad no podrá negarse a recibir al procesado respecto de quien se haya declarado que su libertad constituye riesgo, para el efecto de disponer de él como fuere procedente según sus facultades legales y reglamentarias. A partir de ese momento, no podrá quedar el procesado en ningún establecimiento carcelario o penitenciario, a menos que cuente con dependencias que le permitan mantenerlo transitoriamente bajo el régimen dispuesto por la autoridad sanitaria.


    Artículo 695.- Cuando el proceso penal no pueda proseguirse por enajenación mental del imputado, la acción civil que no hubiere sido intentada sólo podrá ser ejercida ante el juzgado civil. Si dicha acción ya hubiere sido interpuesta en el proceso penal, continuará su ejercicio en él hasta la dictación y el cumplimiento de la sentencia que resuelva la demanda civil.

    Artículo 696.- Los fiscales de las Cortes de  x Apelaciones deberán inspeccionar periódicamente los establecimientos especiales y carcelarios donde se encuentren internados enajenados mentales. La visita la practicarán por lo menos cada tres meses y los jefes de los respectivos establecimientos emitirán un informe relativo a todas las personas que allí se encuentren por orden judicial, indicando el proceso, el juzgado de origen y los datos esenciales relativos al estado actual de la enfermedad de cada internado.
    Los fiscales remitirán al jefe del Ministerio Público un informe sobre las condiciones del local y de la atención de los enfermos, con una copia de la nómina indicada. El fiscal de la Corte Suprema deberá dirigirse a las autoridades judiciales y administrativas representando las deficiencias que se hayan observado.
    Deberán también los fiscales de las Cortes de Apelaciones, de oficio o a petición de parte, solicitar las medidas judiciales tendientes a poner remedio a todo error, abuso o deficiencia que se observe en la tramitación, en las medidas adoptadas o en su cumplimiento, en lo que se refiere a los enfermos mentales.

    Santiago, a treinta de agosto de mil novecientos cuarenta y cuatro.- JUAN ANTONIO RIOS M.- Oscar Gajardo V.
